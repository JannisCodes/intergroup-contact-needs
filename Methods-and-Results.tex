\faQuestionCircle~Unsure about order of study one and two. Currently
chronological. But could also be: (1) main model works in larger student
sample, (2) works in economic migrant sample, (3) test full thing in
medical sample.

\section{Study 1}

Based on our main hypotheses, the aim of our first study is to
specifically test the general contact hypothesis, the influence of core
need fulfillment, and perceived interaction quality during intergroup
contacts. To this aim, we conducted an intensive longitudinal survey
study with recent migrants to the Netherlands, gathering a large body of
ecologically valid data on need satisfaction in real-life intergroup
contact situations. Data was collected from May 5th through June 6th,
2018 (and all participants started the study within the first two days).

The full surveys are available in our Online Supplementary Material A
and the full data description is available in Online Supplementary
Materials B.

\subsection{Methods}

\subsubsection{Participants}

After receiving ethical approval from the University of Groningen, we
recruited 23 migrants using the local paid participant pool and
specifically targeted non-Dutch migrants to participate in our study.
Participants reported on their interactions for at least 30 days with
two daily measures (capturing the morning and afternoon). With this
design, we aimed at getting 50-60 measurements per participant
(\textit{M} = 53.26, \textit{M} = 16.72, \textit{total N} = 1225). This
is a common number of measurements found in experience sampling studies
and should offer sufficient power to model processes within and between
participants \citep[e.g., for a systematic review see][]{AanhetRot2012}.
Participants were compensated for their participation with up to 34
Euros -- each two Euros for pre- and post-questionnaire as well as 50
Eurocents for every experience sampling measurement occasion. The sample
consisted of relatively young, educated, and western migrants from the
global north (\(M_{age}\) = 24.35, \(SD_{age}\) = 4.73, 19 women, 15
students). The sample accurately describes one of the largest groups of
migrants in the region \citep[][]{GemeenteGroningen2015}.

\subsubsection{Procedure}

The study itself consisted of three main parts, an introductory
pre-measurement, and the daily experience sampling measurements, as well
as a concluding post-measurement. After giving informed consent,
participants started by filling in an online pre-questionnaire assessing
demographics and general information about their immigration. Over the
next thirty days, the participants then were invited twice a day (at 12
pm and 7pm) to reflect upon their interactions, psychological need
fulfillments, and current attitudes towards the Dutch outgroup
(\textit{median duration} = 22, \textit{MAD duration} = 19). After the
final day of daily diary measurements, participants were invited to fill
in a longer post measurement survey that mirrored the pre-measurement.
All key variables in for this study were part of the short daily diary
surveys.

\subsubsection{Materials}

\paragraph{Intergroup Contact}

To test the prerequisite effect of intergroup contact, every `Experience
Recap' started with the question
``\textit{Did you meet a Dutch person this morning [/afternoon]? (In person interaction for at least 10 minutes)}''

\paragraph{Intergroup Contact}

To test the prerequisite effect of intergroup contact, every `Experience
Recap' started with the question
``\textit{Did you meet a Dutch person this morning [/afternoon]? (In person interaction for at least 10 minutes)}''

\subsection{Results}

\section{Study 2}

\subsection{Methods}

\subsection{Results}

\section{Study 2}

\subsection{Methods}

\subsection{Results}
