\faQuestionCircle~Unsure about order of study one and two. Currently
chronological. But could also be: (1) main model works in larger student
sample, (2) works in economic migrant sample, (3) test full thing in
medical sample.

\section{Study 1}

Based on our main hypotheses, the aim of our first study is to
specifically test the general contact hypothesis, the influence of core
need fulfillment, and perceived interaction quality during intergroup
contacts. To this aim, we conducted an intensive longitudinal survey
study with recent migrants to the Netherlands, gathering a large body of
ecologically valid data on need satisfaction in real-life intergroup
contact situations. Data was collected from May 5th through June 6th,
2018 (and all participants started the study within the first two days).

The full surveys are available in our Online Supplementary Material A
and the full data description is available in Online Supplementary
Materials B. Correlations and descriptive statistics of the included
variables are available in Table \ref{tab:workerVarDescr} and Table
\ref{tab:workerOutVarDescr}.

\subsection{Methods}

\subsubsection{Participants}

After receiving ethical approval from the University of Groningen, we
recruited 23 migrants using the local paid participant pool and
specifically targeted non-Dutch migrants to participate in our study.
Participants reported on their interactions for at least 30 days with
two daily measures (capturing the morning and afternoon). With this
design, we aimed at getting 50-60 measurements per participant
(\textit{M} = 53.26, \textit{SD} = 16.72, \textit{total N} = 1225). This
is a common number of measurements found in experience sampling studies
and should offer sufficient power to model processes within and between
participants \citep[e.g., for a systematic review see][]{AanhetRot2012}.
Participants were compensated for their participation with up to 34
Euros -- each two Euros for pre- and post-questionnaire as well as 50
Eurocents for every experience sampling measurement occasion. The sample
consisted of relatively young, educated, and western migrants from the
global north (\(M_{age}\) = 24.35, \(SD_{age}\) = 4.73, 19 women, 15
students). The sample accurately describes one of the largest groups of
migrants in the region \citep[][]{GemeenteGroningen2015}.

\subsubsection{Procedure}

The study itself consisted of three main parts, an introductory
pre-measurement, and the daily experience sampling measurements, as well
as a concluding post-measurement. After giving informed consent,
participants started by filling in an online pre-questionnaire assessing
demographics and general information about their immigration. Over the
next thirty days, the participants then were invited twice a day (at 12
pm and 7pm) to reflect upon their interactions, psychological need
fulfillments, and current attitudes towards the Dutch outgroup
(\textit{median duration} = 22, \textit{MAD duration} = 19). General
compliance was high (85.90\% of all invited surveys were filled
in)\footnote{Two participants completed only two days (among the others, participation was 93.70\%)}.
The response rates were approximately equal during mornings (\textit{n}
= 621) and afternoons (\textit{n} = 604) and most measurements were
completed within four hours of the invitation. After the final day of
daily diary measurements, participants were invited to fill in a longer
post measurement survey that mirrored the pre-measurement. All key
variables in for this study were part of the short daily diary surveys.

\subsubsection{Materials}

\paragraph{Intergroup Contact}

To test the prerequisite effect of intergroup contact, every experience
sampling measurement started with the question
``\textit{Did you meet a Dutch person this morning [/afternoon]? (In person interaction for at least 10 minutes)}''.
Our participants recorded between 2--51 (3.23--91.07\% of individual
daily diary measurements; 31.59\% of all 1225 daily diary
responses)\footnote{Two participants only recorded two daily diary measurements each and non of these included outgroup contacts. These participants are removed from any analyses including outgroup contacts.}.

\paragraph{Psychological Needs}

Irrespective of whether participants had an interaction with Dutch
people or not, everyone answered a short series of questions on
psychological need fulfillment. However, whereas participants with
interactions reported on the need fulfillment during the interaction,
people without interactions with Dutch people judged the past daytime
period in general. To assess the fulfillment of psychological needs, we
included two types of need measurement: (1) the core situational need
and (2) general self-determination theory needs.

For the core situational need, we asked participants in an open ended
text field:
``\textit{What was your most important goal [during the interaction / this morning / this afternoon]?}''.
Then, with reference to the text entry, we asked how much this core need
was fulfilled during the interaction or the past daytime period:
``\textit{[The interaction / you] fulfilled your goal: [-previous text entry-]}''
on a continuous slider scale ranging from strongly disagree (-50) to
strongly agree (+50).

We, additionally, included a common measure of three self-determination
theory needs \citep[see][]{Downie2008}. The items were introduced either
by ``\textit{During the interaction:}'' or
``\textit{This morning [/afternoon]:}'' and measured autonomy
(``\textit{I was myself.}''), competence
(``\textit{I felt competent.}''), and relatedness (without intergroup
contact ``\textit{I had a strong need to belong}''; with intergroup
contact: ``\textit{I shared information about myself.}'' and
``\textit{The other(s) shared information about themselves.}''). All
items were rated on a continuous slider scale from very little (-50) to
a great deal (+50).

\paragraph{Perceived Interaction Quality}

As an explanatory mechanism, we assessed ratings of the perceived
interaction quality. As our main measurement, participants rated the
statement ``\textit{Overall the interaction was …}'' on two continuous
slider scales measuring pleasantness
\citep[from unpleasant (-50) to pleasant (+50)) and meaningfulness (from superficial (-50) to meaningful (+50); both items adapted from][]{Downie2008}.

\paragraph{Outgroup Attitudes}

At the end of every daily diary measurement we asked all participants
about their current attitudes towards the Dutch -- our main dependent
variable. To assess the momentary outgroup evaluation we used the common
feeling thermometer: ``How favorable do you feel towards the Dutch?''
\citep[][]{Lavrakas2008}. Participants then rated their attitude on a
continuous slider scale from ``very cold -- 0'' through ``no feeling --
50'' to ``very warm -- 100''. Both the question phrasing as well as the
tick labels were consistent with large-scale panel surveys
\citep[e.g.,][]{DeBell2010}.

\begin{table}
\begin{minipage}[t][\textheight][t]{\textwidth}

\caption{\label{tab:workerVarDescr}Worker: Multilevel Core Variable Descriptives}
\centering
\resizebox{\linewidth}{!}{
\begin{tabular}[t]{llccccc}
\toprule
  & Core Need & Competence & Autonomy & Relatedness & Quality & Attitudes NL\\
\midrule
Core Need &  & 0.82*** & 0.60*** & 0.33 & 0.52** & -0.03\\
Competence & 0.36*** &  & 0.89*** & 0.26 & 0.39 & -0.23\\
Autonomy & 0.28*** & 0.22*** &  & 0.31 & 0.57** & 0.02\\
Relatedness & 0.50*** & 0.39*** & 0.37*** &  & -0.07 & 0.14\\
Quality & 0.17*** & 0.44*** & 0.27*** & 0.26*** &  & 0.50*\\
Attitudes NL & 0.24*** & 0.36*** & 0.24*** & 0.37*** & 0.52*** & \\
\addlinespace
Grand Mean & 27.95 & 12.10 & 22.17 & 5.29 & 24.10 & 71.49\\
Between SD & 14.68 & 13.72 & 12.09 & 14.59 & 9.50 & 12.91\\
Within SD & NA & NA & NA & NA & NA & NA\\
ICC(1) & 0.29 & 0.28 & 0.38 & 0.28 & 0.18 & 0.70\\
ICC(2) & 0.96 & 0.95 & 0.97 & 0.95 & 0.79 & 0.99\\
Within.person.SD & 20.83 & 20.89 & 15.15 & 23.29 & 18.01 & 8.11\\
\bottomrule
\multicolumn{7}{l}{\rule{0pt}{1em}\textit{Note: }}\\
\multicolumn{7}{l}{\rule{0pt}{1em}Upper triangle: Between-person correlations;}\\
\multicolumn{7}{l}{\rule{0pt}{1em}Lower triangle: Within-person correlations;}\\
\multicolumn{7}{l}{\rule{0pt}{1em}*** p < .001, ** p < .01,  * p < .05}\\
\end{tabular}}
\end{minipage}
\end{table}

\begin{table}
\begin{minipage}[t][\textheight][t]{\textwidth}

\caption{\label{tab:workerOutVarDescr}Worker: Multilevel Core Variable Descriptives (Outgroup Contact Only)}
\centering
\resizebox{\linewidth}{!}{
\begin{tabular}[t]{llcc}
\toprule
  & Core Need & Quality & Attitudes NL\\
\midrule
Core Need &  & 0.40 & -0.03\\
Quality & 0.37*** &  & 0.21\\
Attitudes NL & 0.27*** & 0.55*** & \\
\addlinespace
Grand Mean & 82.20 & 67.00 & 72.46\\
Between SD & 12.42 & 9.26 & 13.62\\
Within SD & 17.66 & 18.24 & 9.50\\
\addlinespace
ICC(1) & 0.33 & 0.23 & 0.68\\
ICC(2) & 0.90 & 0.84 & 0.98\\
\bottomrule
\multicolumn{4}{l}{\rule{0pt}{1em}\textit{Note: }}\\
\multicolumn{4}{l}{\rule{0pt}{1em}Upper triangle: Between-person correlations;}\\
\multicolumn{4}{l}{\rule{0pt}{1em}Lower triangle: Within-person correlations;}\\
\multicolumn{4}{l}{\rule{0pt}{1em}*** p < .001, ** p < .01,  * p < .05}\\
\end{tabular}}
\end{minipage}
\end{table}


\subsection{Results}

\subsubsection{Contact Hypothesis}

We tested the most general contact hypothesis in two steps. First, we
assessed whether more intergroup interactions were related to to more
positive outgroup attitudes. Second, we tested whether a potential
positive effect on outgroup attitudes depended on the interaction
quality (jointly with the number of interactions). We find that neither
the number of interactions nor the number of daily diary responses with
an interaction were significantly related with the average outgroup
attitudes. This is to say that within our data, participants with more
outgroup interactions did not have significantly more positive outgroup
attitudes. This might be due to the aggregation within the participants
or the small sample size of between participant data. Nonetheless, the
aggregated data does not support the notion that simply having more
interactions with an outgroup results in more positive outgroup
attitudes.

However, despite the missing relationship with the number of
interactions, we find a medium sized correlation between the
participants' Average Interaction Quality and their Average Outgroup
Attitudes. Thus within our data participants with a higher quality
outgroup interactions also held more positive attitudes towards that
group. And when considering the number of interactions and average
interaction quality jointly in a linear regression, we additionally find
a statistically significant interaction term (\textit{b} = -0.05,
\textit{t}(17) = -2.95, \textit{p} = 0.009, \(\eta_p^2\) = 0.34).
Looking at a floodlight analysis of the effect, we find that in our
sample with an increasing number of interactions the positive effect of
average interaction quality becomes weaker. However, it should be noted
that this is based on data aggregating all within participant nuances
and is only the date of 21 people.

We additionally used a multilevel regression to check whether having an
interaction with an outgroup member had a situational (i.e.,
contemporaneous) effect within the participants. We find that having an
outgroup interaction is indeed associated with significantly more
positive outgroup attitudes within the participants (\textit{b} = 2.48,
\textit{t}(1200) = 4.36, \textit{p} \textless{} .001,
\textit{95\%CI}{[}1.36, 3.59{]}), even after controlling for having an
interaction with a non-Dutch (which did not relate to outgroup attitudes
independently; For full results see Online Supplementary Materials
B)\footnote{Interestingly, adding random slopes to this model did not significantly add explained variance. This is unusual because this might indicate the the effect is very consistent across participants. However, this might also be the case due to a small number of participants, or other measurement issues.}.
Thus, in our first data we find mixed results, where outgroup contacts
show a positive effect on outgroup attitudes in the moment but these
results do not translate into average differences between participants,
keeping in mind that the between-participant analyses in the aggregated
form are based in a small sample size.

\subsubsection{Core Need}

The main proposal of our article is that the success of an outgroup
interaction might be explained by whether or not the interaction
fulfilled the person's core situational need. This should, in turn, be
due to a higher perceived interaction quality. We will this sequentially
test whether the fulfillment of the core need during an interaction is
(1) related to more positive outgroup attitudes, (2) higher perceived
interaction quality, and (3) whether the variance explained by the core
need is assumed by the perceived interaction quality if considered
jointly.

We find that in the highest multilevel models, the fulfillment of core
situational needs during outgroup contacts was associated with more
positive outgroup attitudes (random slopes model; \textit{b} = 0.17,
\textit{t}(365) = 2.93, \textit{p} = 0.004, \textit{95\%CI}{[}0.06,
0.30{]}) and also predicted higher perceived interaction quality (random
intercept model; \textit{b} = 0.42, \textit{t}(365) = 8.07, \textit{p}
\textless{} .001, \textit{95\%CI}{[}0.32, 0.52{]}). However, if we
consider the influences of core need fulfillment and interaction quality
on outgroup attitudes jointly, we find that virtually all variance is
explained by perceived interaction quality (random slopes mode;
\textit{b} = 0.23, \textit{t}(364) = 4.22, \textit{p} \textless{} .001,
\textit{95\%CI}{[}0.12, 0.33{]}) and only little unique variance is
still explained by core need fulfillment (\textit{b} = 0.04,
\textit{t}(364) = 0.64, \textit{p} = 0.523, \textit{95\%CI}{[}-0.08,
0.15{]}). We thus find support for our hypotheses and can conclude that
in this data set the fulfillment of core situational needs had a
significant influence on outgroup attitudes and that this effect is
likely explained by its effect through perceived interaction quality.

\faQuestionCircle~Are we sure that we don't want some sort of
bootstrapped indirect effect?

\subsubsection{Robustness}

To build further confidence in our results, we assessed two additional
models that might offer alternative explanations of the effects we find.
First, to make certain that the effect of core need fulfillment is
specific to the interaction we compare the the effect to fulfillment of
the situation core need when no outgroup interaction took place. Here we
go back to the full dataset and assess the generalized situational core
need fulfillment (either during an interaction or about the daytime in
general) and whether the effect differed during daily diary measurements
with and without outgroup contacts. We find that there is no main effect
of core need fulfillment (random slopes model; \textit{b} = 0.03,
\textit{t}(1199) = 1.07, \textit{p} = 0.284, \textit{95\%CI}{[}-0.02,
0.08{]}) but a significant interaction effect of core need fulfillment
and outgroup contact (\textit{b} = 0.13, \textit{t}(1199) = 4.31,
\textit{p} \textless{} .001, \textit{95\%CI}{[}0.07, 0.19{]}). Together
with a significant main effect of having an outgroup contact, this
indicates that it is not key need fulfillment in general --- but only
key need fulfillment during an outgroup contact that predicts more
positive outgroup attitudes.

In a final step we check whether during the interaction the core
situational need is a meaningful predictor even when taking other
fundamental psychological needs into account. We focus on the three
commonly considered self determination needs: competence, autonomy, and
relatedness. We find that the core need adds significantly above a model
with only the self determination theory needs (random slopes models;
\(\chi^2\)(6, \textit{N} = 21), \textit{p} = 0.003). We find that next
to relatedness (\textit{b} = 0.10, \textit{t}(361) = 4.04, \textit{p}
\textless{} .001, \textit{95\%CI}{[}0.05, 0.14{]}), the core need
explains the most variance in outgroup attitudes after an outgroup
contact (\textit{b} = 0.09, \textit{t}(361) = 2.23, \textit{p} = 0.026,
\textit{95\%CI}{[}0.02, 0.17{]}). When compared to the model with only
the SDT needs, the core need fulfillment flexibly takes on some of the
explained variance of all of the three fundamental needs and competence
and autonomy needs are non-significant (all \textit{b} \textless{} 0.05,
all \textit{p} \textgreater{} 0.544). For full results see Online
Supplementary Information B.

\section{Study 2}

The aim of our study is similar to to that of the first study, as we
again test the general contact hypothesis, the influence of core need
fulfillment, and perceived interaction quality during intergroup
contacts. However, in this second study we collected a substantially
larger intensive longitudinal survey study with recently arrived
international students in the Netherlands. The survey method again
offers a large body of ecologically valid data on need satisfaction in
real-life intergroup contact situations as these students will likely
interact with the Dutch majority outgroup on a daily basis. Data was
collected from November 19th, 2018 through January 6th, 2019.

The full surveys are available in our Online Supplementary Material A
and the full data description is available in Online Supplementary
Materials B. Correlations and descriptive statistics of the included
variables are available in Table \ref{tab:studentVarDescr} and Table
\ref{tab:studentOutVarDescr}.

\subsection{Methods}

\subsubsection{Participants}

After receiving ethical approval from the University of Groningen, we
recruited 113 international students using a local participant pool and
specifically targeted non-Dutch students, who had recently arrived in
the Netherlands. Participants reported on their interactions for at
least 30 days with two daily measures (capturing the morning and
afternoon). With this design, we aimed at getting 50-60 measurements per
participant (\textit{M} = 43.94, \textit{SD} = 15.00, \textit{total N} =
4965). As with the previous study this should offer sufficient power to
model processes within participants and will lend stronger weight to
between-participant results. Participants were compensated for their
participation with partial course credits --- depending on their
participation. The sample consisted of relatively young migrants, who
were mostly from the global north (\(M_{age}\) = 20.24, \(SD_{age}\) =
2.12, 84 women). The sample fairly accurately describes the local
population of international students.

\subsubsection{Procedure}

The study procedure mirrored the setup of study one and consisted of
pre-, daily diary-, and post-measurement. The participants were invited
for daily diary measurements twice a day (at 12 pm and 7pm) for 30 days
(\textit{median duration} = 46, \textit{MAD duration} = 54). General
compliance was high (70.87\% of all invited surveys were filled in). The
response rates were approximately equal during mornings (\textit{n} =
2608) and afternoons (\textit{n} = 2357). All key variables in for this
study were part of the short daily diary surveys.

\subsubsection{Materials}

\paragraph{Intergroup Contact}

To measure intergroup contacts, every experience sampling measurement
started with the question
``\textit{Did you meet a Dutch person this morning [/afternoon]? (in-person interaction for at least 10 minutes)}''.
Participants were additionally offered a pop-up explanation: ``With
in-person interaction, we mean a continued interaction with another
person (potentially in a group) that lasted at least 10 minutes. This
interaction should be offline and face-to-face. It should include some
form of verbal communication and should be uninterrupted to still count
as the same interaction. Any individual interaction can last minutes or
hours. If there were multiple interaction partners, we would like you to
focus on the person that was most important to you during the
interaction.''. The participants recorded between 1--43 (1.64--93.75\%
of individual daily diary measurements; 18.83\% of all 4965 daily diary
responses).

\paragraph{Psychological Needs}

For the core situational need, we asked participants in an open ended
text field:
``\textit{What was your main goal [during the interaction with X / this morning / this afternoon]?}''.
Participants could additionally click on a pop-up explanation: ``Your
main goal during an interaction can vary depending on the interaction.
It could be to connect with friends, to find or provide help, to achieve
academic ambitions, work on your fitness, work for a job, or simply to
get a coffee, just as well as many many other concrete or abstract goals
that are import to you in the moment. It really depends on your
subjective experience of the interaction.''. Then, with reference to the
text entry, we asked how much this core need was fulfilled during the
interaction or the past daytime period:
``\textit{During your interaction with X [this morning / this evening] your goal (-previous text entry-) was fulfilled.}''
on a continuous slider scale ranging from strongly disagree (1) to
strongly agree (100).

The measurement of the self determination needs was identical to study
1.

\paragraph{Perceived Interaction Quality}

The ratings of the perceived interaction quality was identical to study
one..

\paragraph{Outgroup Attitudes}

Attitudes towards the Dutch majority outgroup was again measured using
the feeling thermometer, as in study one. All survey details are also
available in Online Supplemental Materials A and B.

\begin{table}
\begin{minipage}[t][\textheight][t]{\textwidth}

\caption{Student: Multilevel Core Variable Descriptives}
\centering
\resizebox{\linewidth}{!}{
\begin{tabular}[t]{llccccc}
\toprule
  & Core Need & Competence & Autonomy & Relatedness & Quality & Attitudes NL\\
\midrule
Core Need &  & 0.60*** & 0.66*** & 0.43*** & 0.78*** & 0.11\\
Competence & 0.24*** &  & 0.71*** & 0.65*** & 0.74*** & 0.09\\
Autonomy & 0.15*** & 0.34*** &  & 0.56*** & 0.67*** & -0.06\\
Relatedness & 0.45*** & 0.29*** & 0.42*** &  & 0.54*** & -0.11\\
Quality & 0.17*** & 0.39*** & 0.08*** & 0.05** &  & 0.10\\
Attitudes NL & 0.35*** & 0.43*** & 0.11*** & 0.10*** & 0.12*** & \\
\addlinespace
Grand Mean & 84.87 & 72.55 & 82.59 & 61.21 & 83.77 & 67.26\\
Between-person SD & 9.17 & 14.47 & 11.21 & 13.36 & 9.12 & 18.64\\
Within-person SD & 20.33 & 21.17 & 16.06 & 28.74 & 16.80 & 9.40\\
ICC(1) & 0.15 & 0.30 & 0.32 & 0.17 & 0.20 & 0.80\\
ICC(2) & 0.89 & 0.95 & 0.95 & 0.90 & 0.88 & 0.99\\
\bottomrule
\multicolumn{7}{l}{\rule{0pt}{1em}\textit{Note: }}\\
\multicolumn{7}{l}{\rule{0pt}{1em}Upper triangle: Between-person correlations;}\\
\multicolumn{7}{l}{\rule{0pt}{1em}Lower triangle: Within-person coorelations;}\\
\multicolumn{7}{l}{\rule{0pt}{1em}*** p < .001, ** p < .01,  * p < .05}\\
\end{tabular}}
\end{minipage}
\end{table}

\begin{table}
\begin{minipage}[t][\textheight][t]{\textwidth}

\caption{Student: Multilevel Core Variable Descriptives (Outgroup Contact Only)}
\centering
\begin{tabular}[t]{llcccc}
\toprule
  & Core Need & Competence & Autonomy & Relatedness & Attitudes NL\\
\midrule
Core Need &  & 0.46*** & 0.50*** & -0.19 & 0.15\\
Competence & 0.23*** &  & 0.59*** & 0.39** & 0.22\\
Autonomy & 0.12*** & 0.32*** &  & 0.23 & 0.07\\
Relatedness & 0.48*** & 0.25*** & 0.19*** &  & -0.09\\
Attitudes NL & 0.18*** & 0.16*** & 0.15*** & 0.22*** & \\
\addlinespace
Grand Mean & 86.86 & 73.23 & 78.58 & 60.30 & 70.41\\
Between-person SD & 11.20 & 13.95 & 14.07 & 17.35 & 17.13\\
Within-person SD & 15.87 & 16.81 & 14.24 & 26.14 & 9.87\\
ICC(1) & 0.14 & 0.27 & 0.40 & 0.19 & 0.72\\
ICC(2) & 0.58 & 0.76 & 0.85 & 0.67 & 0.96\\
\bottomrule
\multicolumn{6}{l}{\rule{0pt}{1em}\textit{Note: }}\\
\multicolumn{6}{l}{\rule{0pt}{1em}Upper triangle: Between-person correlations;}\\
\multicolumn{6}{l}{\rule{0pt}{1em}Lower triangle: Within-person coorelations;}\\
\multicolumn{6}{l}{\rule{0pt}{1em}*** p < .001, ** p < .01,  * p < .05}\\
\end{tabular}
\end{minipage}
\end{table}


\subsection{Results}

\subsubsection{Contact Hypothesis}

\subsubsection{Core Need}

\subsubsection{Robustness}

\paragraph{interaction type}

\paragraph{other needs}

\section{Study 3}

\subsection{Methods}

\subsection{Results}
