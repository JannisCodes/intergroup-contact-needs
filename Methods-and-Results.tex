\section{Study 1}

Based on our main hypotheses, the aim of our first study was to
specifically test the general contact hypothesis, the influence of core
need fulfillment, and perceived interaction quality during intergroup
contacts. To this aim, we recruited recent migrants to the Netherlands
for an intensive longitudinal survey. Data were collected from May
5\textsuperscript{th} through June 6\textsuperscript{th}, 2018 (and all
participants started the study within the first two days). Correlations
and descriptive statistics of the included variables are available in
\tblref{tab:descrFullWide} and \tblref{tab:descrOutWide} (full data
description is available in Online Supplementary Material A).

\subsection{Methods}

\subsubsection{Participants}

After receiving ethical approval from the University Masked for Peer
Review, we recruited 23 non-Dutch migrants using the local paid
participant pool. Participants reported on their interactions for at
least 30 days with two daily measures (capturing the morning and
afternoon). With this design, we aimed at getting 50-60 measurements per
participant (\textit{M} = 53.26, \textit{SD} = 16.72, \textit{total N} =
1,225). This is a common number of measurements found in experience
sampling studies and offers sufficient power to model processes within
and between participants \citep[e.g.,][]{AanhetRot2012}. Participants
were compensated for their participation with up to 34 Euros -- each two
Euros for pre- and post-questionnaire and 50 Eurocents for every
experience sampling measurement. The sample consisted of relatively
young, educated, and western migrants from the global north (\(M_{age}\)
= 24.35, \(SD_{age}\) = 4.73, 19 women, 15 students). The sample
accurately describes the largest groups of migrants in the region
\citep[][]{GemeenteGroningen2015}.

\subsubsection{Procedure}

The study itself consisted of three main parts, an introductory
pre-measurement, the daily experience sampling measurements, and a
concluding post-measurement. After giving informed consent, participants
filled in an online pre-questionnaire assessing demographics and general
information about their immigration. Over the next thirty days,
participants were invited twice a day (at 12pm and 7pm) to reflect upon
their interactions, psychological need fulfillments, and current
attitudes towards the Dutch outgroup. General compliance was high
(85.90\% of all invited surveys were filled
in)\footnote{Two participants completed only two days (among the others, participation was 93.70\%).}.
The response rates were approximately equal during mornings (\textit{n}
= 621) and afternoons (\textit{n} = 604) and most measurements were
completed within four hours of the invitation. After the final day of
experience sampling measurements, participants were invited to fill in a
longer post measurement survey that mirrored the pre-measurement. All
key variables for this study were part of the short experience sampling
surveys.

\subsubsection{Materials}

\paragraph{Intergroup Contact}

To test the prerequisite effect of intergroup contact, every experience
sampling measurement started with the question
``\textit{Did you meet a Dutch person this morning [/afternoon]? (In person interaction for at least 10 minutes)}''.
Our participants recorded between 2--51 interactions with Dutch outgroup
members (3.23--91.07\% of individual experience sampling measurements;
387 of all 1,225 experience sampling
responses)\footnote{Two participants only recorded two experience sampling measurements each and none of these included outgroup contacts. These participants are removed from any analyses that focus on outgroup contacts.}.

\paragraph{Psychological Needs}

Irrespective of whether participants had an interaction with Dutch
people or not, everyone answered a short series of questions on
psychological need fulfillment. However, whereas participants with
interactions reported on the need fulfillment during the interaction,
people without interactions with Dutch people judged the past daytime
period in general. To assess the fulfillment of psychological needs, we
included two types of need measurement: (1) the core situational need
and (2) general self-determination theory needs.

For the core situational need, we asked participants in an open-ended
text field:
``\textit{What was your most important goal [during the interaction / this morning / this afternoon]?}''.
Then, with reference to the text entry, we asked how much this core need
was fulfilled during the interaction or the past daytime period:
``\textit{[The interaction / You] fulfilled your goal: [-previous text entry-]}''
on a continuous slider scale ranging from strongly disagree (-50) to
strongly agree (+50). The self-determination theory need measurements
are described in \appref{app:AppendixRobustness}.

\paragraph{Perceived Interaction Quality}

As an explanatory mechanism, we assessed ratings of the perceived
interaction quality. As our main measurement, participants rated the
statement ``\textit{Overall the interaction was …}'' on two continuous
slider scales measuring pleasantness
\citep[from unpleasant (-50) to pleasant (+50)) and meaningfulness (from superficial (-50) to meaningful (+50); both items adapted from][]{Downie2008}.

\paragraph{Outgroup Attitudes}

At the end of every experience sampling measurement, we asked all
participants about their current attitudes towards the Dutch -- our main
dependent variable. To assess the momentary outgroup evaluation we used
the common feeling thermometer: ``How favorable do you feel towards the
Dutch?'' \citep[][]{Lavrakas2008}. Participants then rated their
attitude on a continuous slider scale from ``very cold -- 0'' through
``no feeling -- 50'' to ``very warm -- 100''. Both the question phrasing
as well as the tick labels were consistent with large-scale panel
surveys \citep[e.g.,][]{DeBell2010}.

\subsection{Results}

\subsubsection{Contact Hypothesis}

Using a multilevel regression, we find that having an outgroup contact
is indeed associated with significantly more positive outgroup attitudes
(\textit{b} = 2.48, \textit{t}(1,200) = 4.37, \textit{p} \textless{}
.001, \textit{95\%CI}{[}1.37, 3.59{]}), even after controlling for
having an interaction with a non-Dutch (which did not relate to outgroup
attitudes independently). Additionally, while multi-level regressions
are generally robust against unequal cell sizes, we correct for
inequalities by using centered predictors and reintroducing the means as
level two predictors (\citealp{Yaremych2021}; for full results see
\tblref{tab:intergroupGeneralTblLong}, \fgrref{fig:ContactHypothesis},
and Online Supplementary Material
A)\footnote{Interestingly, adding random slopes to this model did not explain additional variance. This is unusual and might indicate that the effect is very consistent across participants. However, the small number of participants, or other measurement issues provide an alternative explanation, which is why we offer a combined data set analyses following the individual studies.}.
Thus, in our first data, we find initial evidence that outgroup contacts
show a positive effect on outgroup attitudes within real-life data.

\subsubsection{Core Need}

The main proposal of our article is that the success of an outgroup
contact might be explained by whether or not the contact fulfilled the
person's core situational need. This should, in turn, be due to a higher
perceived contact quality. We sequentially test whether the fulfillment
of the core need during an interaction is (1) related to more positive
outgroup attitudes, (2) higher perceived contact quality, and (3)
whether the variance explained by the core need is subsumed by the
perceived contact quality if considered jointly. We find that in the
multilevel models, the fulfillment of core situational needs during
outgroup contacts was associated with more positive outgroup attitudes
(random slopes model; \textit{b} = 0.17, \textit{t}(365) = 2.93,
\textit{p} = 0.004, \textit{95\%CI}{[}0.06, 0.29{]}) and also predicted
higher perceived contact quality (random intercept model; \textit{b} =
0.42, \textit{t}(365) = 8.07, \textit{p} \textless{} .001,
\textit{95\%CI}{[}0.32, 0.52{]}). Moreover, when we consider the
influences of core need fulfillment and contact quality on outgroup
attitudes jointly, we find that virtually all variance is explained by
perceived contact quality (random slopes model; \textit{b} = 0.23,
\textit{t}(364) = 4.22, \textit{p} \textless{} .001,
\textit{95\%CI}{[}0.12, 0.33{]}) and only little unique variance is
still explained by core need fulfillment (\textit{b} = 0.04,
\textit{t}(364) = 0.64, \textit{p} = 0.523, \textit{95\%CI}{[}-0.08,
0.15{]}, also see \fgrref[-A]{fig:MainPaths}). We thus find support for
our hypotheses and can conclude that in this data set the fulfillment of
core situational needs had a significant influence on outgroup
attitudes. Additionally, this effect is likely explained by its effect
through perceived contact quality.

\section{Study 2}

The aim of Study 2 is similar to Study 1, as we again test the general
contact hypothesis, the influence of core need fulfillment, and
perceived contact quality during intergroup contacts. However, in this
second study we collected a substantially larger sample of international
students who recently arrived in the Netherlands and also improved the
study design (e.g., pop-up explanations described later). The survey
method again offers a large body of ecologically valid data on need
satisfaction in real-life intergroup contact situations as these
students will likely interact with the Dutch majority outgroup on a
daily basis. Data were collected from November 19\textsuperscript{th},
2018 through January 6\textsuperscript{th}, 2019. Correlations and
descriptive statistics of the included variables are available in
\tblref{tab:descrFullWide} and \tblref{tab:descrOutWide}.

\subsection{Methods}

\subsubsection{Participants}

We recruited 113 international students using a local participant pool.
We specifically targeted non-Dutch students, who had recently arrived in
the Netherlands. Participants reported on their interactions for at
least 30 days with two daily measures (capturing the morning and
afternoon). With this design, we again aimed at receiving 50-60
measurements per participant (\textit{M} = 43.94, \textit{SD} = 15.00,
\textit{total N} = 4,965). As with the previous study, this should offer
sufficient power to model processes within participants and will lend
stronger weight to between-participant results. Participants were
compensated for their participation with partial course credits ---
depending on their participation. The sample consisted of relatively
young migrants, who were mostly from the global north (\(M_{age}\) =
20.24, \(SD_{age}\) = 2.12, 84 women). The sample fairly accurately
describes the local population of international students.

\subsubsection{Procedure}

The study procedure mirrored the setup of Study 1 and consisted of pre-,
experience sampling-, and post-measurements. The participants were
invited for experience sampling measurements twice a day (at 12pm and
7pm) for 30 days. General compliance was high (70.87\% of all invited
surveys were filled in). The response rates were approximately equal
during mornings (\textit{n} = 2,608) and afternoons (\textit{n} =
2,357). All key variables for this study were part of the short
experience sampling surveys.

\subsubsection{Materials}

\paragraph{Intergroup Contact}

To measure intergroup contacts, every experience sampling measurement
started with the question
``\textit{Did you meet a Dutch person this morning [/afternoon]? (in-person interaction for at least 10 minutes)}''.
Participants were additionally offered a pop-up explanation: ``With
in-person interaction, we mean a continued interaction with another
person (potentially in a group) that lasted at least 10 minutes. This
interaction should be offline and face-to-face. It should include some
form of verbal communication and should be uninterrupted to still count
as the same interaction. Any individual interaction can last minutes or
hours. If there were multiple interaction partners, we would like you to
focus on the person that was most important to you during the
interaction.''. The participants recorded between 1--43 interactions
with Dutch majority people (1.64--93.75\% of individual experience
sampling measurements; 935 of all 4,965 experience sampling responses).

\paragraph{Psychological Needs}

For the core situational need, we asked participants in an open-ended
text field:
``\textit{What was your main goal [during the interaction with -X- / this morning / this afternoon]?}''
(where \textit{-X-} was dynamically replaced with the name of the
interaction partner). Participants could additionally click on a pop-up
explanation: ``Your main goal during an interaction can vary depending
on the interaction. It could be to connect with friends, to find or
provide help, to achieve academic ambitions, work on your fitness, work
for a job, or simply to get a coffee, just as well as many other
concrete or abstract goals that are important to you in the moment. It
really depends on your subjective experience of the interaction.''.
Then, with reference to the text entry, we asked how much this core need
was fulfilled during the interaction or the past daytime period:
``\textit{During your interaction with -X- [this morning / this evening] your goal (-previous text entry-) was fulfilled.}''
on a continuous slider scale ranging from strongly disagree (1) to
strongly agree (100).

\paragraph{Perceived Interaction Quality}

The ratings of the perceived contact quality were identical to Study 1.

\paragraph{Outgroup Attitudes}

As in Study 1, attitudes towards the Dutch majority outgroup were again
measured using the feeling thermometer.

\subsection{Results}

\subsubsection{Contact Hypothesis}

We tested the most general contact hypothesis as we did for Study 1. We
find that having an outgroup interaction is indeed associated with
significantly more positive outgroup attitudes within the participants
(random slopes model; \textit{b} = 2.83, \textit{t}(4,850) = 3.57,
\textit{p} \textless{} .001, \textit{95\%CI}{[}1.28, 4.38{]}), even
after controlling for having an interaction with a non-Dutch person
(which did not relate to outgroup attitudes independently). We again
added the participant means back into the model. We find that in this
data set participant-level outgroup contact proportions were also a
positive predictor of outgroup attitudes (\textit{b} = 26.55,
\textit{t}(110) = 2.90, \textit{p} = 0.004, \textit{95\%CI}{[}8.61,
44.46{]}). The relative number of non-outgroup interactions showed no
such effect (for full results see \tblref{tab:intergroupGeneralTblLong},
\fgrref{fig:ContactHypothesis}, as well as Online Supplementary Material
A). Thus, in our second data set we also find that outgroup contacts
show a positive effect on outgroup attitudes in the moment. We
additionally find an average between-participant effect of the relative
number of interactions participants had.

\subsubsection{Core Need}

We again sequentially tested the core need model as we did for Study 1.
We find that in the multilevel models, the fulfillment of core
situational needs during outgroup contacts was associated with more
positive outgroup attitudes (random slopes model; \textit{b} = 0.13,
\textit{t}(826) = 4.18, \textit{p} \textless{} .001,
\textit{95\%CI}{[}0.07, 0.19{]}) and also predicted higher perceived
interaction quality (random slopes model; \textit{b} = 0.40,
\textit{t}(826) = 6.46, \textit{p} \textless{} .001,
\textit{95\%CI}{[}0.28, 0.52{]}). Additionally, if we consider the
influences of core need fulfillment and interaction quality on outgroup
attitudes jointly, we find that virtually all variance is explained by
perceived interaction quality (random slopes model; \textit{b} = 0.17,
\textit{t}(825) = 5.56, \textit{p} \textless{} .001,
\textit{95\%CI}{[}0.11, 0.23{]}) and only little unique variance is
still explained by core need fulfillment (\textit{b} = 0.03,
\textit{t}(825) = 1.47, \textit{p} = 0.141, \textit{95\%CI}{[}-0.01,
0.07{]}; also see \fgrref[-B]{fig:MainPaths} and
\tblref{tab:intergroupNeedsTblLong} for full results). These results are
consistent with the results in Study 1. We, thus, find support for our
hypotheses that the fulfillment of core situational needs had a
significant influence on outgroup attitudes and that this effect is
likely explained by its effect through perceived interaction quality.

\section{Study 3}

The aim of this final study is to extend the previous studies by
additionally testing Allport's conditions in an extensive longitudinal
design and to compare the predictive powers of Allport's conditions and
the core situational need fulfillment. For this study, we specifically
recruited international medical students, because they represent a
particular group of migrants who face structural requirements to
integrate and interact with Dutch majority outgroup members on a daily
basis. As part of their educational program, the migrants are required
to take language courses and interact with patients as part of their
medical internships and medical residency. The extensive longitudinal
survey method again offers a large body of ecologically valid data on
need satisfaction in real-life intergroup contact situations. Data were
collected from November 8\textsuperscript{th}, 2019 to January
10\textsuperscript{th}, 2020. The full preregistration is available at
\citet[][]{KreienkampMasked2021f}. Correlations and descriptive
statistics of the included variables are available in
\tblref{tab:descrFullWide} and \tblref{tab:descrOutWide}.

\subsection{Methods}

\subsubsection{Participants}

We recruited 71 international medical students using contacts within the
University Medical School. We specifically targeted non-Dutch students,
who had recently arrived in the Netherlands. Participants reported on
their interactions for at least 30 days with two daily measures
(capturing the morning and afternoon). With this design, we aimed at
getting 50-60 measurements per participant (\textit{M} = 57.85,
\textit{SD} = 20.68, \textit{total N} = 4,107). As with the previous
studies, this offered sufficient power to model processes within
participants. Participants were compensated in the same manner as during
Study 1. The sample consisted of relatively young migrants (\(M_{age}\)
= 22.68, \(SD_{age}\) = 3.10, 59 women). The sample fairly accurately
describes the local population of young international medical
professionals.

\subsubsection{Procedure}

The study procedure mirrored the setup of studies one and two, in that
it consisted of pre-, experience sampling-, and post-measurement. The
participants were invited for experience sampling measurements twice a
day (at 12pm and 7pm) for 30 days. General compliance was high (85.92\%
filled in at least 31 experience sampling surveys or more). The response
rates were approximately equal during mornings (\textit{n} = 2,092) and
afternoons (\textit{n} = 2,015). All key variables for this study were
part of the short experience sampling surveys.

\subsubsection{Materials}

\paragraph{Intergroup Contact}

The measurement of intergroup contacts was identical to Study 2. The
participants recorded between 1--71 interactions with Dutch outgroup
members (2.13--87.18\% of individual experience sampling measurements;
1,702 of all 4,107 experience sampling responses).

\paragraph{Psychological Needs}

The measurement of the core situational need and its fulfillment was
identical to Study 2.

\paragraph{Allport's Conditions}

We measured how much each of the interactions fulfilled Allport's
conditions of optimal contact using a common short scale comprised of
four attributes \citep{Islam1993, Voci2003, AlRamiah2012a}. In
particular, we asked participants to rate how much the interaction had
equal status
(``\textit{The interaction with [name interaction partner] was on equal footing (same status)}''),
a common goal
(``\textit{[name interaction partner] shared your goal ([free-text entry interaction key need])}''),
support of authorities
(``\textit{The interaction with [name interaction partner] was voluntary}''),
and intergroup cooperation
(``\textit{The interaction with [name interaction partner] was cooperative}'').
We create a mean-averaged index of Allport's conditions in response to
past findings indicating that the conditions are best conceptualized
jointly and as functioning together rather than as fully independent
factors \citep[][, p. 766]{Pettigrew2006}. For full psychometric
information see Online Supplementary Material A.

\paragraph{Perceived Interaction Quality}

The ratings of the perceived interaction quality were identical to Study
1.

\paragraph{Outgroup Attitudes}

Attitudes towards the Dutch majority outgroup were again measured using
the feeling thermometer, as in studies one and two.

\subsection{Results}

\subsubsection{Contact Hypothesis}

In a multilevel regression, we find that having an outgroup interaction
was again associated with significantly more positive outgroup attitudes
within the participants (random slopes model; \textit{b} = 5.57,
\textit{t}(3,834) = 6.52, \textit{p} \textless{} .001,
\textit{95\%CI}{[}3.90, 7.23{]}), even after controlling for having an
interaction with a non-Dutch (which did not relate to outgroup attitudes
independently; for full results see
\tblref{tab:intergroupGeneralTblLong} and
\fgrref{fig:ContactHypothesis}). Thus, in our third data set, we find
that the within-person contemporaneous effect of intergroup contact was
consistent across all three studies.

\subsubsection{Core Need}

We tested the situational core needs model analogous to the previous
studies. We find that the fulfillment of the core need during outgroup
contacts was associated with more positive outgroup attitudes (random
slopes model; \textit{b} = 0.19, \textit{t}(1,601) = 5.29, \textit{p}
\textless{} .001, \textit{95\%CI}{[}0.12, 0.27{]}) and also predicted
higher perceived interaction quality (random slopes model; \textit{b} =
0.43, \textit{t}(1,605) = 8.35, \textit{p} \textless{} .001,
\textit{95\%CI}{[}0.33, 0.54{]}). Additionally, once we consider the
influences of core need fulfillment and interaction quality on outgroup
attitudes jointly, we find that perceived interaction quality is a
substantially stronger predictor (random slopes model; \textit{b} =
0.17, \textit{t}(1,600) = 6.86, \textit{p} \textless{} .001,
\textit{95\%CI}{[}0.12, 0.22{]}) and the unique variance explained by
core need fulfillment was roughly half of its original effect size
(\textit{b} = 0.11, \textit{t}(1,600) = 3.50, \textit{p} \textless{}
.001, \textit{95\%CI}{[}0.05, 0.17{]}; also see
\fgrref[-C]{fig:MainPaths} and \tblref{tab:intergroupNeedsTblLong} for
full results). As with the previous two studies, these results indicate
that in this data set outgroup attitudes were significantly predicted by
the fulfillment of core situational needs and the results suggest that
this might be due to higher perceived interaction quality.

\subsubsection{Allport's Conditions}

We tested the impact of Allport's conditions in the same manner as we
tested our core needs model. In the multilevel models, we find that the
fulfillment of Allport's Conditions during outgroup contacts was
associated with more positive outgroup attitudes (random slopes model;
\textit{b} = 0.22, \textit{t}(1,601) = 5.86, \textit{p} \textless{}
.001, \textit{95\%CI}{[}0.15, 0.29{]}) and also predicted higher
perceived interaction quality (random slopes model; \textit{b} = 0.65,
\textit{t}(1,605) = 11.82, \textit{p} \textless{} .001,
\textit{95\%CI}{[}0.54, 0.76{]}). Moreover, when we considered the
influences of Allport's Conditions and interaction quality on outgroup
attitudes jointly, we found that perceived interaction quality was a
substantially stronger predictor (random slopes model; \textit{b} =
0.17, \textit{t}(1,600) = 6.25, \textit{p} \textless{} .001,
\textit{95\%CI}{[}0.12, 0.23{]}) and the unique variance explained by
Allport's Conditions was less than half of its original effect size
(\textit{b} = 0.09, \textit{t}(1,600) = 2.89, \textit{p} = 0.004,
\textit{95\%CI}{[}0.03, 0.15{]}; also see
\tblref{tab:intergroupNeedsTblLong}). These results indicate that in
this data set the fulfillment of Allport's conditions had a significant
influence on outgroup attitudes and this effect is likely, in parts,
explained by its effect through perceived interaction quality.

\subsubsection{Compare Fulfillment of Core Need and Allport's Conditions}

To test whether Allport's conditions or the core need fulfillment were
better at predicting outgroup attitudes, we first assessed relative
model performance indices (i.e., Akaike information criterion, and
Bayesian information criterion), and then consider the two predictors in
a joint model to see whether the two approaches predicted the same
variance in outgroup attitudes. When comparing the model selection
indices, we found that the fulfillment of the situational core need
indeed performed slightly better than the model using Allport's
conditions (\(AIC_{CoreNeed}\) 12632.02 \textless{} 12651.59
\(AIC_{Allport}\), and \(BIC_{CoreNeed}\) 12664.55 \textless{} 12684.12
\(BIC_{Allport}\)). Additionally, when considering the predictors
jointly, we find that both significantly predict outgroup attitudes with
similar-sized regression parameters (random slopes model; Allport's
Conditions: \textit{b} = 0.16, \textit{t}(1,600) = 4.92, \textit{p}
\textless{} .001, \textit{95\%CI}{[}0.09, 0.24{]}, Core Need: \textit{b}
= 0.14, \textit{t}(1,600) = 3.85, \textit{p} \textless{} .001,
\textit{95\%CI}{[}0.08, 0.17{]}; also see
\tblref{tab:intergroupNeedsTblLong}). This indicates that, although both
Allport's conditions and the core need fulfillment seem to (in part)
work through perceived interaction quality, they explain different
aspects of the variance in outgroup attitudes and do not constitute one
another.

\section{Robustness, Stability, and Embeddedness across Studies}

Beyond the individual results of the three studies, we conducted three
additional analyses to test the broader cross-study claims and
contextualize our results. In particular, we aimed to gain a broader
picture of the individual results by considering them jointly. We also
leveraged the data from all participants to test between-participant
contact effects and explore the content of the participants' situational
contact needs.

\subsection{Stability}

We first assessed the stability of our main analyses across the three
studies. Plotting the effect sizes of each parameter of interest in a
forest plot, as well as the average meta-analytic effect, shows that for
the basic contact hypothesis test outgroup contact had a strong and
consistent effect on outgroup attitudes. Interactions with non-outgroup
members consistently had no meaningful effect on outgroup attitudes
(\fgrref{fig:ContactHypothesis}). While this would be expected from the
general intergroup contact literature, this is not a trivial finding.
Being among the first to assess the contact hypothesis using real-life
intensive longitudinal data, we extend cross-sectional findings to
individual-level assessments. When looking at the fulfillment of core
needs during intergroup contacts, we find that the motivational
mechanism is consistently a meaningful predictor of outgroup attitudes
and interaction quality perceptions (see
\fgrref[ A and B]{fig:AllportNeedFulfillment}). We also see that the
effect of core need fulfillment on outgroup attitudes is strongly
reduced when modeled together with interaction quality perceptions,
supporting our assertion that interaction quality offers a psychological
mechanism for the direct effect (see
\fgrref[-C]{fig:AllportNeedFulfillment}).

\subsection{General contact hypothesis}

During the main analyses, we have thus far shown that participants held
more positive outgroup attitudes following intergroup contacts and that
perceived interaction quality was associated with more positive outgroup
attitudes following an intergroup contact. We have, however, not brought
the two elements of the contact hypothesis together in a single
analysis. To jointly test the effects of contact frequency and average
interaction quality, we ran a linear regression model where average
outgroup attitudes were predicted by the number of interactions and the
average interaction quality ratings of all participants across the three
studies. We did so while controlling for the possible effects of
study-specific differences. To include the study-membership as a control
variable, we used the student sample (Study 2) as the reference group
because it was both the largest and the most homogeneous study. Looking
at the overall model, we found that the model predicted 13.89\% of the
variance in average outgroup attitudes (\textit{F}(9, 189) = 3.39,
\textit{p} = 0.001, \textit{R\textsuperscript{2}} = 0.14). Looking at
the individual effects, we found that only the number of outgroup
interactions has a clear association with average outgroup attitudes
(\textit{b} = 0.55, \textit{t}(189) = 2.82, \textit{p} = 0.005,
\textit{95\%CI}{[}0.16, 0.93{]}). The average interaction quality
perceptions had a much smaller effect (\textit{b} = 0.29,
\textit{t}(189) = 2.09, \textit{p} = 0.038, \textit{95\%CI}{[}0.02,
0.56{]}), and importantly we found no interaction effect at all. In
short, the effect of interaction frequency did not depend on the average
interaction quality.

\subsection{Contact key need content}

Finally, we used the qualitative data from the participants'
self-identified core needs to contextualize the results of our main
analysis. However, because our participants jointly reported on
thousands of intergroup contacts, it would not have been feasible to
analyze these qualitative responses in a traditional qualitative content
analysis. We instead relied on recent machine learning advances within
the natural language processing domain. For our analysis, we used the
BERT language model. BERT (Bidirectional Encoder Representations from
Transformers) was developed by Google in 2018 and today forms a key
element of many natural language processing workflows. In its essence,
BERT is a framework that allows users to codify every word in relation
to every other word within a large set of documents. We extracted 47
topics from the 2,983 interaction goal free-text entries --- a
relatively large number of topics. The higher number of topics allowed
us to retain more of the smaller topics and leaves a relatively low
number of 308 free-text entries unclassified (10.33\%). A full write-up
of the topic modeling process is available in Online Supplemental
Material B.

In terms of the content of the topics, we find that a number of topics
are primarily task-oriented, where participants hope to increase their
study, research, presentation, or work performance. Opposing the task-
and work-oriented needs, are a wide variety of leisure-related needs
wishes, like relaxation and entertainment. Additionally, some clusters
were primarily relationship-oriented, so that participants sought
contact with outgroup members for intimate and casual social contact in
itself. Similarly, socializing and celebrations were also explicit
social needs (incl., parties). This also included a subtopic of
spiritual, religious, and otherwise transcendental needs
(incl.~meditation, prayer, religious services). Among the
leisure-oriented topics was also a set of contact goals that were
specifically migration-specific (e.g., wish to learn about culture,
politics, and language) or were concerned with informational needs more
generally (e.g., seeking answers, bureaucratic information). A similar
set of topics was specifically geared towards a wish to experience
cultural products (e.g., music, theater, food) or had travel-related
goals in their interactions with the majority group members.

One interesting observation was the importance of contact goals specific
to contact through the medical and public health system. This goal type
was partly specific to the medical professionals (e.g., working with
patients, treatment), but also more broadly highlighted interactions of
newcomers with the outgroup majority as patients themselves (e.g.,
therapeutic goals). Also health, fitness, and personal improvement goals
(e.g., sports and music) were common goals that participants shared
during the interactions with the majority group members. Another topic
less commonly found within the literature were more practically oriented
needs, where participants sought to share food and cook together or had
organizational needs (e.g., housing, cleaning, and co-living). In sum,
almost all extracted topics fall into broader or narrower need concepts
that are commonly discussed within the psychological needs literature
\citep[e.g.,][]{orehek2018a} and offer insight into a core aspect of the
migration experience that has remained broadly under-explored
\citep[][]{Kreienkamp2022d}\footnote{It should be noted that these topic models are not without limitations, especially because they depend on a small set of hyperparameters that determine the characteristics of the embedding, dimension reduction, clustering, and core term extraction. The authors have also inspected a major subset of the free-text responses manually and are under the impression that the topics described here accurately represent the broader content.}.
