\faQuestionCircle~Unsure about order of study one and two. Currently
chronological. But could also be: (1) main model works in larger student
sample, (2) works in economic migrant sample, (3) test full thing in
medical sample.

\section{Study 1}

Based on our main hypotheses, the aim of our first study is to
specifically test the general contact hypothesis, the influence of core
need fulfillment, and perceived interaction quality during intergroup
contacts. To this aim, we conducted an intensive longitudinal survey
study with recent economic migrants to the Netherlands, gathering a
large body of ecologically valid data on need satisfaction in real-life
intergroup contact situations.

\subsection{Methods}

\subsubsection{Participants}

After receiving ethical approval from the University of Groningen, we
recruited 23 migrants using the local paid participant pool and
specifically targeted non-Dutch migrants to participate in our study for
a compensation of up to 34 Euros -- each two Euros for pre- and
post-questionnaire as well as 50 Eurocents for every experience sampling
measure. Participants reported on their interactions for at least 30
days with two daily measures (capturing the morning and afternoon). With
this design, we aimed at getting 50-60 measurements per participant
(\textit{M} = 53.26, \textit{M} = 16.72). This is a common number of
measurements found in experience sampling studies
\citep[e.g., for a systematic review see][]{AanhetRot2012} and should
offer sufficient power to model processes within and between
participants. The sample consisted of relatively young, educated, and
western migrants from the global north (\(M_{age}\) = 24.35,
\(SD_{age}\) = 4.73, 19 women, 15 students). The sample accurately
describes one of the largest groups of migrants in the region
\citep[][]{GemeenteGroningen2015}.

\subsubsection{Procedure}

\subsubsection{Materials}

\subsection{Results}

\section{Study 2}

\subsection{Methods}

\subsection{Results}

\section{Study 2}

\subsection{Methods}

\subsection{Results}
