\section{Study 1}

Based on our main hypotheses, the aim of our first study is to
specifically test the general contact hypothesis, the influence of core
need fulfillment, and perceived interaction quality during intergroup
contacts. To this aim, we conducted an intensive longitudinal survey
study with recent migrants to the Netherlands. Data was collected from
May 5\textsuperscript{th} through June 6\textsuperscript{th}, 2018 (and
all participants started the study within the first two days).

The full surveys are available in our Online Supplementary Material A
and the full data description is available in Online Supplementary
Materials B. Correlations and descriptive statistics of the included
variables are available in Table \ref{tab:descrFullLong} and Table
\ref{tab:descrOutLong}.

\subsection{Methods}

\subsubsection{Participants}

After receiving ethical approval from the University of Groningen, we
recruited 23 migrants using the local paid participant pool and
specifically targeted non-Dutch migrants to participate in our study.
Participants reported on their interactions for at least 30 days with
two daily measures (capturing the morning and afternoon). With this
design, we aimed at getting 50-60 measurements per participant
(\textit{M} = 53.26, \textit{SD} = 16.72, \textit{total N} = 1225). This
is a common number of measurements found in experience sampling studies
and offers sufficient power to model processes within and between
participants \citep[e.g., for a systematic review see][]{AanhetRot2012}.
Participants were compensated for their participation with up to 34
Euros -- each two Euros for pre- and post-questionnaire as well as 50
Eurocents for every experience sampling measurement occasion. The sample
consisted of relatively young, educated, and western migrants from the
global north (\(M_{age}\) = 24.35, \(SD_{age}\) = 4.73, 19 women, 15
students). The sample accurately describes one of the largest groups of
migrants in the region \citep[][]{GemeenteGroningen2015}.

\subsubsection{Procedure}

The study itself consisted of three main parts, an introductory
pre-measurement, and the daily experience sampling measurements, as well
as a concluding post-measurement. After giving informed consent,
participants started by filling in an online pre-questionnaire assessing
demographics and general information about their immigration. Over the
next thirty days, the participants were then invited twice a day (at 12
pm and 7pm) to reflect upon their interactions, psychological need
fulfillments, and current attitudes towards the Dutch outgroup
(\textit{median duration} = 2M 22S, \textit{MAD duration} = 1M 19S).
General compliance was high (85.90\% of all invited surveys were filled
in)\footnote{Two participants completed only two days (among the others, participation was 93.70\%)}.
The response rates were approximately equal during mornings (\textit{n}
= 621) and afternoons (\textit{n} = 604) and most measurements were
completed within four hours of the invitation. After the final day of
daily diary measurements, participants were invited to fill in a longer
post measurement survey that mirrored the pre-measurement. All key
variables for this study were part of the short daily diary surveys.

\subsubsection{Materials}

\paragraph{Intergroup Contact}

To test the prerequisite effect of intergroup contact, every experience
sampling measurement started with the question
``\textit{Did you meet a Dutch person this morning [/afternoon]? (In person interaction for at least 10 minutes)}''.
Our participants recorded between 2--51 interactions with Dutch outgroup
members (3.23--91.07\% of individual daily diary measurements; 31.59\%
of all 1225 daily diary
responses)\footnote{Two participants only recorded two daily diary measurements each and non of these included outgroup contacts. These participants are removed from any analyses including outgroup contacts.}.

\paragraph{Psychological Needs}

Irrespective of whether participants had an interaction with Dutch
people or not, everyone answered a short series of questions on
psychological need fulfillment. However, whereas participants with
interactions reported on the need fulfillment during the interaction,
people without interactions with Dutch people judged the past daytime
period in general. To assess the fulfillment of psychological needs, we
included two types of need measurement: (1) the core situational need
and (2) general self-determination theory needs.

For the core situational need, we asked participants in an open ended
text field:
``\textit{What was your most important goal [during the interaction / this morning / this afternoon]?}''.
Then, with reference to the text entry, we asked how much this core need
was fulfilled during the interaction or the past daytime period:
``\textit{[The interaction / You] fulfilled your goal: [-previous text entry-]}''
on a continuous slider scale ranging from strongly disagree (-50) to
strongly agree (+50).

We, additionally, included a common measure of three self-determination
theory needs \citep[see][]{Downie2008}. The items were introduced either
by ``\textit{During the interaction:}'' or
``\textit{This morning [/afternoon]:}'' and measured autonomy
(``\textit{I was myself.}''), competence
(``\textit{I felt competent.}''), and relatedness (without intergroup
contact ``\textit{I had a strong need to belong}''; with intergroup
contact: ``\textit{I shared information about myself.}'' and
``\textit{The other(s) shared information about themselves.}''). All
items were rated on a continuous slider scale from very little (-50) to
a great deal (+50).

\paragraph{Perceived Interaction Quality}

As an explanatory mechanism, we assessed ratings of the perceived
interaction quality. As our main measurement, participants rated the
statement ``\textit{Overall the interaction was …}'' on two continuous
slider scales measuring pleasantness
\citep[from unpleasant (-50) to pleasant (+50)) and meaningfulness (from superficial (-50) to meaningful (+50); both items adapted from][]{Downie2008}.

\paragraph{Outgroup Attitudes}

At the end of every daily diary measurement we asked all participants
about their current attitudes towards the Dutch -- our main dependent
variable. To assess the momentary outgroup evaluation we used the common
feeling thermometer: ``How favorable do you feel towards the Dutch?''
\citep[][]{Lavrakas2008}. Participants then rated their attitude on a
continuous slider scale from ``very cold -- 0'' through ``no feeling --
50'' to ``very warm -- 100''. Both the question phrasing as well as the
tick labels were consistent with large-scale panel surveys
\citep[e.g.,][]{DeBell2010}.

\subsection{Results}

\subsubsection{Contact Hypothesis}

We tested the most general contact hypothesis in two steps. First, we
assessed whether more intergroup interactions were related to to more
positive outgroup attitudes. Second, we tested whether a potential
positive effect on outgroup attitudes depended on the interaction
quality (jointly with the number of interactions). We find that neither
the number of interactions nor the number of daily diary responses with
an interaction were significantly related with the average outgroup
attitudes (all \textbar{} \textit{r} \textbar{} \textless{} 0.03, all
\textit{p} \textgreater{} 0.887). However, we did find a medium sized
correlation between the participants' Average Interaction Quality and
their Average Outgroup Attitudes (\textit{r} = \hl{X.XX}, \textit{p} =
\hl{X.XXX}. And when considering the number of interactions and average
interaction quality jointly in a linear regression, we additionally find
a statistically significant interaction term (\textit{b} = -0.05,
\textit{t}(17) = -2.95, \textit{p} = 0.009, \(\eta_p^2\) = 0.34), where
in our sample with an increasing number of interactions the positive
effect of average interaction quality becomes weaker. We additionally
used a multilevel regression to check whether having an interaction with
an outgroup member had a situational (i.e., contemporaneous) effect
within the participants. We find that having an outgroup interaction is
indeed associated with significantly more positive outgroup attitudes
within the participants (\textit{b} = 2.48, \textit{t}(1200) = 4.36,
\textit{p} \textless{} .001, \textit{95\%CI}{[}1.36, 3.59{]}), even
after controlling for having an interaction with a non-Dutch (which did
not relate to outgroup attitudes independently; For full results see
Online Supplementary Materials
B)\footnote{Interestingly, adding random slopes to this model did not explain additional variance. This is unusual and might indicate the the effect is very consistent across participants. However, the small number of participants, or other measurement issues provide an alternative explanation}.
Thus, in our first data we find mixed results, where outgroup contacts
show a positive effect on outgroup attitudes in the moment but these
results do not translate into average differences between participants
(eventough the average number of interactions does seem to interact with
average interaction quality to predict outgroup attitudes). However, it
should be noted that the between-participant analyses in the aggregated
form are based on a small sample size.

\subsubsection{Core Need}

The main proposal of our article is that the success of an outgroup
interaction might be explained by whether or not the interaction
fulfilled the person's core situational need. This should, in turn, be
due to a higher perceived interaction quality. We sequentially test
whether the fulfillment of the core need during an interaction is (1)
related to more positive outgroup attitudes, (2) higher perceived
interaction quality, and (3) whether the variance explained by the core
need is assumed by the perceived interaction quality if considered
jointly.

We find that in the highest multilevel models, the fulfillment of core
situational needs during outgroup contacts was associated with more
positive outgroup attitudes (random slopes model; \textit{b} = 0.17,
\textit{t}(365) = 2.93, \textit{p} = 0.004, \textit{95\%CI}{[}0.06,
0.29{]}) and also predicted higher perceived interaction quality (random
intercept model; \textit{b} = 0.42, \textit{t}(365) = 8.07, \textit{p}
\textless{} .001, \textit{95\%CI}{[}0.32, 0.52{]}). Moreover, when we
consider the influences of core need fulfillment and interaction quality
on outgroup attitudes jointly, we find that virtually all variance is
explained by perceived interaction quality (random slopes mode;
\textit{b} = 0.23, \textit{t}(364) = 4.22, \textit{p} \textless{} .001,
\textit{95\%CI}{[}0.12, 0.33{]}) and only little unique variance is
still explained by core need fulfillment (\textit{b} = 0.04,
\textit{t}(364) = 0.64, \textit{p} = 0.523, \textit{95\%CI}{[}-0.08,
0.15{]}). We thus find support for our hypotheses and can conclude that
in this data set the fulfillment of core situational needs had a
significant influence on outgroup attitudes. Additionally, this effect
is likely being explained by its effect through perceived interaction
quality.

\faQuestionCircle~Are we sure that we don't want some sort of
bootstrapped indirect effect?

\subsubsection{Robustness}

To build further confidence in our results, we assessed two additional
models that might offer alternative explanations. First, to ensure that
the effect of core need fulfillment is specific to an actual
interaction, we compared the effect to core need fulfillement in
situations without an interaction. For this, we analyzed the generalized
situational core need fulfillment (either during an interaction or about
the daytime in general) and tested whether the effect differed during
daily diary measurements with and without outgroup contacts. We found no
main effect of core need fulfillment (random slopes model; \textit{b} =
-0.10, \textit{t}(1199) = -2.22, \textit{p} = 0.026,
\textit{95\%CI}{[}-0.18, -0.01{]}) but a significant interaction effect
of core need fulfillment and outgroup contact (\textit{b} = 0.13,
\textit{t}(1199) = 4.31, \textit{p} \textless{} .001,
\textit{95\%CI}{[}0.07, 0.18{]}). Together with a significant main
effect of having an outgroup contact, this indicates that it is not key
need fulfillment in general --- but only key need fulfillment during an
outgroup contact that predicts more positive outgroup attitudes.

In a final step, we controlled for other fundamental psychological needs
during the interaction. We focus on the three commonly considered self
determination needs: competence, autonomy, and relatedness. We find that
the core need adds significantly above a model with only the self
determination theory needs (random slopes models; \(\chi^2\)(6,
\textit{N} = 21) = 19.60, \textit{p} = 0.003). We find that next to
relatedness (\textit{b} = 0.10, \textit{t}(361) = 4.04, \textit{p}
\textless{} .001, \textit{95\%CI}{[}0.05, 0.14{]}), the core need
explains the most variance in outgroup attitudes after an outgroup
contact (\textit{b} = 0.09, \textit{t}(361) = 2.23, \textit{p} = 0.026,
\textit{95\%CI}{[}0.01, 0.18{]}). When compared to the model with only
the SDT needs, the core need fulfillment flexibly takes on some of the
explained variance of all three fundamental needs (competence and
autonomy needs turning non-significant; all \textit{b} \textless{} 0.05,
all \textit{p} \textgreater{} 0.544). For full results see Online
Supplementary Information B.

\section{Study 2}

The aim of Study 2 is similar to Study 1, as we again test the general
contact hypothesis, the influence of core need fulfillment, and
perceived interaction quality during intergroup contacts. However, in
this second study we collected a substantially larger sample of
international students who recently arrived in the Netherlands and also
improved the study design (e.g., pop up explanations described later).
The survey method again offers a large body of ecologically valid data
on need satisfaction in real-life intergroup contact situations as these
students will likely interact with the Dutch majority outgroup on a
daily basis. Data was collected from November 19\textsuperscript{th},
2018 through January 6\textsuperscript{th}, 2019.

The full surveys are available in our Online Supplementary Material A
and the full data description is available in Online Supplementary
Materials B. Correlations and descriptive statistics of the included
variables are available in Table \ref{tab:descrFullLong} and Table
\ref{tab:descrOutLong}.

\subsection{Methods}

\subsubsection{Participants}

After receiving ethical approval from the University of Groningen, we
recruited 113 international students using a local participant pool. We
specifically targeted non-Dutch students, who had recently arrived in
the Netherlands. Participants reported on their interactions for at
least 30 days with two daily measures (capturing the morning and
afternoon). With this design, we aimed at getting 50-60 measurements per
participant (\textit{M} = 43.94, \textit{SD} = 15.00, \textit{total N} =
4965). As with the previous study this should offer sufficient power to
model processes within participants and will lend stronger weight to
between-participant results. Participants were compensated for their
participation with partial course credits --- depending on their
participation. The sample consisted of relatively young migrants, who
were mostly from the global north (\(M_{age}\) = 20.24, \(SD_{age}\) =
2.12, 84 women). The sample fairly accurately describes the local
population of international students.

\subsubsection{Procedure}

The study procedure mirrored the setup of study one and consisted of
pre-, daily diary-, and post-measurement. The participants were invited
for daily diary measurements twice a day (at 12 pm and 7pm) for 30 days
(\textit{median duration} = 3M 46S, \textit{MAD duration} = 1M 54S).
General compliance was high (70.87\% of all invited surveys were filled
in). The response rates were approximately equal during mornings
(\textit{n} = 2608) and afternoons (\textit{n} = 2357). All key
variables for this study were part of the short daily diary surveys.

\subsubsection{Materials}

\paragraph{Intergroup Contact}

To measure intergroup contacts, every experience sampling measurement
started with the question
``\textit{Did you meet a Dutch person this morning [/afternoon]? (in-person interaction for at least 10 minutes)}''.
Participants were additionally offered a pop-up explanation: ``With
in-person interaction, we mean a continued interaction with another
person (potentially in a group) that lasted at least 10 minutes. This
interaction should be offline and face-to-face. It should include some
form of verbal communication and should be uninterrupted to still count
as the same interaction. Any individual interaction can last minutes or
hours. If there were multiple interaction partners, we would like you to
focus on the person that was most important to you during the
interaction.''. The participants recorded between 1--43 interactions
with Dutch majority people (1.64--93.75\% of individual daily diary
measurements; 18.83\% of all 4965 daily diary responses).

\paragraph{Psychological Needs}

For the core situational need, we asked participants in an open ended
text field:
``\textit{What was your main goal [during the interaction with X / this morning / this afternoon]?}''.
Participants could additionally click on a pop-up explanation: ``Your
main goal during an interaction can vary depending on the interaction.
It could be to connect with friends, to find or provide help, to achieve
academic ambitions, work on your fitness, work for a job, or simply to
get a coffee, just as well as many many other concrete or abstract goals
that are import to you in the moment. It really depends on your
subjective experience of the interaction.''. Then, with reference to the
text entry, we asked how much this core need was fulfilled during the
interaction or the past daytime period:
``\textit{During your interaction with X [this morning / this evening] your goal (-previous text entry-) was fulfilled.}''
on a continuous slider scale ranging from strongly disagree (1) to
strongly agree (100).

The measurement of the self determination needs was identical to study
1.

\paragraph{Perceived Interaction Quality}

The ratings of the perceived interaction quality was identical to study
one.

\paragraph{Outgroup Attitudes}

As in Study 1, attitudes towards the Dutch majority outgroup were again
measured using the feeling thermometer. All survey details are also
available in Online Supplemental Materials A and B.

\subsection{Results}

\subsubsection{Contact Hypothesis}

We tested the most general contact hypothesis in two steps. First, we
assessed whether more intergroup interactions were related to to more
positive outgroup attitudes. Second, we tested whether a potential
positive effect on outgroup attitudes depended on the interaction
quality (jointly with the number of interactions). We find a significant
correlation with average outgroup attitudes for both the total number of
outgroup interactions (\textit{r} = 0.22, \textit{p} = 0.019) and the
number of measurement beeps with an interaction (\textit{r} = 0.28,
\textit{p} = 0.003). This is to say that within our data, participants
with more outgroup interactions did have significantly more positive
outgroup attitudes. This is inconsistent with the results we found in
the first study and might be due to the larger number of participants.

However, we find no significant correlation between the participants'
Average Interaction Quality and their Average Outgroup Attitudes
(\textit{r} = 0.04, \textit{p} = 0.679). Thus, when considering the
number of interactions and average interaction quality jointly in a
linear regression, we find a main effect of intergroup contacts to
predict outgroup attitudes (\textit{b} = 0.81, \textit{t}(109) = 3.29,
\textit{p} = 0.001, \(\eta_p^2\) = 0.08). Given the missing aggregate
relationship between average interaction quality and outgroup attitudes,
we find no significant effect of average perceived contact quality. We
also do not find that in this sample the impact of the number of
interactions is moderated by the average contact quality. This is not
entirely consistent with the first study, where average contact quality
did have a meaningful effect on outgroup attitudes. This finding is not
necessarily surprising given that the variables aggregate all within
person variation and there were substantially more measurements where
participants did not have an interaction (but reported their outgroup
attitudes) than measurements that followed an outgroup contact.

We additionally used a multilevel regression to check whether having an
interaction with an outgroup member had a situational (i.e.,
contemporaneous) effect within the participants. We find that having an
outgroup interaction is indeed associated with significantly more
positive outgroup attitudes within the participants (random slopes
model; \textit{b} = 2.99, \textit{t}(4850) = 3.80, \textit{p}
\textless{} .001, \textit{95\%CI}{[}1.45, 4.54{]}), even after
controlling for having an interaction with a non-Dutch person (which did
not relate to outgroup attitudes independently; For full results see
Online Supplementary Materials B). Thus, in our second dataset we find
mixed results, outgroup contacts show a positive effect on outgroup
attitudes in the moment and on average between participants. However, we
find that this effect does not depend on the average perceived
interaction quality. This unexpected results, might be due to the
aggregation process and the following analyses will focus in on the role
of the perceived interaction quality during an outgroup interaction.

\subsubsection{Core Need}

We again sequentially tested whether the fulfillment of the core need
during an interaction was (1) related to more positive outgroup
attitudes, (2) higher perceived interaction quality, and (3) whether the
variance explained by the core need is assumed by the perceived
interaction quality if considered jointly. We find that in the
multilevel models, the fulfillment of core situational needs during
outgroup contacts was associated with more positive outgroup attitudes
(random slopes model; \textit{b} = 0.13, \textit{t}(826) = 4.18,
\textit{p} \textless{} .001, \textit{95\%CI}{[}0.07, 0.19{]}) and also
predicted higher perceived interaction quality (random slopes model;
\textit{b} = 0.40, \textit{t}(826) = 6.46, \textit{p} \textless{} .001,
\textit{95\%CI}{[}0.28, 0.52{]}). Additionally, if we consider the
influences of core need fulfillment and interaction quality on outgroup
attitudes jointly, we find that virtually all variance is explained by
perceived interaction quality (random slopes mode; \textit{b} = 0.17,
\textit{t}(825) = 5.56, \textit{p} \textless{} .001,
\textit{95\%CI}{[}0.11, 0.23{]}) and only little unique variance is
still explained by core need fulfillment (\textit{b} = 0.03,
\textit{t}(825) = 1.47, \textit{p} = 0.141, \textit{95\%CI}{[}-0.01,
0.07{]}). These results are consistent with the results in study one and
we thus find support for our hypotheses and can conclude that in this
data set the fulfillment of core situational needs had a significant
influence on outgroup attitudes and that this effect is likely explained
by its effect through perceived interaction quality.

\faQuestionCircle~Are we sure that we don't want some sort of
bootstrapped indirect effect?

\subsubsection{Robustness}

We again checked for alternative models. First, when considering
generalized situational core need fulfillment together with whether an
intergroup contact took place, we find that there is only a minuscule
main effect of core need fulfillment (random slopes model; \textit{b} =
-0.03, \textit{t}(4849) = -1.34, \textit{p} = 0.181,
\textit{95\%CI}{[}-0.08, 0.02{]}) but a stronger interaction effect of
core need fulfillment and outgroup contact (\textit{b} = 0.06,
\textit{t}(4849) = 3.03, \textit{p} = 0.002, \textit{95\%CI}{[}0.02,
0.10{]}). Together with a significant main effect of having an outgroup
contact (\textit{b} = 2.88, \textit{t}(4849) = 3.71, \textit{p}
\textless{} .001, \textit{95\%CI}{[}1.36, 4.40{]}), this indicates that
it is not key need fulfillment in general --- but key need fulfillment
during an outgroup contact that predicts more positive outgroup
attitudes. This finding is consistent with the results of the previous
study, albeit with slightly weaker effect (likely because of the large
number of measurements that did not include an outgroup interaction).

In a final step we again checked whether during the interaction the core
situational need remains a meaningful predictor even when taking other
fundamental psychological needs into account. We find that the core need
adds significantly above a model with only the self determination theory
needs (random slopes models; \(\chi^2\)(6, \textit{N} = 108) = 22.90,
\textit{p} \textless{} .001). We find that the core need explains the
most variance in outgroup attitudes after an outgroup contact
(\textit{b} = 0.08, \textit{t}(823) = 2.95, \textit{p} = 0.003,
\textit{95\%CI}{[}0.03, 0.13{]}). When compared to the model with only
the SDT needs, the core need fulfillment flexibly takes on some of the
explained variance of all of the three fundamental needs. However,
different from the first study, relatedness (\textit{b} = 0.07,
\textit{t}(823) = 4.26, \textit{p} \textless{} .001,
\textit{95\%CI}{[}0.04, 0.10{]}) and autonomy (\textit{b} = 0.02,
\textit{t}(823) = 0.93, \textit{p} = 0.353, \textit{95\%CI}{[}-0.02,
0.08{]}) also predicted positive outgroup attitudes in this larger
sample. For full results see Online Supplementary Information B.

\section{Study 3}

The aim of this final study is to extend the previous studies by
additionally testing Allport's conditions in an extensive longitudinal
design and to compare the predictive powers of Allport's conditions and
the core situational need fulfillment. We will, thus, again test the
general contact hypothesis, the influence of core need fulfillment, and
perceived interaction quality during intergroup contacts. However, we
additionally have the opportunity to also assess the role of Allport's
conditions in this and to compare the two approaches (Allport's
conditions and core need fulfillment) directly.

For this study we specifically recruited international medical students,
because they represent a particular group of migrants who face
structural requirements to integrate and interact with Dutch majority
outgroup members on a daily basis. As part of their educational program,
the migrants are required to take language courses and interact with
patients as part of their medical internships and medical residency. The
extensive longitudinal survey method again offers a large body of
ecologically valid data on need satisfaction in real-life intergroup
contact situations. Data were collected from November
8\textsuperscript{th}, 2019 to January 10\textsuperscript{th}, 2020.

The full surveys are available in our Online Supplementary Material A
and the full data description is available in Online Supplementary
Materials B. Correlations and descriptive statistics of the included
variables are available in Tables \ref{tab:descrFullLong} and
\ref{tab:descrOutLong}.

\subsection{Methods}

\subsubsection{Participants}

After receiving ethical approval from the University of Groningen, we
recruited 71 international medical students using a contacts within the
University Medical Department and specifically targeted non-Dutch
students, who had recently arrived in the Netherlands. Participants
reported on their interactions for at least 30 days with two daily
measures (capturing the morning and afternoon). With this design, we
aimed at getting 50-60 measurements per participant (\textit{M} = 57.85,
\textit{SD} = 20.68, \textit{total N} = 4107). As with the previous
study this should offer sufficient power to model processes within
participants and will lend stronger weight to between-participant
results. Participants were compensated for their participation with
partial course credits --- depending on their participation. The sample
consisted of relatively young migrants, who were mostly from the global
north (\(M_{age}\) = 22.68, \(SD_{age}\) = 3.10, 59 women). The sample
fairly accurately describes the local population of international
students.

\subsubsection{Procedure}

The study procedure mirrored the setup of studies one and two, in that
it consisted of pre-, daily diary-, and post-measurement. The
participants were invited for daily diary measurements twice a day (at
12 pm and 7pm) for 30 days (\textit{median duration} = 4M 13S,
\textit{MAD duration} = 2M 6S). General compliance was high (93.30\%
filled daily diary surveys for 31 days or more NEED TO RE-CHECK THIS).
The response rates were approximately equal during mornings (\textit{n}
= 2092) and afternoons (\textit{n} = 2015). All key variables in for
this study were part of the short daily diary surveys.

\subsubsection{Materials}

\paragraph{Intergroup Contact}

The measurement of intergroup contacts was identical to study two. The
participants recorded between 1--71 interactions with Dutch outgroup
members (2.13--87.18\% of individual daily diary measurements; 41.44\%
of all 4107 daily diary responses).

\paragraph{Psychological Needs}

The measurement of the core situational need and its fulfillment was
identical to study two. Similarly, the measurement of the self
determination needs was identical to studies one and two.

\paragraph{Allport's Conditions}

To measure how much each of the interactions fulfilled Allport's
conditions of optimal contact we asked participants to rate how much the
interaction had equal status
(``\textit{The interaction with [name interaction partner] was on equal footing (same status)}''),
a common goal
(``\textit{[name interaction partner] shared your goal ([free-text entry interaction key need])}''),
support of authorities
(``\textit{The interaction with [name interaction partner] was voluntary}''),
and intergroup cooperation
(``\textit{The interaction with [name interaction partner] was cooperative}'').

\paragraph{Perceived Interaction Quality}

The ratings of the perceived interaction quality was identical to study
one.

\paragraph{Outgroup Attitudes}

Attitudes towards the Dutch majority outgroup was again measured using
the feeling thermometer, as in studies one and two. All survey details
are also available in Online Supplemental Materials A and B.

\subsection{Results}

\subsubsection{Contact Hypothesis}

We tested the most general contact hypothesis in two steps. First, we
assessed whether more intergroup interactions were related to to more
positive outgroup attitudes. Second, we tested whether a potential
positive effect on outgroup attitudes depended on the interaction
quality (jointly with the number of interactions). We find no
significant correlation between intergroup contact and average outgroup
attitudes --- for neither the total number of outgroup interactions
(\textit{r} = 0.01, \textit{p} = 0.946) and the number of measurement
beeps with an interaction (\textit{r} = 0.13, \textit{p} = 0.291). This
is to say that within our data, participants with more outgroup
interactions did not have significantly more positive or negative
outgroup attitudes. This is again consistent with the results we found
in the first study but inconsitent with the significant relationship we
find in the second study.

However, we do find a significant correlation between the participants'
Average Interaction Quality and their Average Outgroup Attitudes
(\textit{r} = 0.25, \textit{p} = 0.036). Thus, when considering the
number of interactions and average interaction quality jointly in a
linear regression, we only find a main effect of average perceived
interaction quality on predict outgroup attitudes (\textit{b} = 0.36,
\textit{t}(67) = 2.22, \textit{p} = 0.030, \(\eta_p^2\) = 0.07). Given
the missing aggregate relationship between number of interactions and
outgroup attitudes, we find no significant effect of interactions
themselves nor do we find a significant interaction effect. This result
mirrors that of study one but is inconsistent with the second study. But
again this inconsistency is not necessarily surprising given that the
variables aggregate all within person variation and there were
substantially more measurements where participants did not have an
interaction (but reported their outgroup attitudes) than measurements
that followed an outgroup contact.

We additionally used a multilevel regression to check whether having an
interaction with an outgroup member had a situational (i.e.,
contemporaneous) effect within the participants. We find that having an
outgroup interaction is indeed associated with significantly more
positive outgroup attitudes within the participants (random slopes
model; \textit{b} = 5.63, \textit{t}(3834) = 6.61, \textit{p}
\textless{} .001, \textit{95\%CI}{[}3.97, 7.29{]}), even after
controlling for having an interaction with a non-Dutch (which did not
relate to outgroup attitudes independently; For full results see Online
Supplementary Materials B). Thus, in our third dataset we find some
mixed results, outgroup contacts are not significantly related to
outgroup attitudes on aggregate between participants and this effect is
also not suppressed by a average interaction quality (i.e., no
interaction effect). However, while the aggregate data is partly
inconsistent with the first two studies, the within person contemoranous
effect of intergroup contact is consistent across all three studies.

\subsubsection{Allport's Conditions}

We sequentially tested whether the fulfillment of Allport's Contact
Conditions was (1) related to more positive outgroup attitudes, (2)
higher perceived interaction quality, and (3) whether the variance
explained by Allport's Conditions is assumed by the perceived
interaction quality if considered jointly. We find that in the
multilevel models, the fulfillment of Allport's Conditions during
outgroup contacts was associated with more positive outgroup attitudes
(random slopes model; \textit{b} = 0.22, \textit{t}(1601) = 5.86,
\textit{p} \textless{} .001, \textit{95\%CI}{[}0.15, 0.29{]}) and also
predicted higher perceived interaction quality (random slopes model;
\textit{b} = 0.65, \textit{t}(1605) = 11.82, \textit{p} \textless{}
.001, \textit{95\%CI}{[}0.54, 0.76{]}). Moreover, when we consider the
influences of Allport's Conditions and interaction quality on outgroup
attitudes jointly, we find that perceived interaction quality is a
substantially stronger predictor (random slopes mode; \textit{b} = 0.17,
\textit{t}(1600) = 6.25, \textit{p} \textless{} .001,
\textit{95\%CI}{[}0.12, 0.23{]}) and the unique variance explained by
Allport's Conditions was less than half of its original effect size
(\textit{b} = 0.09, \textit{t}(1600) = 2.89, \textit{p} = 0.004,
\textit{95\%CI}{[}0.03, 0.15{]}). These results indicate that in this
dataset the the fulfillment of Allport's conditions had a significant
influence on outgroup attitudes and this effect is likely in parts
explained by its effect through perceived interaction quality.

\faQuestionCircle~Are we sure that we don't want some sort of
bootstrapped indirect effect?

\subsubsection{Core Need}

We again sequentially tested whether the fulfillment of the core need
during an interaction was (1) related to more positive outgroup
attitudes, (2) higher perceived interaction quality, and (3) whether the
variance explained by the core need is assumed by the perceived
interaction quality if considered jointly. We find that in the
multilevel models, the fulfillment of core need fulfillment during
outgroup contacts was associated with more positive outgroup attitudes
(random slopes model; \textit{b} = 0.19, \textit{t}(1601) = 5.29,
\textit{p} \textless{} .001, \textit{95\%CI}{[}0.12, 0.27{]}) and also
predicted higher perceived interaction quality (random slopes model;
\textit{b} = 0.43, \textit{t}(1605) = 8.35, \textit{p} \textless{} .001,
\textit{95\%CI}{[}0.33, 0.54{]}). Additionally, once we consider the
influences of core need fulfillment and interaction quality on outgroup
attitudes jointly, we find that perceived interaction quality is a
substantially stronger predictor (random slopes mode; \textit{b} = 0.17,
\textit{t}(1600) = 6.86, \textit{p} \textless{} .001,
\textit{95\%CI}{[}0.12, 0.22{]}) and the unique variance explained by
core need fulfillment was roughly half of its original effect size
(\textit{b} = 0.11, \textit{t}(1600) = 3.50, \textit{p} \textless{}
.001, \textit{95\%CI}{[}0.05, 0.17{]}). These results indicate that in
this dataset the the fulfillment of core situational needs had a
significant influence on outgroup attitudes and this effect is likely in
parts explained by its effect through perceived interaction quality.

\faQuestionCircle~Are we sure that we don't want some sort of
bootstrapped indirect effect?

\subsubsection{Compare Fulfillment of Core Need and Allport's Conditions}

To compare the models using either Allport's conditions or the core need
fulfillment to predict outgroup attitudes, we first assess relative
model performance indices (i.e., Akaike information criterion, and
Bayesian information criterion), and then consider the two predictors in
a joint model to see whether the two approaches predict the same
variance in outgroup attitudes. When comparing the model selection
indices we find that the fulfillment of the situation core need, indeed
performs slightly better than the model using Allport's conditions
(\(AIC_{CoreNeed}\) 12632.02 \textless{} 12651.59 \(AIC_{Allport}\), and
\(BIC_{CoreNeed}\) 12664.55 \textless{} 12684.12 \(BIC_{Allport}\)).
Additionally, when considering the predictors jointly, we find that both
significantly predict outgroup attitudes with similar sized regression
parameters (random slopes model; Allport's Conditions: \textit{b} =
0.16, \textit{t}(1600) = 4.92, \textit{p} \textless{} .001,
\textit{95\%CI}{[}0.09, 0.24{]}, Core Need: \textit{b} = 0.14,
\textit{t}(1600) = 3.85, \textit{p} \textless{} .001,
\textit{95\%CI}{[}0.08, 0.17{]}). This indicates that, although both
Allport's conditions and the core need fulfillment seem to (in part)
work through perceived interaction quality, they explain different
aspects of the variance in outgroup attitudes and do not constitute one
another.

\subsubsection{Robustness}

We again checked for alternative models to the core need fulfillment.
First, when considering generalized situational core need fulfillment
together with whether an intergroup contact took place, we find that
there is no significant main effect of core need fulfillment (random
slopes model; \textit{b} = 0.03, \textit{t}(3835) = 1.51, \textit{p} =
0.131, \textit{95\%CI}{[}-0.01, 0.06{]}) but a stronger interaction
effect of core need fulfillment and outgroup contact (\textit{b} = 0.17,
\textit{t}(3835) = 7.32, \textit{p} \textless{} .001,
\textit{95\%CI}{[}0.12, 0.21{]}). Together with a significant main
effect of having an outgroup contact (\textit{b} = 5.41,
\textit{t}(3835) = 6.36, \textit{p} \textless{} .001,
\textit{95\%CI}{[}3.74, 7.08{]}), this indicates that it is not key need
fulfillment in general --- but key need fulfillment during an outgroup
contact that predicts more positive outgroup attitudes. This finding is
consistent with the results of the previous studies.

In a final step we again checked whether during the interaction the core
situational need remains a meaningful predictor even when taking other
fundamental psychological needs into account. We find that the core need
adds significantly above a model with only the self determination theory
needs (random slopes models; \(\chi^2\)(6, \textit{N} = 70) = 100.20,
\textit{p} \textless{} .001). We find that the core need explains the
most variance in outgroup attitudes after an outgroup contact
(\textit{b} = 0.15, \textit{t}(1598) = 3.88, \textit{p} \textless{}
.001, \textit{95\%CI}{[}0.07, 0.22{]}). When compared to the model with
only the SDT needs, the core need fulfillment flexibly takes on some of
the explained variance of all of the three fundamental needs. However,
similar to the previous study, in this large sample relatedness
(\textit{b} = 0.05, \textit{t}(1598) = 3.31, \textit{p} \textless{}
.001, \textit{95\%CI}{[}0.02, 0.09{]}), competence (\textit{b} = 0.06,
\textit{t}(1598) = 2.73, \textit{p} = 0.006, \textit{95\%CI}{[}0.02,
0.10{]}) and autonomy (\textit{b} = 0.04, \textit{t}(1598) = 2.12,
\textit{p} = 0.034, \textit{95\%CI}{[}-0.01, 0.08{]}) each also
predicted positive outgroup attitudes independently. However, the
regression coefficient is three times as large for the core need
fulfillment (with all scaling being equal). For full results see Online
Supplementary Information B.

\begin{table}
\begin{minipage}[t][\textheight][t]{\textwidth}

\caption{\label{tab:descrFullLong}Full Sample: Correlation Table and Descriptive Statistics}
\centering
\resizebox{\linewidth}{!}{
\begin{tabular}[t]{lccccccc}
\toprule
Variable & Core Need & Competence & Autonomy & Relatedness & Quality & Attitudes NL & Allport\\
\midrule
\addlinespace[0.3em]
\multicolumn{8}{l}{\textbf{Study 1}}\\
\hspace{1em}Core Need &  & 0.63*** & 0.47** & 0.00 & 0.40* & -0.22 & \\
\hspace{1em}Competence & 0.37*** &  & 0.68*** & -0.13 & 0.17 & -0.26 & \\
\hspace{1em}Autonomy & 0.31*** & 0.08** &  & -0.04 & 0.64*** & 0.16 & \\
\hspace{1em}Relatedness & 0.45*** & 0.41*** & 0.22*** &  & 0.02 & -0.07 & \\
\hspace{1em}Quality & 0.05 & 0.45*** & 0.11*** & 0.12*** &  & 0.47* & \\
\hspace{1em}Attitudes NL & 0.10*** & 0.34*** & 0.14*** & 0.14*** & 0.43*** &  & \\
\addlinespace
\hspace{1em}Grand Mean & 27.95 & 12.10 & 22.17 & 5.29 & 24.10 & 71.49 & \\
\hspace{1em}Between SD & 14.68 & 13.72 & 12.09 & 14.59 & 9.50 & 12.91 & \\
\hspace{1em}Within SD & 20.83 & 20.89 & 15.15 & 23.29 & 18.01 & 8.11 & \\
\hspace{1em}ICC(1) & 0.29 & 0.28 & 0.38 & 0.28 & 0.18 & 0.70 & \\
\hspace{1em}ICC(2) & 0.96 & 0.95 & 0.97 & 0.95 & 0.79 & 0.99 & \\
\addlinespace[0.3em]
\multicolumn{8}{l}{\textbf{Study 2}}\\
\hspace{1em}Core Need &  & 0.60*** & 0.66*** & 0.43*** & 0.78*** & 0.11 & \\
\hspace{1em}Competence & 0.24*** &  & 0.71*** & 0.65*** & 0.74*** & 0.09 & \\
\hspace{1em}Autonomy & 0.15*** & 0.34*** &  & 0.56*** & 0.67*** & -0.06 & \\
\hspace{1em}Relatedness & 0.45*** & 0.29*** & 0.42*** &  & 0.54*** & -0.11 & \\
\hspace{1em}Quality & 0.17*** & 0.39*** & 0.08*** & 0.05** &  & 0.10 & \\
\hspace{1em}Attitudes NL & 0.35*** & 0.43*** & 0.11*** & 0.10*** & 0.12*** &  & \\
\addlinespace
\hspace{1em}Grand Mean & 84.87 & 72.55 & 82.59 & 61.21 & 83.77 & 67.26 & \\
\hspace{1em}Between SD & 9.17 & 14.47 & 11.21 & 13.36 & 9.12 & 18.64 & \\
\hspace{1em}Within SD & 20.33 & 21.17 & 16.06 & 28.74 & 16.80 & 9.40 & \\
\hspace{1em}ICC(1) & 0.15 & 0.30 & 0.32 & 0.17 & 0.20 & 0.80 & \\
\hspace{1em}ICC(2) & 0.89 & 0.95 & 0.95 & 0.90 & 0.88 & 0.99 & \\
\addlinespace[0.3em]
\multicolumn{8}{l}{\textbf{Study 3}}\\
\hspace{1em}Core Need &  & 0.49*** & 0.58*** & 0.29* & 0.60*** & 0.10 & 0.60***\\
\hspace{1em}Competence & 0.27*** &  & 0.79*** & 0.58*** & 0.52*** & 0.10 & 0.63***\\
\hspace{1em}Autonomy & 0.31*** & 0.43*** &  & 0.53*** & 0.67*** & 0.09 & 0.57***\\
\hspace{1em}Relatedness & 0.55*** & 0.40*** & 0.38*** &  & 0.50*** & 0.23 & 0.40***\\
\hspace{1em}Allport & 0.20*** & 0.46*** & 0.51*** & 0.10*** & 0.70*** & 0.25* & \\
\hspace{1em}Quality & 0.39*** & 0.45*** & 0.44*** & 0.06** &  & 0.23* & -0.03\\
\hspace{1em}Attitudes NL & 0.51*** & 0.37*** & 0.55*** & 0.01 & 0.12*** &  & 0.05*\\
\addlinespace
\hspace{1em}Grand Mean & 83.57 & 77.45 & 83.76 & 63.44 & 84.26 & 64.77 & 86.74\\
\hspace{1em}Between SD & 8.02 & 11.49 & 9.72 & 13.34 & 10.40 & 14.37 & 7.08\\
\hspace{1em}Within SD & 17.14 & 18.92 & 15.87 & 28.85 & 15.91 & 10.88 & 11.87\\
\hspace{1em}ICC(1) & 0.18 & 0.26 & 0.28 & 0.17 & 0.29 & 0.66 & 0.25\\
\hspace{1em}ICC(2) & 0.92 & 0.95 & 0.96 & 0.92 & 0.95 & 0.99 & 0.95\\
\addlinespace[0.3em]
\multicolumn{8}{l}{\textbf{Across Studies}}\\
\hspace{1em}Core Need &  & 0.91*** & 0.92*** & 0.85*** & 0.94*** & -0.04 & \\
\hspace{1em}Competence & 0.25*** &  & 0.94*** & 0.89*** & 0.92*** & -0.03 & \\
\hspace{1em}Autonomy & 0.22*** & 0.41*** &  & 0.88*** & 0.93*** & -0.08 & \\
\hspace{1em}Relatedness & 0.50*** & 0.33*** & 0.42*** &  & 0.88*** & -0.06 & \\
\hspace{1em}Quality & 0.18*** & 0.39*** & 0.10*** & 0.04*** &  & 0.00 & \\
\hspace{1em}Attitudes NL & 0.36*** & 0.46*** & 0.09*** & 0.05*** & 0.14*** &  & \\
\addlinespace
\hspace{1em}Grand Mean & 78.10 & 67.51 & 76.28 & 55.76 & 77.82 & 66.88 & \\
\hspace{1em}Between SD & 20.17 & 23.86 & 22.00 & 22.39 & 20.56 & 16.76 & \\
\hspace{1em}Within SD & 19.35 & 20.40 & 15.90 & 28.23 & 16.64 & 9.81 & \\
\hspace{1em}ICC(1) & 0.52 & 0.58 & 0.66 & 0.38 & 0.60 & 0.75 & \\
\hspace{1em}ICC(2) & 0.98 & 0.99 & 0.99 & 0.97 & 0.98 & 0.99 & \\
\bottomrule
\multicolumn{8}{l}{\rule{0pt}{1em}\textit{Note: }}\\
\multicolumn{8}{l}{\rule{0pt}{1em}Upper triangle: Between-person correlations;}\\
\multicolumn{8}{l}{\rule{0pt}{1em}Lower triangle: Within-person correlations;}\\
\multicolumn{8}{l}{\rule{0pt}{1em}*** p < .001, ** p < .01,  * p < .05}\\
\end{tabular}}
\end{minipage}
\end{table}


\begin{tabular}[t]{lccccccc}
\toprule
Variable & Core Need & Competence & Autonomy & Relatedness & Quality & Attitudes NL & Allport\\
\midrule
\addlinespace[0.3em]
\multicolumn{8}{l}{\textbf{Study 1}}\\
\hspace{1em}Core Need &  & 0.82*** & 0.60*** & 0.32 & 0.52** & -0.03 & \\
\hspace{1em}Competence & 0.36*** &  & 0.89*** & 0.24 & 0.38 & -0.24 & \\
\hspace{1em}Autonomy & 0.28*** & 0.23*** &  & 0.29 & 0.56** & 0.02 & \\
\hspace{1em}Relatedness & 0.51*** & 0.39*** & 0.37*** &  & -0.09 & 0.14 & \\
\hspace{1em}Quality & 0.17*** & 0.44*** & 0.27*** & 0.27*** &  & 0.50* & \\
\hspace{1em}Attitudes NL & 0.24*** & 0.37*** & 0.24*** & 0.37*** & 0.52*** &  & \\
\addlinespace
\hspace{1em}Grand Mean & 32.20 & 16.53 & 22.64 & 1.58 & 24.10 & 72.46 & \\
\hspace{1em}Between-person SD & 12.42 & 13.41 & 10.22 & 14.16 & 9.50 & 13.62 & \\
\hspace{1em}Within-person SD & 17.66 & 18.78 & 14.93 & 23.97 & 18.85 & 9.50 & \\
\hspace{1em}ICC(1) & 0.33 & 0.31 & 0.30 & 0.21 & 0.18 & 0.68 & \\
\hspace{1em}ICC(2) & 0.90 & 0.89 & 0.89 & 0.83 & 0.80 & 0.98 & \\
\addlinespace[0.3em]
\multicolumn{8}{l}{\textbf{Study 2}}\\
\hspace{1em}Core Need &  & 0.44** & 0.49*** & -0.22 & 0.51*** & 0.14 & \\
\hspace{1em}Competence & 0.23*** &  & 0.58*** & 0.40** & 0.55*** & 0.21 & \\
\hspace{1em}Autonomy & 0.12*** & 0.24*** &  & 0.22 & 0.61*** & 0.05 & \\
\hspace{1em}Relatedness & 0.48*** & 0.31*** & 0.38*** &  & 0.35* & -0.09 & \\
\hspace{1em}Quality & 0.19*** & 0.40*** & 0.16*** & 0.15*** &  & 0.16 & \\
\hspace{1em}Attitudes NL & 0.32*** & 0.36*** & 0.19*** & 0.23*** & 0.30*** &  & \\
\addlinespace
\hspace{1em}Grand Mean & 86.86 & 73.23 & 78.58 & 60.30 & 78.80 & 70.41 & \\
\hspace{1em}Between-person SD & 11.20 & 13.95 & 14.07 & 17.35 & 10.71 & 17.13 & \\
\hspace{1em}Within-person SD & 15.87 & 16.81 & 14.24 & 26.14 & 17.88 & 9.87 & \\
\hspace{1em}ICC(1) & 0.14 & 0.27 & 0.40 & 0.19 & 0.14 & 0.72 & \\
\hspace{1em}ICC(2) & 0.58 & 0.76 & 0.85 & 0.67 & 0.59 & 0.96 & \\
\addlinespace[0.3em]
\multicolumn{8}{l}{\textbf{Study 3}}\\
\hspace{1em}Core Need &  & 0.52*** & 0.57*** & 0.12 & 0.63*** & 0.25* & 0.58***\\
\hspace{1em}Competence & 0.23*** &  & 0.79*** & 0.42*** & 0.57*** & 0.32** & 0.60***\\
\hspace{1em}Autonomy & 0.26*** & 0.37*** &  & 0.41*** & 0.61*** & 0.32** & 0.44***\\
\hspace{1em}Relatedness & 0.52*** & 0.33*** & 0.31*** &  & 0.40*** & 0.38*** & 0.34**\\
\hspace{1em}Allport & 0.14*** & 0.37*** & 0.41*** & 0.24*** & 0.71*** & 0.44*** & \\
\hspace{1em}Quality & 0.33*** & 0.36*** & 0.39*** & 0.20*** &  & 0.48*** & 0.20***\\
\hspace{1em}Attitudes NL & 0.43*** & 0.34*** & 0.48*** & 0.20*** & 0.34*** &  & 0.23***\\
\addlinespace
\hspace{1em}Grand Mean & 84.84 & 75.94 & 79.07 & 59.62 & 81.14 & 68.24 & 80.87\\
\hspace{1em}Between-person SD & 9.27 & 12.23 & 12.88 & 19.26 & 12.38 & 13.72 & 10.87\\
\hspace{1em}Within-person SD & 13.00 & 17.21 & 15.26 & 23.45 & 16.25 & 11.23 & 12.14\\
\hspace{1em}ICC(1) & 0.30 & 0.29 & 0.36 & 0.34 & 0.33 & 0.63 & 0.42\\
\hspace{1em}ICC(2) & 0.91 & 0.91 & 0.93 & 0.93 & 0.92 & 0.98 & 0.95\\
\bottomrule
\end{tabular}



\begin{landscape}\begin{table}
\begin{minipage}[t][\textheight][t]{\textwidth}

\caption{Intergroup General}
\centering
\resizebox{\linewidth}{!}{
\begin{tabular}[t]{lcccccc}
\toprule
\multicolumn{1}{c}{} & \multicolumn{2}{c}{Study 1} & \multicolumn{2}{c}{Study 2} & \multicolumn{2}{c}{Study 3} \\
\cmidrule(l{3pt}r{3pt}){2-3} \cmidrule(l{3pt}r{3pt}){4-5} \cmidrule(l{3pt}r{3pt}){6-7}
 & $B$ & Effect Size & $B$ & Effect Size & $B$ & Effect Size\\
\midrule
\addlinespace[0.3em]
\multicolumn{7}{l}{\textbf{Between Participant [aggregated linear regression] (Effect Size = $\eta_{partial}$)}}\\
\hspace{1em}(Intercept) & 70.50**** [66.06, 74.94] &  & 67.62**** [64.16, 71.08] &  & 64.83**** [61.50, 68.16] & \\
\hspace{1em}SumContactNL & 0.27 [-0.15,  0.69] & 0.02 & 0.81** [ 0.32,  1.29] & 0.08 & 0.15 [-0.08,  0.38] & 0.02\\
\hspace{1em}AvQuality & 0.84** [ 0.36,  1.33] & 0.39 & 0.25 [-0.14,  0.64] & 0.01 & 0.36* [ 0.04,  0.68] & 0.07\\
\hspace{1em}SumContactNL:AvQuality & -0.05** [-0.08, -0.01] & 0.34 & 0.02 [-0.03,  0.08] & 0.01 & 0.01 [-0.02,  0.03] & 0.01\\
\hspace{1em}$R^2 / R^2_{adjusted}$ & 0.540 / 0.459 &  & 0.094 / 0.069 &  & 0.088 / 0.047 & \\
\addlinespace[0.3em]
\multicolumn{7}{l}{\textbf{Within Participants [multilevel linear regression] (Effect Size = $\beta$)}}\\
\hspace{1em}(Intercept) & 67.44**** [61.71, 73.17] &  & 63.72**** [59.42, 68.01] &  & 56.30**** [51.58, 61.02] & \\
\hspace{1em}OutgroupInteraction & 2.48**** [ 1.36,  3.59] & 1.07 [ 0.59, 1.54] & 2.99*** [ 1.45,  4.54] & 0.11 [ 0.07, 0.16] & 5.63**** [ 3.97,  7.29] & 0.21 [ 0.15, 0.27]\\
\hspace{1em}NonOutgroupInteraction & 0.43 [-0.68,  1.54] & 0.16 [-0.32, 0.63] & 0.01 [-0.76,  0.78] & 0.01 [-0.03, 0.05] & 0.34 [-0.49,  1.18] & 0.02 [-0.02, 0.05]\\
\hspace{1em}$R^2_{marginal} / R^2_{conditional}$ & 0.006 / 0.703 &  & 0.003 / 0.817 &  & 0.026 / 0.708 & \\
\bottomrule
\multicolumn{7}{l}{\rule{0pt}{1em}\textit{Note: }}\\
\multicolumn{7}{l}{\rule{0pt}{1em}**** $p$ < .0001, *** $p$ < .001, ** $p$ < .01, * $p$ < .05}\\
\end{tabular}}
\end{minipage}
\end{table}
\end{landscape}



\begin{landscape}\begin{table}
\begin{minipage}[t][\textheight][t]{\textwidth}

\caption{\label{tab:intergroupNeedsTblLong}Theoretical Test: Core Need Fulfillment and Allport's Conditions}
\centering
\resizebox{\linewidth}{!}{
\begin{tabular}[t]{lcccccccc}
\toprule
\multicolumn{1}{c}{ } & \multicolumn{2}{c}{Quality} & \multicolumn{6}{c}{Attitude} \\
\cmidrule(l{3pt}r{3pt}){2-3} \cmidrule(l{3pt}r{3pt}){4-9}
coef & B & Beta & B & Beta & B & Beta & B & Beta\\
\midrule
\addlinespace[0.3em]
\multicolumn{9}{l}{\textbf{Study 1: Needs}}\\
\hspace{1em}(Intercept) & 24.28**** [20.20, 28.36] &  & 72.53**** [66.83, 78.24] &  & 72.50**** [66.75, 78.24] &  &  & \\
\hspace{1em}CoreNeed & 0.42**** [ 0.32,  0.52] & 0.40 [ 0.31, 0.49] & 0.17** [ 0.06,  0.29] & 0.30 [ 0.18, 0.41] & 0.04 [-0.08,  0.15] & 0.13 [ 0.00, 0.26] &  & \\
\hspace{1em}Quality &  &  &  &  & 0.23*** [ 0.12,  0.33] & 0.41 [ 0.29, 0.54] &  & \\
\hspace{1em}$R^2_{marginal}$ / $R^2_{conditional}$ & 0.117 / 0.306 &  & 0.030 / 0.752 &  & 0.067 / 0.851 &  &  & \\
\addlinespace[0.3em]
\multicolumn{9}{l}{\textbf{Study 2: Needs}}\\
\hspace{1em}(Intercept) & 78.84**** [76.86, 80.83] &  & 70.71**** [67.55, 73.87] &  & 70.67**** [67.50, 73.84] &  &  & \\
\hspace{1em}CoreNeed & 0.40**** [ 0.28,  0.52] & 0.27 [ 0.18, 0.37] & 0.13*** [ 0.07,  0.19] & 0.14 [ 0.06, 0.22] & 0.03 [-0.01,  0.07] & 0.07 [-0.01, 0.15] &  & \\
\hspace{1em}Quality &  &  &  &  & 0.17**** [ 0.11,  0.23] & 0.24 [ 0.14, 0.34] &  & \\
\hspace{1em}$R^2_{marginal}$ / $R^2_{conditional}$ & 0.102 / 0.331 &  & 0.012 / 0.750 &  & 0.028 / 0.789 &  &  & \\
\addlinespace[0.3em]
\multicolumn{9}{l}{\textbf{Study 3: Needs}}\\
\hspace{1em}(Intercept) & 81.36**** [78.53, 84.19] &  & 68.32**** [65.10, 71.54] &  & 68.32**** [65.09, 71.54] &  &  & \\
\hspace{1em}CoreNeed & 0.43**** [ 0.33,  0.54] & 0.31 [ 0.24, 0.38] & 0.19**** [ 0.12,  0.27] & 0.18 [ 0.11, 0.24] & 0.11** [ 0.05,  0.17] & 0.11 [ 0.05,  0.17] &  & \\
\hspace{1em}Quality &  &  &  &  & 0.17**** [ 0.12,  0.22] & 0.17 [ 0.12,  0.22] &  & \\
\hspace{1em}$R^2_{marginal}$ / $R^2_{conditional}$ & 0.070 / 0.436 &  & 0.019 / 0.680 &  & 0.041 / 0.713 &  &  & \\
\addlinespace[0.3em]
\multicolumn{9}{l}{\textbf{Study 3: Allport}}\\
\hspace{1em}(Intercept) & 81.42**** [78.57, 84.26] &  & 68.36**** [65.16, 71.56] &  & 68.34**** [65.12, 71.56] &  &  & \\
\hspace{1em}Allport & 0.65**** [ 0.54,  0.76] & 0.46 [ 0.39, 0.52] & 0.22**** [ 0.15,  0.29] & 0.21 [ 0.14, 0.27] & 0.09** [ 0.03,  0.15] & 0.10 [ 0.03, 0.16] &  & \\
\hspace{1em}Quality &  &  &  &  & 0.17**** [ 0.12,  0.23] & 0.22 [ 0.15, 0.29] &  & \\
\hspace{1em}$R^2_{marginal}$ / $R^2_{conditional}$ & 0.149 / 0.526 &  & 0.024 / 0.676 &  & 0.040 / 0.709 &  &  & \\
\addlinespace[0.3em]
\multicolumn{9}{l}{\textbf{Study 3: Needs \& Allport}}\\
\hspace{1em}(Intercept) &  &  &  &  &  &  & 68.33**** [65.05, 71.61] & \\
\hspace{1em}CoreNeed &  &  &  &  &  &  & 0.13**** [ 0.08,  0.17] & 0.12 [ 0.05, 0.18]\\
\hspace{1em}Allport &  &  &  &  &  &  & 0.17*** [ 0.09,  0.24] & 0.17 [ 0.11, 0.23]\\
\hspace{1em}$R^2_{marginal}$ / $R^2_{conditional}$ &  &  &  &  &  &  & 0.086 / NA & \\
\bottomrule
\end{tabular}}
\end{minipage}
\end{table}
\end{landscape}


\begin{table}
\begin{minipage}[t][\textheight][t]{\textwidth}

\caption{\label{tab:robustnessTblLong}Robustness Analyses}
\centering
\resizebox{\linewidth}{!}{
\begin{tabular}[t]{lcccccccc}
\toprule
\multicolumn{1}{c}{ } & \multicolumn{2}{c}{Overall} & \multicolumn{2}{c}{Study 1} & \multicolumn{2}{c}{Study 2} & \multicolumn{2}{c}{Study 3} \\
\cmidrule(l{3pt}r{3pt}){2-3} \cmidrule(l{3pt}r{3pt}){4-5} \cmidrule(l{3pt}r{3pt}){6-7} \cmidrule(l{3pt}r{3pt}){8-9}
 & $B$ & $\beta$ & $B$ & $\beta$ & $B$ & $\beta$ & $B$ & $\beta$\\
\midrule
\addlinespace[0.3em]
\multicolumn{9}{l}{\textbf{Contact specific}}\\
\hspace{1em}(Intercept) & 65.76** [ 49.86, 72.04] &  & 71.60**** [66.33, 76.86] &  & 68.22**** [64.95, 71.49] &  & 65.17**** [62.00, 68.34] & \\
\hspace{1em}CoreNeed & 0.62 [-13.92, 14.10] & 0.01 [-0.44, 0.45] & 0.09** [ 0.04,  0.14] & 0.21 [0.13, 0.28] & 0.06**** [ 0.04,  0.08] & 0.13 [0.08, 0.17] & 0.11**** [ 0.08,  0.14] & 0.13 [0.09, 0.17]\\
\hspace{1em}OutgroupInteraction & 3.59 [-20.13,  7.99] & -0.60 [-1.74, 0.53] & 1.81* [ 0.25,  3.37] & 0.28 [0.05, 0.50] & 2.88*** [ 1.35,  4.40] & 0.23 [0.16, 0.31] & 5.41**** [ 3.72,  7.10] & 0.37 [0.27, 0.48]\\
\hspace{1em}CoreNeed:OutgroupInteraction & 2.44**** [  0.09,  0.14] & 0.01 [ 0.01, 0.01] & 0.13**** [ 0.07,  0.18] & 0.32 [0.19, 0.45] & 0.06** [ 0.02,  0.10] & 0.14 [0.06, 0.22] & 0.17**** [ 0.12,  0.21] & 0.19 [0.12, 0.26]\\
\hspace{1em}$R^2_{marginal}$ / $R^2_{conditional}$ & 0.000 / 1.000 &  & 0.017 / 0.727 &  & 0.005 / 0.818 &  & 0.037 / 0.712 & \\
\addlinespace[0.3em]
\multicolumn{9}{l}{\textbf{Interaction Intent}}\\
\hspace{1em}(Intercept) & 70.14**** [42.53, 65.97] &  & 71.43**** [65.57, 77.29] &  & 70.40**** [67.36, 73.45] &  & 68.16**** [64.95, 71.36] & \\
\hspace{1em}CoreNeed & 2.87**** [-8.58,  8.95] & 0.02 [-1.22, 1.25] & 0.17** [ 0.06,  0.29] & 0.42 [ 0.26, 0.58] & 0.13*** [ 0.07,  0.20] & 0.25 [ 0.12, 0.37] & 0.19**** [ 0.12,  0.26] & 0.24 [ 0.15, 0.32]\\
\hspace{1em}InteractionAccidental & 0.12 [-9.82,  9.92] & 0.00 [-1.07, 1.07] & 0.03 [ 0.00,  0.06] & 0.13 [ 0.02, 0.24] & 0.01 [-0.02,  0.03] & -0.02 [-0.09, 0.05] & -0.01 [-0.04,  0.01] & -0.02 [-0.08, 0.03]\\
\hspace{1em}CoreNeed:InteractionAccidental & -0.27 [ 0.00,  0.00] & 0.00 [ 0.00, 0.00] & 0.00 [ 0.00,  0.00] & 0.00 [-0.11, 0.11] & 0.00 [ 0.00,  0.00] & -0.01 [-0.08, 0.07] & 0.00 [ 0.00,  0.00] & -0.04 [-0.09, 0.02]\\
\hspace{1em}$R^2_{marginal}$ / $R^2_{conditional}$ & 0.000 / 1.000 &  & 0.135 / NA &  & 0.015 / 0.728 &  & 0.021 / 0.687 & \\
\addlinespace[0.3em]
\multicolumn{9}{l}{\textbf{Well-being Outcome}}\\
\hspace{1em}(Intercept) & 65.25*** [ 28.58, 63.85] &  & 61.34**** [55.56, 67.12] &  & 67.18**** [64.66, 69.70] &  & 63.63**** [60.85, 66.41] & \\
\hspace{1em}CoreNeed & 3.50** [-16.18, 16.62] & 0.01 [-1.33, 1.35] & 0.25** [ 0.10,  0.40] & 0.31 [0.17, 0.46] & 0.10** [ 0.04,  0.16] & 0.17 [0.11, 0.23] & 0.20*** [ 0.11,  0.29] & 0.17 [0.10, 0.25]\\
\hspace{1em}$R^2_{marginal}$ / $R^2_{conditional}$ & 0.000 / 1.000 &  & 0.054 / 0.541 &  & 0.008 / 0.348 &  & 0.017 / 0.375 & \\
\addlinespace[0.3em]
\multicolumn{9}{l}{\textbf{Need Type}}\\
\hspace{1em}(Intercept) & 69.40**** [50.54, 64.07] &  & 73.09**** [66.87, 79.31] &  & 70.37**** [67.24, 73.51] &  & 68.18**** [64.77, 71.60] & \\
\hspace{1em}CoreNeed & 2.27** [ 0.09,  0.21] & 0.01 [-0.82, 0.84] & 0.14* [ 0.03,  0.26] & 0.29 [ 0.12, 0.46] & 0.14** [ 0.06,  0.21] & 0.15 [ 0.04, 0.26] & 0.19*** [ 0.08,  0.29] & 0.18 [ 0.08, 0.29]\\
\hspace{1em}practical need & 0.01 [-4.82,  7.25] & -0.11 [-1.00, 0.78] & -0.04 [-2.76,  2.69] & -0.06 [-0.33, 0.20] & 0.80 [-1.22,  2.82] & 0.17 [ 0.01, 0.32] & -0.99 [-2.85,  0.86] & -0.02 [-0.16, 0.12]\\
\hspace{1em}psychological need & 1.45 [-8.48,  4.45] & -0.13 [-1.11, 0.84] & 1.53 [-3.66,  6.73] & 0.08 [-0.28, 0.43] & 1.50 [-0.98,  3.98] & 0.13 [-0.03, 0.30] & 1.25 [-0.42,  2.93] & 0.10 [-0.04, 0.24]\\
\hspace{1em}CoreNeed:practical need & -0.23 [-0.08,  0.05] & 0.00 [ 0.00, 0.01] & 0.01 [-0.13,  0.15] & -0.12 [-0.36, 0.12] & -0.02 [-0.15,  0.10] & -0.04 [-0.23, 0.14] & -0.05 [-0.18,  0.08] & 0.03 [-0.11, 0.18]\\
\hspace{1em}CoreNeed:psychological need & 0.67 [-0.03,  0.11] & 0.00 [ 0.00, 0.01] & -0.10 [-0.31,  0.11] & -0.02 [-0.33, 0.29] & 0.04 [-0.07,  0.15] & 0.04 [-0.13, 0.21] & 0.04 [-0.11,  0.19] & -0.06 [-0.22, 0.10]\\
\hspace{1em}$R^2_{marginal}$ / $R^2_{conditional}$ & 0.078 / NA &  & 0.046 / NA &  & 0.017 / 0.760 &  & 0.018 / 0.696 & \\
\addlinespace[0.3em]
\multicolumn{9}{l}{\textbf{Specific Psychological Needs}}\\
\hspace{1em}(Intercept) & 70.93** [41.14, 58.40] &  & 71.54**** [65.79, 77.28] &  & 70.75**** [67.59, 73.90] &  & 68.68**** [65.53, 71.82] & \\
\hspace{1em}CoreNeed & 1.88* [-5.45,  5.68] & 0.01 [-1.13, 1.15] & 0.10* [ 0.02,  0.18] & 0.27 [ 0.11, 0.43] & 0.06** [ 0.02,  0.10] & 0.14 [ 0.06, 0.22] & 0.14*** [ 0.07,  0.21] & 0.18 [ 0.09, 0.26]\\
\hspace{1em}Autonomy & 0.78 [-8.61,  8.70] & 0.01 [-1.15, 1.16] & 0.05 [-0.12,  0.23] & 0.12 [-0.02, 0.26] & 0.04 [-0.01,  0.09] & 0.05 [-0.03, 0.13] & 0.04 [ 0.00,  0.08] & 0.05 [-0.03, 0.12]\\
\hspace{1em}Competence & 0.95 [-3.70,  3.78] & 0.00 [-1.11, 1.11] & 0.05 [-0.16,  0.26] & 0.02 [-0.21, 0.25] & 0.05* [ 0.01,  0.10] & 0.06 [-0.03, 0.15] & 0.06* [ 0.01,  0.10] & 0.09 [ 0.02, 0.16]\\
\hspace{1em}Relatednesss & 2.10 [-4.04,  4.19] & 0.01 [-1.20, 1.22] & 0.10*** [ 0.06,  0.14] & 0.29 [ 0.17, 0.41] & 0.06**** [ 0.03,  0.09] & 0.16 [ 0.09, 0.24] & 0.06*** [ 0.02,  0.09] & 0.10 [ 0.02, 0.19]\\
\hspace{1em}$R^2_{marginal}$ / $R^2_{conditional}$ & 0.176 / NA &  & 0.061 / 0.860 &  & 0.025 / 0.747 &  & 0.041 / 0.702 & \\
\bottomrule
\multicolumn{9}{l}{\rule{0pt}{1em}\textit{Note: }}\\
\multicolumn{9}{l}{\rule{0pt}{1em}**** $p$ < .0001, *** $p$ < .001, ** $p$ < .01, * $p$ < .05}\\
\end{tabular}}
\end{minipage}
\end{table}

