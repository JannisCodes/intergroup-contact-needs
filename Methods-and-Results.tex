\faQuestionCircle~Unsure about order of study one and two. Currently
chronological. But could also be: (1) main model works in larger student
sample, (2) works in economic migrant sample, (3) test full thing in
medical sample.

\section{Study 1}

Based on our main hypotheses, the aim of our first study is to
specifically test the general contact hypothesis, the influence of core
need fulfillment, and perceived interaction quality during intergroup
contacts. To this aim, we conducted an intensive longitudinal survey
study with recent migrants to the Netherlands, gathering a large body of
ecologically valid data on need satisfaction in real-life intergroup
contact situations. Data was collected from May 5th through June 6th,
2018 (and all participants started the study within the first two days).

The full surveys are available in our Online Supplementary Material A
and the full data description is available in Online Supplementary
Materials B. Correlations and descriptive statistics of the included
variables are available in Table \ref{tab:workerVarDescr} and Table
\ref{tab:workerOutVarDescr}.

\subsection{Methods}

\subsubsection{Participants}

After receiving ethical approval from the University of Groningen, we
recruited 23 migrants using the local paid participant pool and
specifically targeted non-Dutch migrants to participate in our study.
Participants reported on their interactions for at least 30 days with
two daily measures (capturing the morning and afternoon). With this
design, we aimed at getting 50-60 measurements per participant
(\textit{M} = 53.26, \textit{M} = 16.72, \textit{total N} = 1225). This
is a common number of measurements found in experience sampling studies
and should offer sufficient power to model processes within and between
participants \citep[e.g., for a systematic review see][]{AanhetRot2012}.
Participants were compensated for their participation with up to 34
Euros -- each two Euros for pre- and post-questionnaire as well as 50
Eurocents for every experience sampling measurement occasion. The sample
consisted of relatively young, educated, and western migrants from the
global north (\(M_{age}\) = 24.35, \(SD_{age}\) = 4.73, 19 women, 15
students). The sample accurately describes one of the largest groups of
migrants in the region \citep[][]{GemeenteGroningen2015}.

\subsubsection{Procedure}

The study itself consisted of three main parts, an introductory
pre-measurement, and the daily experience sampling measurements, as well
as a concluding post-measurement. After giving informed consent,
participants started by filling in an online pre-questionnaire assessing
demographics and general information about their immigration. Over the
next thirty days, the participants then were invited twice a day (at 12
pm and 7pm) to reflect upon their interactions, psychological need
fulfillments, and current attitudes towards the Dutch outgroup
(\textit{median duration} = 22, \textit{MAD duration} = 19). After the
final day of daily diary measurements, participants were invited to fill
in a longer post measurement survey that mirrored the pre-measurement.
All key variables in for this study were part of the short daily diary
surveys.

\subsubsection{Materials}

\paragraph{Intergroup Contact}

To test the prerequisite effect of intergroup contact, every experience
sampling measurement started with the question
``\textit{Did you meet a Dutch person this morning [/afternoon]? (In person interaction for at least 10 minutes)}''.
Our participants recorded between 2--51 (3.23--91.07\% of individual
daily diary measurements; 31.59\% of all 1225 daily diary
responses)\footnote{Two participants only recorded two daily diary measurements each and non of these included outgroup contacts. These participants are removed from any analyses including outgroup contacts.}.

\paragraph{Psychological Needs}

Irrespective of whether participants had an interaction with Dutch
people or not, everyone answered a short series of questions on
psychological need fulfillment. However, whereas participants with
interactions reported on the need fulfillment during the interaction,
people without interactions with Dutch people judged the past daytime
period in general. To assess the fulfillment of psychological needs, we
included two types of need measurement: (1) the core situational need
and (2) general self-determination theory needs.

For the core situational need, we asked participants in an open ended
text field:
``\textit{What was your most important goal [during the interaction / this morning / this afternoon]?}''.
Then, with reference to the text entry, we asked how much this core need
was fulfilled during the interaction or the past daytime period:
``\textit{[The interaction / you] fulfilled your goal: [-previous text entry-]}''
on a continuous slider scale ranging from strongly disagree (-50) to
strongly agree (+50).

We, additionally, included a common measure of three self-determination
theory needs \citep[see][]{Downie2008}. The items were introduced either
by ``\textit{During the interaction:}'' or
``\textit{This morning [/afternoon]:}'' and measured autonomy
(``\textit{I was myself.}''), competence
(``\textit{I felt competent.}''), and relatedness (without intergroup
contact ``\textit{I had a strong need to belong}''; with intergroup
contact: ``\textit{I shared information about myself.}'' and
``\textit{The other(s) shared information about themselves.}''). All
items were rated on a continuous slider scale from very little (-50) to
a great deal (+50).

\paragraph{Perceived Interaction Quality}

As an explanatory mechanism, we assessed ratings of the perceived
interaction quality. As our main measurement, participants rated the
statement ``\textit{Overall the interaction was …}'' on two continuous
slider scales measuring pleasantness
\citep[from unpleasant (-50) to pleasant (+50)) and meaningfulness (from superficial (-50) to meaningful (+50); both items adapted from][]{Downie2008}.

\paragraph{Outgroup Attitudes}

At the end of every daily diary measurement we asked all participants
about their current attitudes towards the Dutch -- our main dependent
variable. To assess the momentary outgroup evaluation we used the common
feeling thermometer: ``How favorable do you feel towards the Dutch?''
\citep[][]{Lavrakas2008}. Participants then rated their attitude on a
continuous slider scale from ``very cold -- 0'' through ``no feeling --
50'' to ``very warm -- 100''. Both the question phrasing as well as the
tick labels were consistent with large-scale panel surveys
\citep[e.g.,][]{DeBell2010}.

\begin{table}
\begin{minipage}[t][\textheight][t]{\textwidth}

\caption{\label{tab:workerVarDescr}Worker: Multilevel Core Variable Descriptives}
\centering
\resizebox{\linewidth}{!}{
\begin{tabular}[t]{llccccc}
\toprule
  & Core Need & Competence & Autonomy & Relatedness & Quality & Attitudes NL\\
\midrule
Core Need &  & 0.82*** & 0.60*** & 0.33 & 0.52** & -0.03\\
Competence & 0.36*** &  & 0.89*** & 0.26 & 0.39 & -0.23\\
Autonomy & 0.28*** & 0.22*** &  & 0.31 & 0.57** & 0.02\\
Relatedness & 0.50*** & 0.39*** & 0.37*** &  & -0.07 & 0.14\\
Quality & 0.17*** & 0.44*** & 0.27*** & 0.26*** &  & 0.50*\\
Attitudes NL & 0.24*** & 0.36*** & 0.24*** & 0.37*** & 0.52*** & \\
\addlinespace
Grand Mean & 27.95 & 12.10 & 22.17 & 5.29 & 24.10 & 71.49\\
Between SD & 14.68 & 13.72 & 12.09 & 14.59 & 9.50 & 12.91\\
Within SD & NA & NA & NA & NA & NA & NA\\
ICC(1) & 0.29 & 0.28 & 0.38 & 0.28 & 0.18 & 0.70\\
ICC(2) & 0.96 & 0.95 & 0.97 & 0.95 & 0.79 & 0.99\\
Within.person.SD & 20.83 & 20.89 & 15.15 & 23.29 & 18.01 & 8.11\\
\bottomrule
\multicolumn{7}{l}{\rule{0pt}{1em}\textit{Note: }}\\
\multicolumn{7}{l}{\rule{0pt}{1em}Upper triangle: Between-person correlations;}\\
\multicolumn{7}{l}{\rule{0pt}{1em}Lower triangle: Within-person correlations;}\\
\multicolumn{7}{l}{\rule{0pt}{1em}*** p < .001, ** p < .01,  * p < .05}\\
\end{tabular}}
\end{minipage}
\end{table}

\begin{table}
\begin{minipage}[t][\textheight][t]{\textwidth}

\caption{\label{tab:workerOutVarDescr}Worker: Multilevel Core Variable Descriptives (Outgroup Contact Only)}
\centering
\resizebox{\linewidth}{!}{
\begin{tabular}[t]{llcc}
\toprule
  & Core Need & Quality & Attitudes NL\\
\midrule
Core Need &  & 0.40 & -0.03\\
Quality & 0.37*** &  & 0.21\\
Attitudes NL & 0.27*** & 0.55*** & \\
\addlinespace
Grand Mean & 82.20 & 67.00 & 72.46\\
Between SD & 12.42 & 9.26 & 13.62\\
Within SD & 17.66 & 18.24 & 9.50\\
\addlinespace
ICC(1) & 0.33 & 0.23 & 0.68\\
ICC(2) & 0.90 & 0.84 & 0.98\\
\bottomrule
\multicolumn{4}{l}{\rule{0pt}{1em}\textit{Note: }}\\
\multicolumn{4}{l}{\rule{0pt}{1em}Upper triangle: Between-person correlations;}\\
\multicolumn{4}{l}{\rule{0pt}{1em}Lower triangle: Within-person correlations;}\\
\multicolumn{4}{l}{\rule{0pt}{1em}*** p < .001, ** p < .01,  * p < .05}\\
\end{tabular}}
\end{minipage}
\end{table}


\subsection{Results}

\paragraph{Contact Hypothesis}

\paragraph{Core Need}

\begin{itemize}
\tightlist
\item
  Att \textasciitilde{} Need \& Qlt \textasciitilde{} Need
\item
  Att \textasciitilde{} Need + Qual
\end{itemize}

\paragraph{Robustness}

\section{Study 2}

\subsection{Methods}

\subsection{Results}

\section{Study 2}

\subsection{Methods}

\subsection{Results}
