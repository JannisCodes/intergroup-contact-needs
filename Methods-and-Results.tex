\section{Study 1}

Based on our main hypotheses, the aim of our first study was to
specifically test the general contact hypothesis, the influence of core
need fulfillment, and perceived interaction quality during intergroup
contacts. To this aim, we conducted an intensive longitudinal survey
study with recent migrants to the Netherlands. Data was collected from
May 5\textsuperscript{th} through June 6\textsuperscript{th}, 2018 (and
all participants started the study within the first two days).

The full surveys are available in our OSF repository
\citep{KreienkampMasked2022a} and the full data description is available
in Online Supplementary Material A. Correlations and descriptive
statistics of the included variables are available in Table
\ref{tab:descrFullWide} and Table \ref{tab:descrOutWide}.

\subsection{Methods}

\subsubsection{Participants}

After receiving ethical approval from the University Masked for Peer
Review, we recruited 23 migrants using the local paid participant pool
and specifically targeted non-Dutch migrants to participate in our
study. Participants reported on their interactions for at least 30 days
with two daily measures (capturing the morning and afternoon). With this
design, we aimed at getting 50-60 measurements per participant
(\textit{M} = 53.26, \textit{SD} = 16.72, \textit{total N} = 1,225).
This is a common number of measurements found in experience sampling
studies and offers sufficient power to model processes within and
between participants
\citep[e.g., for a systematic review see][]{AanhetRot2012}. Participants
were compensated for their participation with up to 34 Euros -- each two
Euros for pre- and post-questionnaire as well as 50 Eurocents for every
experience sampling measurement occasion. The sample consisted of
relatively young, educated, and western migrants from the global north
(\(M_{age}\) = 24.35, \(SD_{age}\) = 4.73, 19 women, 15 students). The
sample accurately describes the largest groups of migrants in the region
\citep[][]{GemeenteGroningen2015}.

\subsubsection{Procedure}

The study itself consisted of three main parts, an introductory
pre-measurement, and the daily experience sampling measurements, as well
as a concluding post-measurement. After giving informed consent,
participants started by filling in an online pre-questionnaire assessing
demographics and general information about their immigration. Over the
next thirty days, the participants were then invited twice a day (at 12
pm and 7pm) to reflect upon their interactions, psychological need
fulfillments, and current attitudes towards the Dutch outgroup
(\textit{median duration} = 2M 22S, \textit{MAD duration} = 1M 19S).
General compliance was high (85.90\% of all invited surveys were filled
in)\footnote{Two participants completed only two days (among the others, participation was 93.70\%).}.
The response rates were approximately equal during mornings (\textit{n}
= 621) and afternoons (\textit{n} = 604) and most measurements were
completed within four hours of the invitation. After the final day of
experience sampling measurements, participants were invited to fill in a
longer post measurement survey that mirrored the pre-measurement. All
key variables for this study were part of the short experience sampling
surveys.

\subsubsection{Materials}

\paragraph{Intergroup Contact}

To test the prerequisite effect of intergroup contact, every experience
sampling measurement started with the question
``\textit{Did you meet a Dutch person this morning [/afternoon]? (In person interaction for at least 10 minutes)}''.
Our participants recorded between 2--51 interactions with Dutch outgroup
members (3.23--91.07\% of individual experience sampling measurements;
31.59\% of all 1,225 experience sampling
responses)\footnote{Two participants only recorded two experience sampling measurements each and non of these included outgroup contacts. These participants are removed from any analyses that focus on outgroup contacts.}.

\paragraph{Psychological Needs}

Irrespective of whether participants had an interaction with Dutch
people or not, everyone answered a short series of questions on
psychological need fulfillment. However, whereas participants with
interactions reported on the need fulfillment during the interaction,
people without interactions with Dutch people judged the past daytime
period in general. To assess the fulfillment of psychological needs, we
included two types of need measurement: (1) the core situational need
and (2) general self-determination theory needs.

For the core situational need, we asked participants in an open ended
text field:
``\textit{What was your most important goal [during the interaction / this morning / this afternoon]?}''.
Then, with reference to the text entry, we asked how much this core need
was fulfilled during the interaction or the past daytime period:
``\textit{[The interaction / You] fulfilled your goal: [-previous text entry-]}''
on a continuous slider scale ranging from strongly disagree (-50) to
strongly agree (+50).

We, additionally, included a common measure of three self-determination
theory needs \citep[see][]{Downie2008}. The items were introduced either
by ``\textit{During the interaction:}'' or
``\textit{This morning [/afternoon]:}'' and measured autonomy
(``\textit{I was myself.}''), competence
(``\textit{I felt competent.}''), and relatedness (without intergroup
contact ``\textit{I had a strong need to belong}''; with intergroup
contact: ``\textit{I shared information about myself.}'' and
``\textit{The other(s) shared information about themselves.}''). All
items were rated on a continuous slider scale from very little (-50) to
a great deal (+50).

\paragraph{Perceived Interaction Quality}

As an explanatory mechanism, we assessed ratings of the perceived
interaction quality. As our main measurement, participants rated the
statement ``\textit{Overall the interaction was …}'' on two continuous
slider scales measuring pleasantness
\citep[from unpleasant (-50) to pleasant (+50)) and meaningfulness (from superficial (-50) to meaningful (+50); both items adapted from][]{Downie2008}.

\paragraph{Outgroup Attitudes}

At the end of every experience sampling measurement we asked all
participants about their current attitudes towards the Dutch -- our main
dependent variable. To assess the momentary outgroup evaluation we used
the common feeling thermometer: ``How favorable do you feel towards the
Dutch?'' \citep[][]{Lavrakas2008}. Participants then rated their
attitude on a continuous slider scale from ``very cold -- 0'' through
``no feeling -- 50'' to ``very warm -- 100''. Both the question phrasing
as well as the tick labels were consistent with large-scale panel
surveys \citep[e.g.,][]{DeBell2010}.

\subsection{Results}

\subsubsection{Contact Hypothesis}

We tested the most general contact hypothesis in two steps. First, we
assessed whether overall between participants more intergroup
interactions were related to to more positive average outgroup
attitudes. Second, we tested whether a potential positive effect on
outgroup attitudes depended on the interaction quality (jointly with the
number of interactions). We find that neither the overall number of
contacts nor the number of experience sampling responses with a contact
were significantly related with the average outgroup attitudes (all
\textbar{} \textit{r} \textbar{} \textless{} 0.03, all \textit{p}
\textgreater{} 0.887). However, we did find a medium sized correlation
between the participants' average contact quality and their average
outgroup attitudes (\textit{r} = 0.55, \textit{p} = 0.010). And when
considering the number of contacts and average contact quality jointly
in a linear regression, we additionally found a statistically
significant interaction term (\textit{b} = -0.05, \textit{t}(17) =
-2.95, \textit{p} = 0.009, \(\eta_p^2\) = 0.34), where in our sample
with an increasing number of contacts the positive effect of average
contact quality becomes weaker (also see Table
\ref{tab:intergroupGeneralTblLong}). We additionally used a multilevel
regression to check whether having an interaction with an outgroup
member had a situational (i.e., contemporaneous) effect within the
participants. We find that having an outgroup contact is indeed
associated with significantly more positive outgroup attitudes within
the participants (\textit{b} = 2.48, \textit{t}(1,200) = 4.36,
\textit{p} \textless{} .001, \textit{95\%CI}{[}1.36, 3.59{]}), even
after controlling for having an interaction with a non-Dutch (which did
not relate to outgroup attitudes independently; For full results see
Table \ref{tab:intergroupGeneralTblLong}, Figure
\ref{fig:ContactHypothesis}, and Online Supplementary Material
A)\footnote{Interestingly, adding random slopes to this model did not explain additional variance. This is unusual and might indicate that the effect is very consistent across participants. However, the small number of participants, or other measurement issues provide an alternative explanation.}.
Thus, in our first data we find mixed results, where outgroup contacts
show a positive effect on outgroup attitudes in the moment but these
results do not translate into average differences between participants
(even tough the average number of contacts does seem to interact with
average contact quality to predict outgroup attitudes). However, it
should be noted that the between-participant analyses in the aggregated
form are based on a small sample size.

\subsubsection{Core Need}

The main proposal of our article is that the success of an outgroup
contact might be explained by whether or not the contact fulfilled the
person's core situational need. This should, in turn, be due to a higher
perceived contact quality. We sequentially test whether the fulfillment
of the core need during an interaction is (1) related to more positive
outgroup attitudes, (2) higher perceived contact quality, and (3)
whether the variance explained by the core need is subsumed by the
perceived contact quality if considered jointly (for full regression
results see Table \ref{tab:intergroupNeedsTblLong} and Figure
\ref{fig:AllportNeedFulfillment}).

We find that in the multilevel models, the fulfillment of core
situational needs during outgroup contacts was associated with more
positive outgroup attitudes (random slopes model; \textit{b} = 0.17,
\textit{t}(365) = 2.93, \textit{p} = 0.004, \textit{95\%CI}{[}0.06,
0.29{]}) and also predicted higher perceived contact quality (random
intercept model; \textit{b} = 0.42, \textit{t}(365) = 8.07, \textit{p}
\textless{} .001, \textit{95\%CI}{[}0.32, 0.52{]}). Moreover, when we
consider the influences of core need fulfillment and contact quality on
outgroup attitudes jointly, we find that virtually all variance is
explained by perceived contact quality (random slopes mode; \textit{b} =
0.23, \textit{t}(364) = 4.22, \textit{p} \textless{} .001,
\textit{95\%CI}{[}0.12, 0.33{]}) and only little unique variance is
still explained by core need fulfillment (\textit{b} = 0.04,
\textit{t}(364) = 0.64, \textit{p} = 0.523, \textit{95\%CI}{[}-0.08,
0.15{]}). We thus find support for our hypotheses and can conclude that
in this data set the fulfillment of core situational needs had a
significant influence on outgroup attitudes. Additionally, this effect
is likely explained by its effect through perceived contact quality.

\subsubsection{Robustness}

To build further confidence in our results, we assessed two additional
models that might offer alternative explanations. First, to ensure that
the effect of core need fulfillment is specific to an actual contact, we
compared the effect to core need fulfillment in situations without an
intergroup contact. For this, we analyzed the generalized situational
core need fulfillment (either during a contact or about the daytime in
general) and tested whether the effect differed during experience
sampling measurements with and without outgroup contacts. We found no
main effect of core need fulfillment (random slopes model; \textit{b} =
-0.10, \textit{t}(1,199) = -2.22, \textit{p} = 0.026,
\textit{95\%CI}{[}-0.18, -0.01{]}) but a significant interaction effect
of core need fulfillment and outgroup contact (\textit{b} = 0.13,
\textit{t}(1,199) = 4.31, \textit{p} \textless{} .001,
\textit{95\%CI}{[}0.07, 0.18{]}; also see Table
\ref{tab:robustnessTblLong} and Figure \ref{fig:Robustness}). Together
with a significant main effect of having an outgroup contact, this
indicates that it is not key need fulfillment in general --- but only
key need fulfillment during an outgroup contact that predicts more
positive outgroup attitudes.

In a final step, we controlled for other fundamental psychological needs
during the contact. We focus on the three commonly considered
self-determination needs (SDT): competence, autonomy, and relatedness.
We find that the core need fulfillment adds significantly above a model
with only the self-determination theory needs (random slopes models;
\(\chi^2\)(6, \textit{N} = 21) = 19.60, \textit{p} = 0.003). We also
find that next to relatedness (\textit{b} = 0.10, \textit{t}(361) =
4.04, \textit{p} \textless{} .001, \textit{95\%CI}{[}0.05, 0.14{]}), the
core need explains the most variance in outgroup attitudes after an
outgroup contact (\textit{b} = 0.09, \textit{t}(361) = 2.23, \textit{p}
= 0.026, \textit{95\%CI}{[}0.01, 0.18{]}). When compared to the model
with only the SDT needs, the core need fulfillment flexibly takes on
some of the explained variance of all three fundamental needs
(competence and autonomy needs turning non-significant; all \textit{b}
\textless{} 0.05, all \textit{p} \textgreater{} 0.544). For full results
see see Table \ref{tab:robustnessTblLong}, Figure \ref{fig:Robustness},
and Online Supplementary Material A. There is, thus, considerable
evidence lending confidence to the stability and relevance of
psychological need fulfillment as a predictor of positive outgroup
attitudes for natural intergroup contacts.

\section{Study 2}

The aim of Study 2 is similar to Study 1, as we again test the general
contact hypothesis, the influence of core need fulfillment, and
perceived contact quality during intergroup contacts. However, in this
second study we collected a substantially larger sample of international
students who recently arrived in the Netherlands and also improved the
study design (e.g., pop-up explanations described later). The survey
method again offers a large body of ecologically valid data on need
satisfaction in real-life intergroup contact situations as these
students will likely interact with the Dutch majority outgroup on a
daily basis. Data was collected from November 19\textsuperscript{th},
2018 through January 6\textsuperscript{th}, 2019.

The full surveys are available in our OSF repository
\citep{KreienkampMasked2022a} and the full data description is available
in Online Supplementary Material A. Correlations and descriptive
statistics of the included variables are available in Table
\ref{tab:descrFullWide} and Table \ref{tab:descrOutWide}.

\subsection{Methods}

\subsubsection{Participants}

We recruited 113 international students using a local participant pool.
We specifically targeted non-Dutch students, who had recently arrived in
the Netherlands. Participants reported on their interactions for at
least 30 days with two daily measures (capturing the morning and
afternoon). With this design, we again aimed at receiving 50-60
measurements per participant (\textit{M} = 43.94, \textit{SD} = 15.00,
\textit{total N} = 4,965). As with the previous study this should offer
sufficient power to model processes within participants and will lend
stronger weight to between-participant results. Participants were
compensated for their participation with partial course credits ---
depending on their participation. The sample consisted of relatively
young migrants, who were mostly from the global north (\(M_{age}\) =
20.24, \(SD_{age}\) = 2.12, 84 women). The sample fairly accurately
describes the local population of international students.

\subsubsection{Procedure}

The study procedure mirrored the setup of Study 1 and consisted of pre-,
experience sampling-, and post-measurements. The participants were
invited for experience sampling measurements twice a day (at 12 pm and
7pm) for 30 days (\textit{median duration} = 3M 46S,
\textit{MAD duration} = 1M 54S). General compliance was high (70.87\% of
all invited surveys were filled in). The response rates were
approximately equal during mornings (\textit{n} = 2,608) and afternoons
(\textit{n} = 2,357). All key variables for this study were part of the
short experience sampling surveys.

\subsubsection{Materials}

\paragraph{Intergroup Contact}

To measure intergroup contacts, every experience sampling measurement
started with the question
``\textit{Did you meet a Dutch person this morning [/afternoon]? (in-person interaction for at least 10 minutes)}''.
Participants were additionally offered a pop-up explanation: ``With
in-person interaction, we mean a continued interaction with another
person (potentially in a group) that lasted at least 10 minutes. This
interaction should be offline and face-to-face. It should include some
form of verbal communication and should be uninterrupted to still count
as the same interaction. Any individual interaction can last minutes or
hours. If there were multiple interaction partners, we would like you to
focus on the person that was most important to you during the
interaction.''. The participants recorded between 1--43 interactions
with Dutch majority people (1.64--93.75\% of individual experience
sampling measurements; 18.83\% of all 4,965 experience sampling
responses).

\paragraph{Psychological Needs}

For the core situational need, we asked participants in an open ended
text field:
``\textit{What was your main goal [during the interaction with -X- / this morning / this afternoon]?}''
(where \textit{-X-} was dynamically replaced with the name of the
interaction partner). Participants could additionally click on a pop-up
explanation: ``Your main goal during an interaction can vary depending
on the interaction. It could be to connect with friends, to find or
provide help, to achieve academic ambitions, work on your fitness, work
for a job, or simply to get a coffee, just as well as many other
concrete or abstract goals that are import to you in the moment. It
really depends on your subjective experience of the interaction.''.
Then, with reference to the text entry, we asked how much this core need
was fulfilled during the interaction or the past daytime period:
``\textit{During your interaction with -X- [this morning / this evening] your goal (-previous text entry-) was fulfilled.}''
on a continuous slider scale ranging from strongly disagree (1) to
strongly agree (100).

The measurement of the self-determination needs was identical to Study
1.

\paragraph{Perceived Interaction Quality}

The ratings of the perceived contact quality was identical to Study 1.

\paragraph{Outgroup Attitudes}

As in Study 1, attitudes towards the Dutch majority outgroup were again
measured using the feeling thermometer. All survey details are also
available in our OSF repository \citep{KreienkampMasked2022a} and Online
Supplemental Materials A.

\subsection{Results}

\subsubsection{Contact Hypothesis}

We tested the most general contact hypothesis as in Study 1. We found a
significant correlation with average outgroup attitudes for both the
total number of outgroup interactions (\textit{r} = 0.22, \textit{p} =
0.019) and the number of measurement beeps with an interaction
(\textit{r} = 0.28, \textit{p} = 0.003). However, we found no
significant correlation between the participants' average interaction
quality and their average outgroup attitudes (\textit{r} = 0.04,
\textit{p} = 0.679). Thus, when considering the number of interactions
and average interaction quality jointly in a linear regression, we find
a main effect of intergroup contacts to predict outgroup attitudes
(\textit{b} = 0.81, \textit{t}(109) = 3.29, \textit{p} = 0.001,
\(\eta_p^2\) = 0.08) but no significant effect of average perceived
contact quality and no significant interaction term (for full results
see Table \ref{tab:intergroupGeneralTblLong} and Figure
\ref{fig:ContactHypothesis}). We additionally used a multilevel
regression to check whether having an interaction with an outgroup
member had a situational (i.e., contemporaneous) effect within the
participants. We find that having an outgroup interaction is indeed
associated with significantly more positive outgroup attitudes within
the participants (random slopes model; \textit{b} = 2.99,
\textit{t}(4,850) = 3.80, \textit{p} \textless{} .001,
\textit{95\%CI}{[}1.45, 4.54{]}), even after controlling for having an
interaction with a non-Dutch person (which did not relate to outgroup
attitudes independently; For full results see Table
\ref{tab:intergroupGeneralTblLong}, Figure \ref{fig:ContactHypothesis},
as well as Online Supplementary Material A). Thus, in our second data
set we find mixed results; outgroup contacts show a positive effect on
outgroup attitudes in the moment and on average between participants.
However, we find that this effect does not depend on the average
perceived interaction quality. This unexpected results, might be due to
the aggregation process and the following analyses will focus in on the
role of the perceived interaction quality during an outgroup
interaction.

\subsubsection{Core Need}

We again sequentially tested whether the fulfillment of the core need
during an interaction was (1) related to more positive outgroup
attitudes, (2) higher perceived interaction quality, and (3) whether the
variance explained by the core need is subsumed by the perceived
interaction quality if considered jointly. We find that in the
multilevel models, the fulfillment of core situational needs during
outgroup contacts was associated with more positive outgroup attitudes
(random slopes model; \textit{b} = 0.13, \textit{t}(826) = 4.18,
\textit{p} \textless{} .001, \textit{95\%CI}{[}0.07, 0.19{]}) and also
predicted higher perceived interaction quality (random slopes model;
\textit{b} = 0.40, \textit{t}(826) = 6.46, \textit{p} \textless{} .001,
\textit{95\%CI}{[}0.28, 0.52{]}). Additionally, if we consider the
influences of core need fulfillment and interaction quality on outgroup
attitudes jointly, we find that virtually all variance is explained by
perceived interaction quality (random slopes mode; \textit{b} = 0.17,
\textit{t}(825) = 5.56, \textit{p} \textless{} .001,
\textit{95\%CI}{[}0.11, 0.23{]}) and only little unique variance is
still explained by core need fulfillment (\textit{b} = 0.03,
\textit{t}(825) = 1.47, \textit{p} = 0.141, \textit{95\%CI}{[}-0.01,
0.07{]}; also see Table \ref{tab:intergroupNeedsTblLong} and Figure
\ref{fig:AllportNeedFulfillment} for full results). These results are
consistent with the results in Study 1. We, thus, find support for our
hypotheses that the fulfillment of core situational needs had a
significant influence on outgroup attitudes and that this effect is
likely explained by its effect through perceived interaction quality.

\subsubsection{Robustness}

We again checked for alternative models. First, when considering
generalized situational core need fulfillment together with whether an
intergroup contact took place, we find that there is only a minuscule
main effect of core need fulfillment (random slopes model; \textit{b} =
-0.03, \textit{t}(4,849) = -1.34, \textit{p} = 0.181,
\textit{95\%CI}{[}-0.08, 0.02{]}) but a stronger interaction effect of
core need fulfillment and outgroup contact (\textit{b} = 0.06,
\textit{t}(4,849) = 3.03, \textit{p} = 0.002, \textit{95\%CI}{[}0.02,
0.10{]}). Together with a significant main effect of having an outgroup
contact (\textit{b} = 2.88, \textit{t}(4,849) = 3.71, \textit{p}
\textless{} .001, \textit{95\%CI}{[}1.36, 4.40{]}), this indicates that
it is not key need fulfillment in general --- but key need fulfillment
during an outgroup contact that predicts more positive outgroup
attitudes. This finding is consistent with the results of the previous
study, albeit with a slightly weaker effect (likely because of the large
number of measurements that did not include an outgroup interaction; For
full results see Table \ref{tab:intergroupNeedsTblLong} and for visual
comparison see Figure \ref{fig:AllportNeedFulfillment}).

In a final step, we again checked whether during the interaction the
core situational need remains a meaningful predictor even when taking
other fundamental psychological needs into account. We find that the
core need fulfillment adds significantly above a model with only the
self-determination theory needs (random slopes models; \(\chi^2\)(6,
\textit{N} = 108) = 22.90, \textit{p} \textless{} .001). We find that
the core need fulfillment explained the most variance in outgroup
attitudes after an outgroup contact (\textit{b} = 0.08, \textit{t}(823)
= 2.95, \textit{p} = 0.003, \textit{95\%CI}{[}0.03, 0.13{]}). When
compared to the model with only the SDT needs, the core need fulfillment
flexibly takes on some of the explained variance of all of the three
fundamental needs. However, different from the first study, relatedness
(\textit{b} = 0.07, \textit{t}(823) = 4.26, \textit{p} \textless{} .001,
\textit{95\%CI}{[}0.04, 0.10{]}) and autonomy (\textit{b} = 0.02,
\textit{t}(823) = 0.93, \textit{p} = 0.353, \textit{95\%CI}{[}-0.02,
0.08{]}) also predicted positive outgroup attitudes in this larger
sample. For full results see Table \ref{tab:intergroupNeedsTblLong},
Figure \ref{fig:Robustness}, and Online Supplementary Material A. This
means that also within this second sample, the fulfillment of
psychological needs during intergroup contact remained a key predictor
of positive outgroup attitudes, even when taking into account several
alternative models.

\section{Study 3}

The aim of this final study is to extend the previous studies by
additionally testing Allport's conditions in an extensive longitudinal
design and to compare the predictive powers of Allport's conditions and
the core situational need fulfillment. We will, thus, again test the
general contact hypothesis, the influence of core need fulfillment, and
perceived interaction quality during intergroup contacts. However, we
additionally have the opportunity to also assess the role of Allport's
conditions in this and to compare the two approaches (Allport's
conditions and core need fulfillment) directly.

For this study, we specifically recruited international medical
students, because they represent a particular group of migrants who face
structural requirements to integrate and interact with Dutch majority
outgroup members on a daily basis. As part of their educational program,
the migrants are required to take language courses and interact with
patients as part of their medical internships and medical residency. The
extensive longitudinal survey method again offers a large body of
ecologically valid data on need satisfaction in real-life intergroup
contact situations. Data were collected from November
8\textsuperscript{th}, 2019 to January 10\textsuperscript{th}, 2020.

The full surveys are available in our OSF repository
\citep{KreienkampMasked2022a} and the full data description is available
in Online Supplementary Material A. Correlations and descriptive
statistics of the included variables are available in Tables
\ref{tab:descrFullWide} and \ref{tab:descrOutWide}.

\subsection{Methods}

\subsubsection{Participants}

We recruited 71 international medical students using contacts within the
University Medical School. We specifically targeted non-Dutch students,
who had recently arrived in the Netherlands. Participants reported on
their interactions for at least 30 days with two daily measures
(capturing the morning and afternoon). With this design, we aimed at
getting 50-60 measurements per participant (\textit{M} = 57.85,
\textit{SD} = 20.68, \textit{total N} = 4,107). As with the previous
studies, this should offer sufficient power to model processes within
participants and will lend stronger weight to between-participant
results. Participants were compensated for their participation in the
same manner as during Study 1. The sample consisted of relatively young
migrants, who were mostly from the global north (\(M_{age}\) = 22.68,
\(SD_{age}\) = 3.10, 59 women). The sample fairly accurately describes
the local population of young international medical professionals.

\subsubsection{Procedure}

The study procedure mirrored the setup of studies one and two, in that
it consisted of pre-, experience sampling-, and post-measurement. The
participants were invited for experience sampling measurements twice a
day (at 12 pm and 7pm) for 30 days (\textit{median duration} = 4M 13S,
\textit{MAD duration} = 2M 6S). General compliance was high (85.92\%
filled in at least 31 experience sampling surveys or more). The response
rates were approximately equal during mornings (\textit{n} = 2,092) and
afternoons (\textit{n} = 2,015). All key variables for this study were
part of the short experience sampling surveys.

\subsubsection{Materials}

\paragraph{Intergroup Contact}

The measurement of intergroup contacts was identical to Study 2. The
participants recorded between 1--71 interactions with Dutch outgroup
members (2.13--87.18\% of individual experience sampling measurements;
41.44\% of all 4,107 experience sampling responses).

\paragraph{Psychological Needs}

The measurement of the core situational need and its fulfillment was
identical to Study 2. Similarly, the measurement of the
self-determination needs was identical to Studies 1 and 2

\paragraph{Allport's Conditions}

We measured how much each of the interactions fulfilled Allport's
conditions of optimal contact using a common short scale comprised of
four attributes \citep{Islam1993, Voci2003, AlRamiah2012a, Dixon2005}.
In particular, we asked participants to rate how much the interaction
had equal status
(``\textit{The interaction with [name interaction partner] was on equal footing (same status)}''),
a common goal
(``\textit{[name interaction partner] shared your goal ([free-text entry interaction key need])}''),
support of authorities
(``\textit{The interaction with [name interaction partner] was voluntary}''),
and intergroup cooperation
(``\textit{The interaction with [name interaction partner] was cooperative}'').
We create a mean-averaged index of Allport's conditions in response to
past findings indicating that the conditions are best conceptualized
jointly and as functioning together rather than as fully independent
factors \citep[][, p. 766]{Pettigrew2006}. For full psychometric
information see Online Supplementary Material A.

\paragraph{Perceived Interaction Quality}

The ratings of the perceived interaction quality was identical to Study
1.

\paragraph{Outgroup Attitudes}

Attitudes towards the Dutch majority outgroup was again measured using
the feeling thermometer, as in studies one and two. All survey details
are also available in our OSF repository \citep{KreienkampMasked2022a}
and Online Supplemental Materials A.

\subsection{Results}

\subsubsection{Contact Hypothesis}

As in Studies 1 and 2, we first conducted a general contact hypothesis
test. We found no correlation between the overall number of intergroup
contacts and average outgroup attitudes --- for neither the total number
of outgroup interactions nor the number of measurement beeps with an
interaction (\textbar{}\textit{r}\textbar{} \textless{} 0.13, \textit{p}
\textgreater{} 0.291). However, we did find a significant correlation
between the participants' average interaction quality and their average
outgroup attitudes (\textit{r} = 0.25, \textit{p} = 0.036). When
considering the number of interactions and average interaction quality
jointly in a linear regression, we only found a main effect of average
perceived interaction quality on outgroup attitudes (\textit{b} = 0.36,
\textit{t}(67) = 2.22, \textit{p} = 0.030, \(\eta_p^2\) = 0.07) but no
significant effect of the number of contacts nor an interaction effect.
This result mirrors that of Study 1 but is inconsistent with the second
study. Additionally, in a multilevel regression to check for
contemporaneous within person effects, we found that having an outgroup
interaction was indeed associated with significantly more positive
outgroup attitudes within the participants (random slopes model;
\textit{b} = 5.63, \textit{t}(3,834) = 6.61, \textit{p} \textless{}
.001, \textit{95\%CI}{[}3.97, 7.29{]}), even after controlling for
having an interaction with a non-Dutch (which did not relate to outgroup
attitudes independently; for full results see Table
\ref{tab:intergroupGeneralTblLong}, Figure \ref{fig:ContactHypothesis},
and Online Supplementary Material A). Thus, in our third data set we
again found mixed results. Aggregate outgroup contacts were generally
not significantly related to average outgroup attitudes on a between
participant level and this effect was also not suppressed by an average
interaction quality (i.e., no interaction effect). However, while the
aggregate data was partly inconsistent with the first two studies, the
within person contemporaneous effect of intergroup contact was
consistent across all three studies.

\subsubsection{Allport's Conditions}

As we did with core need fulfillment in previous studies, we
sequentially tested whether the fulfillment of Allport's contact
conditions was (1) related to more positive outgroup attitudes, (2)
higher perceived interaction quality, and (3) whether the variance
explained by Allport's Conditions is assumed by the perceived
interaction quality if considered jointly. In the multilevel models we
find that the fulfillment of Allport's Conditions during outgroup
contacts was associated with more positive outgroup attitudes (random
slopes model; \textit{b} = 0.22, \textit{t}(1,601) = 5.86, \textit{p}
\textless{} .001, \textit{95\%CI}{[}0.15, 0.29{]}) and also predicted
higher perceived interaction quality (random slopes model; \textit{b} =
0.65, \textit{t}(1,605) = 11.82, \textit{p} \textless{} .001,
\textit{95\%CI}{[}0.54, 0.76{]}). Moreover, when we consider the
influences of Allport's Conditions and interaction quality on outgroup
attitudes jointly, we find that perceived interaction quality is a
substantially stronger predictor (random slopes mode; \textit{b} = 0.17,
\textit{t}(1,600) = 6.25, \textit{p} \textless{} .001,
\textit{95\%CI}{[}0.12, 0.23{]}) and the unique variance explained by
Allport's Conditions was less than half of its original effect size
(\textit{b} = 0.09, \textit{t}(1,600) = 2.89, \textit{p} = 0.004,
\textit{95\%CI}{[}0.03, 0.15{]}; also see Table
\ref{tab:intergroupNeedsTblLong} and Figure
\ref{fig:AllportNeedFulfillment}). These results indicate that in this
data set the fulfillment of Allport's conditions had a significant
influence on outgroup attitudes and this effect is likely, in parts,
explained by its effect through perceived interaction quality.

\subsubsection{Core Need}

Similarly, we again sequentially tested whether the fulfillment of the
core need during an interaction was (1) related to more positive
outgroup attitudes, (2) higher perceived interaction quality, and (3)
whether the variance explained by the core need is assumed by the
perceived interaction quality if considered jointly. We find that the
fulfillment of the core need during outgroup contacts was associated
with more positive outgroup attitudes (random slopes model; \textit{b} =
0.19, \textit{t}(1,601) = 5.29, \textit{p} \textless{} .001,
\textit{95\%CI}{[}0.12, 0.27{]}) and also predicted higher perceived
interaction quality (random slopes model; \textit{b} = 0.43,
\textit{t}(1,605) = 8.35, \textit{p} \textless{} .001,
\textit{95\%CI}{[}0.33, 0.54{]}). Additionally, once we consider the
influences of core need fulfillment and interaction quality on outgroup
attitudes jointly, we find that perceived interaction quality is a
substantially stronger predictor (random slopes mode; \textit{b} = 0.17,
\textit{t}(1,600) = 6.86, \textit{p} \textless{} .001,
\textit{95\%CI}{[}0.12, 0.22{]}) and the unique variance explained by
core need fulfillment was roughly half of its original effect size
(\textit{b} = 0.11, \textit{t}(1,600) = 3.50, \textit{p} \textless{}
.001, \textit{95\%CI}{[}0.05, 0.17{]}; for full results see Table
\ref{tab:intergroupNeedsTblLong} and Figure
\ref{fig:AllportNeedFulfillment}). These results indicate that in this
data set outgroup attitudes were significantly predicted by the
fulfillment of core situational needs and this effect partially emerged
via its effect on perceived interaction quality.

\subsubsection{Compare Fulfillment of Core Need and Allport's Conditions}

To test wether Allport's conditions or the core need fulfillment are
better at predicting outgroup attitudes, we first assess relative model
performance indices (i.e., Akaike information criterion, and Bayesian
information criterion), and then consider the two predictors in a joint
model to see whether the two approaches predict the same variance in
outgroup attitudes. When comparing the model selection indices we find
that the fulfillment of the situational core need, indeed performs
slightly better than the model using Allport's conditions
(\(AIC_{CoreNeed}\) 12632.02 \textless{} 12651.59 \(AIC_{Allport}\), and
\(BIC_{CoreNeed}\) 12664.55 \textless{} 12684.12 \(BIC_{Allport}\)).
Additionally, when considering the predictors jointly, we find that both
significantly predict outgroup attitudes with similar sized regression
parameters (random slopes model; Allport's Conditions: \textit{b} =
0.16, \textit{t}(1,600) = 4.92, \textit{p} \textless{} .001,
\textit{95\%CI}{[}0.09, 0.24{]}, Core Need: \textit{b} = 0.14,
\textit{t}(1,600) = 3.85, \textit{p} \textless{} .001,
\textit{95\%CI}{[}0.08, 0.17{]}; also see Table
\ref{tab:intergroupNeedsTblLong} and Figure
\ref{fig:AllportNeedFulfillment}). This indicates that, although both
Allport's conditions and the core need fulfillment seem to (in part)
work through perceived interaction quality, they explain different
aspects of the variance in outgroup attitudes and do not constitute one
another.

\subsubsection{Robustness}

We again checked for alternative models of the key need fulfillment.
First, when considering generalized situational core need fulfillment
together with whether an intergroup contact took place, we find that
there is no significant main effect of core need fulfillment (random
slopes model; \textit{b} = 0.03, \textit{t}(3,835) = 1.51, \textit{p} =
0.131, \textit{95\%CI}{[}-0.01, 0.06{]}) but a stronger interaction
effect of core need fulfillment and outgroup contact (\textit{b} = 0.17,
\textit{t}(3,835) = 7.32, \textit{p} \textless{} .001,
\textit{95\%CI}{[}0.12, 0.21{]}). Together with a significant main
effect of having an outgroup contact (\textit{b} = 5.41,
\textit{t}(3,835) = 6.36, \textit{p} \textless{} .001,
\textit{95\%CI}{[}3.74, 7.08{]}), this indicates that it is not key need
fulfillment in general --- but key need fulfillment during an outgroup
contact that predicts more positive outgroup attitudes. This finding is
consistent with the results of the previous studies.

In a final step, we again checked whether during the interaction the
core situational need remains a meaningful predictor even when taking
other fundamental psychological needs into account. We find that the
core need fulfillment adds additional variance above a model with only
the self-determination theory needs (random slopes models; \(\chi^2\)(6,
\textit{N} = 70) = 100.20, \textit{p} \textless{} .001). We find that
the core need explains the most variance in outgroup attitudes after an
outgroup contact (\textit{b} = 0.15, \textit{t}(1,598) = 3.88,
\textit{p} \textless{} .001, \textit{95\%CI}{[}0.07, 0.22{]}). When
compared to the model with only the SDT needs, the core need fulfillment
flexibly takes on some of the explained variance of all of the three
fundamental needs. However, similar to the previous study, in this large
sample relatedness (\textit{b} = 0.05, \textit{t}(1,598) = 3.31,
\textit{p} \textless{} .001, \textit{95\%CI}{[}0.02, 0.09{]}),
competence (\textit{b} = 0.06, \textit{t}(1,598) = 2.73, \textit{p} =
0.006, \textit{95\%CI}{[}0.02, 0.10{]}) and autonomy (\textit{b} = 0.04,
\textit{t}(1,598) = 2.12, \textit{p} = 0.034, \textit{95\%CI}{[}-0.01,
0.08{]}) each also predicted positive outgroup attitudes independently.
This being said, the regression coefficient for the core need was three
times larger (with all scaling being equal). For full results see Table
\ref{tab:robustnessTblLong} and Figure \ref{fig:Robustness} as well as
Online Supplementary Material A. Across all three studies, psychological
need fulfillment, thus, remained a robust and flexible predictor of
positive outgroup attitudes.
