\documentclass[10pt,a4paper]{protocol}

% Change the page layout if you need to
\geometry{left=1cm,right=9cm,marginparwidth=6.8cm,marginparsep=1.2cm,top=1cm,bottom=1cm}

% Change the font if you want to.

% If using pdflatex:
\usepackage[utf8]{inputenc}
\usepackage[T1]{fontenc}
\usepackage[default]{lato}
\usepackage{booktabs}
\usepackage{hyperref}
\usepackage[utf8]{inputenc}
\usepackage[english]{babel}
\usepackage{csquotes}
\usepackage{fancyhdr}
\usepackage{lastpage}
\usepackage{tcolorbox}
\usepackage[style=apa, sortcites=true, sorting=nyt, backend=biber, natbib=true, uniquename=false, uniquelist=false, useprefix=true]{biblatex}
\usepackage{authblk}
\usepackage{setspace,caption}
\usepackage{subcaption}
\usepackage{enumitem}
\usepackage{lipsum}
\usepackage{soul}
\usepackage{xcolor}
\usepackage{fourier}
\usepackage{stackengine}
\usepackage{scalerel}
\usepackage{fontawesome}
\usepackage[normalem]{ulem}
\usepackage{longtable}
\usepackage{amsmath}
\usepackage{afterpage}
\usepackage{float}
\usepackage{titling}
\usepackage{censor}
\usepackage{pdfpages}
\usepackage{bashful}
% needed for the fact box label
\usepackage{xparse}
% this allows for the easy resize box
\usepackage{graphicx}

\tcbuselibrary{skins,breakable}
\usetikzlibrary{shadings,shadows}
 
% Turn on the style
\pagestyle{fancy}
\fancyhead{}
\fancyfoot{}

\fancyfoot[C]{%
    \vspace{-3em}
    \makebox[19cm][c]{%
        \thepage}
    }
\renewcommand{\headrulewidth}{0pt}

% If using xelatex or lualatex:
% \setmainfont{Lato}

\hypersetup{
    colorlinks=true,
    urlcolor=blue
}

% These commands create shortcuts for the nutrition label
% Modified from: https://tex.stackexchange.com/questions/118600/how-can-i-create-a-nutrition-facts-label
\newlength{\NFwidth}
\setlength{\NFwidth}{4.5in}

\NewDocumentCommand{\NFelement}{mmm}{\normalsize\textbf{#1} #2\hfill #3}
\NewDocumentCommand{\NFline}{O{l}m}{\footnotesize\makebox[\NFwidth][#1]{#2}}

\NewDocumentCommand{\NFentry}{sm}{%
  \makebox[.5\NFwidth][l]{\normalsize
    \IfBooleanT{#1}{\makebox[0pt][r]{\textbullet\ }}%
    #2}\ignorespaces}
\NewDocumentCommand{\NFtext}{+m}
 {\parbox{\NFwidth}{\raggedright#1}}

\newcommand{\NFtitle}{\multicolumn{1}{c}{\huge\bfseries Dataset Sheet}}

\newcommand{\NFRULE}{\midrule[5pt]}
\newcommand{\NFRule}{\midrule[2pt]}
\newcommand{\NFrule}{\midrule}

% Change the colours if you want to
\definecolor{Bright}{HTML}{58318f}
\definecolor{Green}{HTML}{319866}
\definecolor{Black}{HTML}{111111}
\definecolor{LightGrey}{HTML}{515c50}
\colorlet{heading}{MidnightBlue}
\colorlet{accent}{NavyBlue}
\colorlet{emphasis}{Black}
\colorlet{body}{LightGrey}

% Change the bullets for itemize and rating marker
% for \risk if you want to
\renewcommand{\itemmarker}{{\small\textbullet}}
\renewcommand{\ratingmarker}{\faSpinner}

% add freetext option
\newlength{\rulewidth}
\setlength{\rulewidth}{1pt}
\newlength{\ruleandnamegap}
\setlength{\ruleandnamegap}{.4\baselineskip}
\newcommand{\namefont}{\tiny}
\newcommand{\ruleandname}[2]{%
  \par\noindent
  \rule{#2}{\rulewidth}\par
  \vspace{\dimexpr-\baselineskip+\ruleandnamegap}
  \noindent{\namefont #1}\par
  \addvspace{\baselineskip}
}
\newcommand{\rulewithattribute}[3][-.1\baselineskip]{%
  \par\noindent
  \rule[#1]{1cm}{\rulewidth} #2 \rule[#1]{#3}{\rulewidth}\par
  \vspace{\dimexpr-\baselineskip-#1+\ruleandnamegap}
  \noindent{\namefont Score}\par
  \addvspace{\baselineskip}
}

\newcommand\category[2]{
{\Large\bfseries\color{emphasis} \vspace{0.25em} #1 \hspace{0.5em} #2 \\ [-0.6em] \rule{\textwidth}{0.4pt} \vspace{0.25em}}
}

% box
\newenvironment{myblock}[1]{%
    \tcolorbox[beamer,%
    noparskip,breakable,
    colback=LightBlue,colframe=DarkBlue,%
    colbacklower=DarkBlue!75!LightBlue,%
    title=#1]}%
    {\endtcolorbox}
    
\newtcolorbox{topbot}[1][]{empty, notitle, sharp corners, 
borderline north={1pt}{0pt}{black},
borderline east={1pt}{0pt}{black},
borderline south={1pt}{0pt}{black},
borderline west={5pt}{0pt}{black},
#1}

%% links.bib contains references
\addbibresource{links.bib}

% create PDF of title page
\immediate\write18{pdflatex CodingProtocolTitlePage}

\begin{document}

% fix page numbering with title page
\setcounter{page}{0}

\includepdf{CodingProtocolTitlePage.pdf}

\name{Documentation: \par Coding Protocol}
\tagline{For: `Need Fulfillment During Intergroup Contact: Three Experience Sampling Studies'}
\made{March 10 2023}
%%\logo{6.8cm}{images/MJ Research Design}
%%\logo{6.8cm}{example-image}
%\logo{6.8cm}{images/rugLogoBlack}
\logo{6.8cm}{images/rugLogoBlackMasked}


\docinfo{%
  % can add more \addedtopeople
  %\madeby{Jannis Kreienkamp}{j.kreienkamp@rug.nl}{March 10, 2023}
  %\addedto{Maximilian Agostini}
  %\addedto{Laura F. Bringmann}
  %\addedto{Peter de Jonge}
  %\addedto{Kai Epstude}
  \madeby{[masked for peer review]}{[masked for peer review]}{March 10, 2023}
  \addedto{[masked for peer review]}
  \addedto{[masked for peer review]}
  \addedto{[masked for peer review]}
  \addedto{[masked for peer review]}
}


\purpose{
	The purpose of this coding protocol is to assess the goal-directedness of participants' free-text responses regarding their main goal during interactions with outgroup members. The coding will capture two dimensions: practical needs and underlying psychological needs. Two independent coders will evaluate each response, and the dimensions will be assessed separately to ensure the validity and reliability of the coding process.
} % add a short description of the purpose for this protocol


%% Make the header extend all the way to the right, if you want.
\begin{fullwidth}
\makeheader
\end{fullwidth}

%% Provide the file name containing the sidebar contents as an optional parameter to \need.
%% You can always just use \marginpar{...} if you do
%% not need to align the top of the contents to any
%% \need title in the "main" bar.
\need[margins/materials]{Goal Directedness}

\category{1}{Introduction}

The goal of this coding protocol is to assess the goal-directedness of free-text responses obtained from recent migrants regarding their interactions with outgroup members. Each coder should independently read and analyze each free-text response. For each response, it is your task to determine whether the response presents a practical need and whether it presents a psychological need. You will be asked to assign a code to each dimension based on the presence or absence of the respective need type. We will first explain the coding procedure and will then provide more detailed information on the coding criteria for each dimension.

\vspace{1em}

\category{2}{Coding Procedure}

\begin{enumerate}
    \item Familiarize yourself with the research objectives: Gain a clear understanding of the study's goals and the specific practical and psychological needs, as well as non-goal-directed actions. This understanding will guide the coding process.
    \item Read and understand the research question prompt: Review the survey question that asks participants about their main goal during the interaction. Ensure a thorough understanding of the wording and context to accurately identify responses that indicate underlying needs.
    \item Assign a rating for each dimension (0, 1, or 2) based on the criteria provided.
    \item Rate each dimension independently without being influenced by the rating given to the other dimension.
    \item Use the examples provided as references, but do not limit your assessment to these examples alone.
    \item If a response contains multiple goals, consider the most prominent or emphasized goal in the rating.
    \item In case of ambiguity or uncertainty, code conservatively and opt for a lower rating.
    \item Document your ratings and any notes or comments regarding challenging cases or exceptional responses within the \textit{comment} field.
    \item Regular communication with the research team is encouraged to clarify any uncertainties and maintain consistency in coding.
\end{enumerate}

\newpage
\vspace*{3em}
\textbf{Coding Agreement and Disagreement:}
\begin{itemize}
    \item After coding all responses, we compare the codes assigned by each coder.
    \item The PI will calculate the inter-coder reliability using appropriate statistical measures (e.g., Cohen's kappa) to determine the agreement between coders.
    \item The coders will resolve any disagreements through discussion and consensus until a final set of codes is reached.
    \item If necessary, we will refine the coding protocol based on the discussion and repeat the coding process to ensure consistency and validity.
\end{itemize}

\vspace{1em}

\category{3}{Coding Criteria}

\step{practical\_need}{Practical Needs Dimension}{needs}
\textit{Practical needs refer to specific, tangible goals or tasks that participants aimed to accomplish during the interaction. These instrumental goals are usually observable, concrete, and often centered on external outcomes. Individuals strive to fulfill these needs to navigate their daily lives efficiently and achieve desired results. Examples of practical needs include engaging in activities such as acquiring resources, completing tasks, and addressing immediate challenges. These needs are closely tied to instrumental and utilitarian considerations, highlighting the pursuit of external rewards, resource acquisition, or problem-solving. It is your task to assess the extent to which the free-text response captures a practical need. Practical needs can include activities such as getting coffee together, working on an assignment, discussing a project, or seeking information.}\\
\vspace{0.5em}
\textit{\textbf{Options:}} \\
\textit{Either there is no practical need to be identified (1), a vague or general mention of a practical need without specific details (2), or a clear mention of a practical need (3).}\\
\vspace{0.5em}
\textit{\textbf{Additional clarification no practical need:}}\\
\textit{Given that many aspects of the human life a driven by needs or at least are recalled or perceived as need-driven, it can be difficult to identify situational reports that do not include a practical need, motive, or goal. Examples of responses that are not goal-directed would be:
\begin{itemize}[nosep]
    \item[--] "I wasn't seeking any particular outcome, I simply wanted to be present in the moment and enjoy the interaction."
    \item[--] "There wasn't a specific objective I aimed to achieve, I just wanted to see where the conversation would take us."
    \item[--] "I wasn't focused on achieving a particular outcome, my intention was to engage in a spontaneous and genuine conversation."
\end{itemize}
These examples reflect a lack of specific goals or intentions during intergroup interactions. These examples express a more open and non-directed approach, where the primary focus was on the process of interaction itself rather than aiming for specific outcomes or results.}

\vspace{0.5em}
\textit{\textbf{Codes:}}
\vspace{0.5em}

\marginpar{
\vspace{-4em}
Practical Need Explanations

\begin{tabular*}{6cm}{>{\raggedright\arraybackslash}p{0.03\linewidth} 
>{\raggedright\arraybackslash}p{0.75\linewidth}}
%\begin{tabular}
\toprule
\# & Examples / Explanation\\
\midrule
0 & ``We just exchanged greetings and went our separate ways.'' \\
1 & ``We had a task to complete.'' \\
2 & ``We met to discuss the upcoming presentation.'' \\
\bottomrule
\end{tabular*}
}
\begin{itemize}
	\item \parbox[t]{6cm}{No practical need \dotfill[0]}
	\item \parbox[t]{6cm}{Vague or general mention \dotfill[1]}
    \item \parbox[t]{6cm}{Clear mention \dotfill[2]}
\end{itemize}

\vspace{1em}
\divider

\newpage
\vspace*{3em}
\divider
\vspace{1em}

\step{psychological\_need}{Psychological Needs Dimension}{needs}
\textit{Psychological needs refer to underlying motives or desires that are more abstract and relate to personal fulfillment and well-being. In contrast to practical needs, psychological needs delve into the subjective and internal aspects of human experiences. These needs pertain to emotions, social connections, and cognitive processes, reflecting individuals' quest for personal growth, well-being, and thriving in social relationships. It is your task to assess the extent to which the free-text response captures an underlying psychological need. Psychological needs can include motives such as sharing and bonding, competence and mastery, feeling valued or respected, or seeking emotional support.\\
\vspace{0.5em}
\textbf{Options:} \\
Similar to the practical need coding, responses might either include no underlying psychological need identified (1), include a vague or general mention of a psychological need without specific details (2), or present a clear mention of a psychological need (3).\\
\vspace{0.5em}
\textbf{Additional clarification no psychological need:}\\
Similar to the practical needs, it can be difficult to identify situational reports that do not include a any underlying psychological need, motive, or goal. Examples of responses that are not goal-directed would be:
\begin{itemize}[nosep]
    \item[--] "I didn't have any specific expectations or goals in mind, I just wanted to see how the conversation would unfold."
    \item[--] "I didn't have a specific goal in mind, I was open to any potential insights or connections that might emerge."
\end{itemize}
These examples reflect a lack of specific goals or intentions during intergroup interactions. If in doubt, try to code conservatively (i.e., try to not infer a need that might not be present) and make a comment if necessary.
}
\vspace{0.5em}
\textit{\textbf{Codes:}}
\vspace{0.5em}

\marginpar{
\vspace{-4em}
Psychological Need Explanations

\begin{tabular*}{6cm}{>{\raggedright\arraybackslash}p{0.03\linewidth} 
>{\raggedright\arraybackslash}p{0.75\linewidth}}
%\begin{tabular}
\toprule
\# & Examples / Explanation\\
\midrule
0 & ``I just wanted to pass the time.'' \\
1 & ``I wanted to connect with someone.'' \\
2 & ``I wanted to share experiences and build rapport.'' \\
\bottomrule
\end{tabular*}
}
\begin{itemize}
	\item \parbox[t]{8.5cm}{No psychological need \dotfill[0]}
	\item \parbox[t]{8.5cm}{Vague or general mention \dotfill[1]}
    \item \parbox[t]{8.5cm}{Clear mention \dotfill[2]}
\end{itemize}

\vspace{1em}
\divider
\vspace{1em}

\step{Comment}{Any necessary comments of the coder.}{context}
\textit{This can include additional information about ambiguous, uncertain, or exceptional codings.}
\vspace{1.5em}
\ruleandname{character string}{10cm}
\divider

% --------------------------------
%           Last Page
% --------------------------------

\clearpage

\need[margins/otherinfo]{Links}

\nocite{*}

\printbibliography[heading=none]
%[reference to GitHub repository masked for peer review]

%[reference to DataVerse repository masked for peer review]

\divider


%% If the NEXT page doesn't start with a \need but you'd
%% still like to add a sidebar, then use this command on THIS
%% page to add it. The optional argument lets you pull up the
%% sidebar a bit so that it looks aligned with the top of the
%% main column.
% \addnextpagesidebar[-1ex]{page3sidebar}


\end{document}
