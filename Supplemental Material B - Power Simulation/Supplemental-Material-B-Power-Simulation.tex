% define document type (i.e., template. Here: A4 APA manuscript with 12pt font)
\documentclass[man, 12pt, a4paper]{apa7}

% add packages
\usepackage[american]{babel}
\usepackage[utf8]{inputenc}
\usepackage{csquotes}
\usepackage{hyperref}
\usepackage[style=apa, sortcites=true, sorting=nyt, backend=biber, natbib=true, uniquename=false, uniquelist=false, useprefix=true]{biblatex}
\usepackage{authblk}
\usepackage{graphicx}
\usepackage{setspace,caption}
\usepackage{subcaption}
\usepackage{enumitem}
\usepackage{lipsum}
\usepackage{soul}
\usepackage{xcolor}
\usepackage{fourier}
\usepackage{stackengine}
\usepackage{scalerel}
\usepackage{fontawesome}
\usepackage[normalem]{ulem}
\usepackage{longtable}
\usepackage{amsmath}
\usepackage{afterpage}
\usepackage{float}
\usepackage{titling}
\usepackage{censor}
\usepackage{tcolorbox}

% formatting links in the PDF file
\hypersetup{
pdfpagemode={UseOutlines},
bookmarksopen=true,
bookmarksopenlevel=0,
hypertexnames=false,
colorlinks   = true, %Colours links instead of ugly boxes
urlcolor     = blue, %Colour for external hyperlinks
linkcolor    = blue, %Colour of internal links
citecolor   = cyan, %Colour of citations
pdfstartview={FitV},
unicode,
breaklinks=true,
}

% language settings
\DeclareLanguageMapping{american}{american-apa}

% add reference library file
\addbibresource{../references.bib}

% Title and header
\title{Supplemental Information B: Power Simulation}
\shorttitle{SI B: Power Simulation}
%\author{Jannis Kreienkamp, Laura F. Bringmann, Raili F. Engler, Peter de Jonge, Kai Epstude}
\author{[authors masked for peer review]}

% set indentation size
\setlength\parindent{1.27cm}

% adapt table and figure labels
\setcounter{equation}{0}
\setcounter{figure}{0}
\setcounter{table}{0}
\setcounter{page}{1}
\makeatletter
\renewcommand{\theequation}{S\arabic{equation}}
\renewcommand{\thefigure}{S\arabic{figure}}
\renewcommand{\thetable}{S\arabic{table}}

% Start of the main document:
\begin{document}

% add title information (incl. title page and abstract)
\begin{titlepage}
	{\noindent\Large Supplementary Information for \par}
	\vspace{0.5cm}
	{\noindent\Large Need Fulfillment During Intergroup Contact: Three Experience Sampling Studies\par}
	\vspace{1.5cm}
	{\noindent\LARGE\bfseries \thetitle \par}
	\vspace{2cm}
	{\noindent\Large\itshape \theauthor \par}
	\vfill
	%\noindent Corresponding Author: Jannis Kreienkamp\par
	%\noindent E-mail: j.kreienkamp@rug.nl\par
	\noindent Corresponding Author: [masked for peer review]\par
	\noindent E-mail: [masked for peer review]\par
	\vfill

    % Bottom of the page
	{\noindent Last updated: \today\par}
\end{titlepage}

% add title again on page 1 (after title page)
\begin{center}
   \textbf{\thetitle} 
\end{center}

This supplementary information documents the power simulations we conducted after Study 1. As part of our open supplemental materials, we share the full RMarkdown file which includes transparent and reproducible analysis code. We rendered the RMarkdown as an interactive HTML report, which we host as part of our open GitHub repository. We recommend the rendered version for almost all readers (full R code is also available via the rendered version).

\vspace{.5cm}
\begin{tcolorbox}
    \vspace{0.2cm} \centering 
    \href{https://janniscodes.github.io/intergroup-contact-needs/Supplemental-Material-B-Power-Simulation}{https://janniscodes.github.io/intergroup-contact-needs/Supplemental-Material-B-Power-Simulation}
    \vspace{0.2cm} 
\end{tcolorbox}

For readers interested in the raw files, the raw RMarkdown file is available in our OSF repository \citep[see][]{KreienkampMasked2022a} and can also be accessed as part of the full GitHub repository \citep[][]{KreienkampMasked2022}.

\printbibliography

\end{document}