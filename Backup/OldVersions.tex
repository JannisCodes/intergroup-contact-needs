\section{Introduction}

% Problem: NEEDS REFORMULATION (in line with aims of present study. should include main aims [see email 2021-11-25])
In its most essential interpretation, the intergroup contact hypothesis postulates that frequent and positive contact with an out-group reduces prejudice and increases favorable attitudes towards the other group \citep[e.g.,][]{Hewstone1996, Pettigrew1998}. A key condition for these contact benefits, has been that the interaction is indeed perceived as positive --- making the interaction quality a crucial mechanism of inter-group contact \citep[e.g.,][]{MacInnis2015}. It is widely accepted that equal status, common goals, collaboration, and structural support during the interaction form the optimal conditions for a positive contact \citep[Allport's Optimal Contact conditions,][]{Allport1954b, Pettigrew1969}. And indeed a major meta-analytic review showed, that intergroup contact benefits were larger when Allport's conditions were met. However, the meta-analysis also showed that contact resulted in more positive intergroup relations even when Allport's conditions were not met \citep[][]{Pettigrew2006}\footnote{It should be noted here that following Allport's original conditions, several additional conditions of optimal contact were proposed \citep[for a critical discussion see][]{Pettigrew1986}. Similarly, since the early 21\textsuperscript{st} century many psychological processes during intergroup contact were examined \citep[e.g. see,][]{Paolini2021}. Among others, researchers have for example explored different forms of social categorizations \citep[][]{Pettigrew1998}, the salience of social categories \citep[][]{Brown2005}, intimacy \citep[e.g.,][]{Marinucci2021} and attachment \citep[e.g.,][]{Tropp2021}, threat and intergroup anxiety \citep[e.g.,][]{Stephan2008, Paolini2004}, and to a lesser extent knowledge about the other group \citep[][]{Pettigrew2008c}. However, all these advances share the underlying criticism that they are not geared towards understanding when and why an interaction is perceived as positive.}. It, thus, remains unclear when and why exactly an interaction is perceived as positive and what factors (beyond Allport's conditions) explain positive contact effects.

% Problem: focus on everyday life interactions: 
As one of the most researched and practically implemented theories of social psychology, intergroup contact has received a vast amount of empirical attention \citep[][]{Dovidio2017}. Several meta-analytic reviews found that positive intergroup contact reduces prejudice and increases positive attitudes in experimental and cross-sectional studies \citep[][]{Tropp2005, Pettigrew2006, Davies2011}. Later meta-analyses have shown that intergroup contact interventions outside the lab show similarly positive effects \citep[][]{Beelmann2014, Lemmer2015}\footnote{Additional reviews have shown similar effects for online contact \citep[][]{White2020}, as well as during several more indirect forms of contact \citep[][]{Miles2014, Zhou2019, Harwood2021}}. \hl{However, an increasing number of reviews and perspective papers has pointed out that in order to truly understand intergroup contacts we need more fine grained data of how intergroup contacts unfolded in people's everyday lives} \citep[e.g.,][]{MacInnis2015}. 

This criticism of understudying the real-life mechanisms of intergroup contact, can be separated into two interconnected issues. Firstly, the focus on lab experiments, interventions, and cross-sectional survey designs are not aligned with the experiences of people in everyday life and the theories proposed around these natural interactions. Secondly, this methodological disconnect has led to an underdeveloped understanding of when and why an interaction is positive in everyday life. 

We propose to test a motivational mechanism using a daily dairy study design.



Method:

so far the only studies that have focused on interaction specific mechanisms have been artificial lab studies or structured interventions. 

Whenever natural interactions were assessed they were assessed in cross-sectional recall studies.

Method that brings these together is extensive longitudinal: Allows us to assess event-specific mechanism while being close to the natural event. 

--------- 

However, an increasing number of reviews and perspective papers has pointed out these broader findings might not generalize to the more common everyday interactions.

that these broader findings might not be useful if we want to understand how such intergroup contact evolves/works/functions/unfold during natural interactions in everyday life. 

However, an increasing number of reviews and perspective papers has pointed out that in order to truly understand intergroup contacts ...

However, an increasing number of reviews and perspective papers has pointed out that in order to move forward ...

However, virtually no research 

---------

... we need to understand/investigate how intergroup contacts unfolded in people's everyday lives

... we need to understand/investigate what actually makes for a positive intergroup contact in real life.

... we need more fine grained data of how intergroup contacts unfolded in people's everyday lives \citep[e.g.,][]{MacInnis2015}.

... we need to gain a deeper understanding of the psychological mechanisms .

---------


\section{Intergroup Contact Mechanisms}
\faQuestionCircle\ \textit{Not sure whether we need this section.}

\faExclamationCircle\ \textit{This section has become way too complex (i.e., too many arguments to get to our own proposed work). I think we ultimately need to introduce the concept of interaction quality and that we don't necessarily know why interactions are perceived as positive, therefore we propose that we can look towards the motivational literature for this. The difficulties started for me when I tried to give credit to past theoretical work (which I thought we need to acknowledge in some way).}

% Intergroup contact = Robust and strong effect (from meta-analyses) → much theoretical interest
As one of the most researched and practically implemented theories of social psychology, intergroup contact has received a vast amount of empirical, and theoretical attention \citep[][]{Dovidio2017}. Several meta-analytic reviews have found moderately sized positive effects of intergroup contact \citep[][]{Tropp2005, Pettigrew2006, Davies2011} and later meta-analyses have additionally shown that intergroup contact is also effective outside the lab \citep[][]{Beelmann2014, Lemmer2015}, during online contact \citep[][]{White2020}, as well as during several more indirect forms of contact \citep[][]{Miles2014, Zhou2019, Harwood2021}. With such a persistent and meaningful effect, researchers have long proposed theoretical models to understand these intergroup contact effects. 

% what makes a 'positive contact' has received much theoretical interest. (1) conditions of positive contact, then (2) contact quality. But why is it perceived as positive? 
Interestingly, one concept that has arguably remained at the center of theoretical debates throughout the decades has been the role of 'positive' contact. In his now famous chapter on 'the effects of contact', Gordon Allport (\citeyear{Allport1954b}) set out to summarize under what circumstances an interaction between groups tends to be positive, leading to favourable attitudes and reduced prejudice. In the years that followed, several additional conditions were proposed under which a positive intergroup contact would occur \citep[for a critical review see][]{Pettigrew1986}. As a counter-movement to this rather restrictive approach, with the beginning of the 21\textsuperscript{st} century attention had shifted towards investigating the psychological mechanisms of intergroup contact \citep[e.g. see,][]{Paolini2021}. With this shift in focus, the idea of 'positive contact' has seemingly seen a renaissance, where the adjective 'positive' is now not merely an evaluation of the outcome, but rather a feature of the interaction itself. As such, recent work has focused on the role of 'perceived interaction quality' \citep[e.g.,][]{Brown2007, Voci2003} or 'interaction valence' \citep[e.g.,][]{Tropp2016, Barlow2012}, to show that only if an interaction is perceived as positive will it lead to more favourable intergroup relations. And inversely interactions that are perceived as negative have the strong propensity to harm outgroup attitudes. However, relatively little research has investigated why interactions are perceived as positive.

% many moderators and mediators but none of these necessarily leads to higher interaction quality. Solution: look at need fulfillment (motivational literature)
Thus although a sizeable number of moderators and mediators of intergroup contact have been proposed, their effects might be undermined if they do not result in a 'positive' contact. As an example, important moderators and mediators that have been identified are different forms of social categorizations \citep[][]{Pettigrew1998}, the salience of social categories \citep[][]{Brown2005}, intimacy \citep[e.g.,][]{Marinucci2021} and attachment \citep[e.g.,][]{Tropp2021}, threat and intergroup anxiety \citep[e.g.,][]{Stephan2008, Paolini2004}, and to a lesser extent knowledge about the other group \citep[][]{Pettigrew2008c}. However, there is a discernible disconnect between these mechanisms and perceptions that an interaction is positive. That is to say that, just because someone learns something new about the other group, or reduces their threat perceptions does not mean that such an interaction is perceived as positive. Or inversely, an interaction can, for example, be very intimate but perceived as negative. If that were the case, there is a reasonable chance that such an interaction might not translate into improved outgroup attitudes \citep[e.g.,][]{Barlow2012}. We propose that we may look to the literature on motivation and psychological needs to understand when and why exactly an interaction is perceived as positive.


ALTERNATIVE: One potential remedy might come from looking to the literature on motivation and psychological needs to understand when and why exactly an interaction is perceived as positive. 