As with the main analyses, full surveys are available in our OSF
repository \citep{KreienkampMasked2022a} and the full data description
is available in Online Supplementary Material A. Correlations and
descriptive statistics of the included variables are available in
\tblref{tab:descrFullWide} and \tblref{tab:descrOutWide}.

\subsection{Additional Materials}

In addition to the measurement of whether or not participants had an
intergroup interaction and their situational core need fulfillment, we
also included a number of variables that allowed us to assess the
robustness of our results.

\subsubsection{Specific Psychological Needs}

In addition to the intergroup contact dummy and situational core need
reported in the main text, we included a common measure of three
self-determination theory needs \citep[see][]{Downie2008}. The
measurement was identical in all three studies. The items were
introduced either by ``\textit{During the interaction:}'' or
``\textit{This morning [/afternoon]:}'' and measured autonomy
(``\textit{I was myself.}''), competence
(``\textit{I felt competent.}''), and relatedness (without intergroup
contact ``\textit{I had a strong need to belong}''; with intergroup
contact: ``\textit{I shared information about myself.}'' and
``\textit{The other(s) shared information about themselves.}''). All
items were rated on a continuous slider scale from very little (-50) to
a great deal (+50).

\subsubsection{Interaction Intend}

To assess whether an interaction was accidental (vs.~planned), we asked
participants with a single item to report the extend to which ``The
interaction with -X- was accidental''. The respondents were asked to
report this context variable for all interactions they reported on using
a continuous slider ranging from ``not at all'' (0), through ``very
little'' (33) and ``somewhat'' (66), to ``a great deal'' (100). In all
studies the scale showed a right skew (\textit{mean} = 29.60,
\textit{sd} = 33.68).

\subsubsection{Goal-directedness}

To assess whether the need content (i.e., the motives) would impact the
effect of the need fulfillment experiences, we manually coded the topics
we extracted during the topic modeling on two dimensions of how much
they reflect a practical and a psychological goal-directedness. We chose
practical and psychological needs specifically as our dimensions to
account for differences in the types of needs that participants commonly
reported. With practical motives we refer to specific, tangible goals or
tasks that participants aimed to accomplish during the interaction.
These instrumental goals are usually observable, concrete, and often
centered on external outcomes, such as acquiring resources, completing
tasks, and addressing immediate challenges \citep[][]{}. With
psychological motives we refer to underlying motives or desires that are
more abstract and relate to personal fulfillment and well-being. In
contrast to practical needs, psychological needs delve into the
subjective and internal aspects of human experiences. These needs
pertain to emotions, social connections, and cognitive processes,
reflecting individuals' quest for personal growth, well-being, and
thriving in social relationships \citep[][]{}. Note that with this
approach any particular motive can include a practical and/or a
psychological goal-directedness but can also be classified as not having
any goal at all. The full coding protocol we developed with examples for
each of the codes is available in our Online Supplemental Material D.
After an initial training, each of the two coders independently coded
the 47 topics on the two dimensions, using one of three options each
(i.e., 0 = no goal, 1 = vague goal, 2 = clear goal). Inter-rater
reliability assessments showed that for both the practical as well as
the psychological needs, agreement was not optimal using the three
answer options (agreement practical = 0.72\%, agreement psychological =
0.78\%). However, most disagreements were, if a need was present,
whether that need was vague (1) or concrete (2). We, thus, collapsed
these two categories making the ratings binary (need absent vs.~need
present). With the simpler coding, inter-rater agreement for practical
needs (0.93\%) and the psychological need (0.96\%) were much more
reliable. Using Cohen's \(\kappa\) as our measure of inter-rater
reliability, we find that both the practical need coding
(\textit{Cohen's} \(\kappa\) = 0.80, 95\%CI{[}0.58, 1.00{]}) as well as
the psychological need coding (\textit{Cohen's} \(\kappa\) = 0.86,
95\%CI{[}0.67, 1.00{]}) were very good. We thus proceeded with this
collapsed coding. After resolving coder disagreements and merging the
codings back to the free-text responses, we found that a majority of
responses showed both a practical as well as a psychological need
(57.95\%) and only few responses had no goal at all (1.03\%) with the
remaining 41.02\% having either a practical or a psychological need only
(see Online Supplemental Material A for more detailed tables and
visualizations).

\subsubsection{Well-being}

We measured experienced well-being using a visual analog scale adapted
from \citet{davies2022}. Participants were asked to respond to the the
question ``How do you feel right now?'' using a continuous visual slider
ranging from ``very sad''(-100) to ``very happy'' (100). The well-being
ratings were generally normally distributed (\textit{mean} = 64.80,
\textit{sd} = 19.25).

\subsection{Results}

To build further confidence in our results, we assessed a number of
additional models that might offer alternative explanations. We will
discuss the results in sequential order --- in every case first
considering the a global test of the model across the three studies and
only then assessing whether the global three-level regression model
suppresses any important person-level variations within the studies.

\subsubsection{Contact specific}

We begin our robustness analysis by testing whether the effect of core
need fulfillment is specific to an actual outgroup contact, rather than
need fulfillment in general. For this, we analyzed the generalized
situational core need fulfillment (either during a contact or about the
daytime in general) and tested whether the effect differed during
experience sampling measurements with and without outgroup contacts. We
start this test by assessing the effect across all three studies, using
a three-level hierarchical model, where measurements are nested within
participants, and participants are nested within studies. In this
overall model, we found no main effect of core need fulfillment (random
slopes model, grand-mean standardized to account for all levels of
variance; \textit{b} = 0.62, \textit{t}( 3.187) = 2.89, \textit{p} =
0.058, \textit{95\%CI}{[} 0.20, 1.03{]}) but a significant interaction
effect of core need fulfillment and outgroup contact (\textit{b} = 2.44,
\textit{t}(4,663.172) = 8.62, \textit{p} \textless{} .001,
\textit{95\%CI}{[} 1.89, 3.00{]}; also see
\tblref{tab:robustnessTblLong} and \fgrref{fig:Robustness}). While the
three-level hierarchical model can be sensitive to scaling issues, this
already indicates that it is not key need fulfillment in general --- but
only key need fulfillment during an outgroup contact that predicts more
positive outgroup attitudes.

To ensure that the results are not affected by scaling issues (e.g.,
study-level variances suppressing person-level variances) or a similar
Simpson's paradox, we additionally assess the model within each of the
three studies. Within each of the three studies, the effects are more
pronounced, so that we also see a significant effect of core need
fulfillemnt (all \textit{b} \textgreater{} 0.06, all \textit{p}
\textless{} 0.005) as well as outgroup contact itself (all
\textbar{}\textit{b}\textbar{} \textgreater{} 1.81, all \textit{p}
\textless{} 0.034) but the interaction effect consistently remains the
most reliable predictor of outgroup attitudes (all
\textbar{}\textit{b}\textbar{} \textgreater{} 0.06, all \textit{p}
\textless{} 0.002, also see \tblref{tab:robustnessTblLong} and
\fgrref{fig:Robustness}). There is thus consistent evidence that need
fulfillment relates to outgroup attitudes for outgroup contacts in
particular but not need fulfillment in general.

\subsubsection{Interaction intend}

Secondly, to assess whether the need fulfillment mechanism affected by
whether the interaction was accidental or planned we ran an exploratory
moderation analysis using the participants' ratings of how much they
perceived the interaction as `accidental'. It should be noted that we
asked our participants to focus on most the important interaction (i.e.,
``\textit{The following questions will be about the interaction \underline{you consider most significant}.}'''\,'\,``;
emphasis as in original). We again start our analysis approach by
assessing the model across all three studies, using a three-level
hierarchical model. In this overall model, we retain the main effect of
core need fulfillment (random slopes model, grand-mean standardized to
account for all levels of variance; \textit{b} = 2.87, \textit{t}(
7.979) = 7.32, \textit{p} \textless{} .001, \textit{95\%CI}{[} 2.10,
3.64{]}) but neither contact intend nor the moderation effect affect the
results (all \textbar{}\textit{b}\textbar{} \textless{} 0.27 and all
\textit{p} \textgreater{} 0.225; see \tblref{tab:robustnessTblLong} for
full results).

We again sought to ensure that the results were not affected by scaling
issues by additionally assessing the interaction intentionality model
within each of the three studies. Within each of the three studies, the
effect of core need fulfillemnt became even clearer (all
\textbar{}\textit{b}\textbar{} \textgreater{} 0.13, all \textit{p}
\textless{} 0.006). But in none of the studies neither outgroup contact
intention nor the moderation effect explained a significant amount of
variance in outgroup attitudes (all \textbar{}\textit{b}\textbar{}
\textless{} 0.03 and all \textit{p} \textgreater{} 0.073; also see
\tblref{tab:robustnessTblLong}). There is thus consistent evidence that
need fulfillment is related to outgroup attitudes, even when taking the
intentionality of the interaction into account --- at least in our three
samples and with a focus on the most significant interactions.

\subsubsection{Well-being outcome}

Thirdly, to build a stronger case for the relevance of need fulfillment
to minority group members, we exploratorily assessed the effect of need
fulfilling outgroup interactions on self-reported well-being. We, thus,
re-ran our main analysis but substituted the outgroup attitudes outcome
with situational well-being. As with the previous robustness analyses,
we begin with a global three-level hierarchical model (across the three
studies). We find that need fulfillment during outgroup contacts,
indeed, has a similar effect on experienced well-being (random slopes
model, grand-mean standardized to account for all levels of variance;
\textit{b} = 3.50, \textit{t}(4.219) = 6.33, \textit{p} = 0.003,
\textit{95\%CI}{[} 2.42, 4.58{]}). We found the same result when we
assessed each of the three studies individually. In each of the studies
situational need fulfillment during the outgroup interaction was related
with higher well-being ratings by the participants (random slopes model,
centered within participants; all \textbar{}\textit{b}\textbar{}
\textgreater{} 0.10, all \textit{p} \textless{} 0.009). We, thus, find
consistent and meaningful evidence that need fulfilling outgroup
interactions also relate to higher everyday well-being.

\subsubsection{Need types}

\paragraph{overall}

Fourthly, to assess the role of different types of motives reported by
our participants, we added our coding of practical and psychological
goal-directedness as additional predictors to our base model. We thus
had core need fulfillment predicting outgroup attitudes while also
accounting for whether the reported motives were capturing practical
and/or psychological motives. We again ran a global model, across the
three studies first. We found that core need fulfillment remain a core
predictor of outgroup attitudes (random slopes model, grand-mean
standardized to account for all levels of variance; \textit{b} = 2.27,
\textit{t}( 48.194) = 2.72, \textit{p} = 0.009, \textit{95\%CI}{[} 0.63,
3.90{]}), even after accounting for different types of motives. None of
the motive types nor the moderation effects reached statistical
significance within the overall analysis (all
\textbar{}\textit{b}\textbar{} \textless{} 1.45 and all \textit{p}
\textgreater{} 0.073; see \tblref{tab:robustnessTblLong} for full
results).

\paragraph{Study follow-up}

When looking at the individual studies, we again saw that the effect of
core need fulfillemnt remained the on only clear effect (all
\textbar{}\textit{b}\textbar{} \textgreater{} 0.14, all \textit{p}
\textless{} 0.015). Additionally, in none of the studies neither motive
type dummies nor the moderation effect explained a significant amount of
variance in outgroup attitudes (all \textbar{}\textit{b}\textbar{}
\textless{} 1.53 and all \textit{p} \textgreater{} 0.144; also see
\tblref{tab:robustnessTblLong}). We, thus, find consistent evidence that
need fulfillment is related to outgroup attitudes, even when taking the
type of need into account --- at least in our three samples.

\subsubsection{Specific psychological needs}

In a final step, we checked whether during the interaction the core
situational need remains a meaningful predictor even when taking other
fundamental psychological needs into account. We again take a two-step
approach, starting with cross-study global three-level test and then
assessing the effects within the individual studies. Within the overall
model we find that across the studies core need fulfillment remained a
strong predictor of outgroup attitudes, even after controlling for the
three self-determination theory need (random slopes model, grand-mean
standardized to account for all levels of variance; \textit{b} = 1.88,
\textit{t}(2.705) = 4.75, \textit{p} = 0.022, \textit{95\%CI}{[} 1.11,
2.66{]}). Within this overall analysis, none of the self-determination
theory needs independently predicted outgroup attitudes to a
statistically significant extent (all \textit{p} \textgreater{} 0.095).
However, some of the effect sizes were largely comparable to that of the
core need fulfillment (all \textbar{}\textit{b}\textbar{} \textless{}
2.10, particularly that of relatedness fulfillment; see
\tblref{tab:robustnessTblLong} for full results).

When looking at the individual studies, we again saw that core need
fulfillemnt remained a consistent predictor of outgroup attitudes, even
after accounting for the self-determination theory need (all
\textbar{}\textit{b}\textbar{} \textgreater{} 0.06, all \textit{p}
\textless{} 0.033). However, across all three studies the fulfillment of
relatedness motives also emerged as a consistent predictor of outgroup
attitudes (all \textbar{}\textit{b}\textbar{} \textgreater{} 0.06, all
\textit{p} \textless{} 0.001). Additionally, in the larger studies 2 and
3 competence fulfillment was also related to more positive outgroup
attitudes (study 2: \textit{b} = 0.05, \textit{t}(841.8) = 2.43,
\textit{p} = 0.015, \textit{95\%CI}{[} 0.01, 0.10{]}, study 3:
\textit{b} = 0.06, \textit{t}(30.20) = 2.62, \textit{p} = 0.013,
\textit{95\%CI}{[} 0.01, 0.10{]}). None of the autonomy fulfillment
effects reached statistical significance nor did the competence
fulfillment during study 1 (see \tblref{tab:robustnessTblLong} for the
full results). In short, find that across our samples, relatedness
fulfillment (and to a smaller extend competence fulfillment) are
instrumental in understanding when an outgroup contact leads to more
positive outgroup attitudes. Importantly, even when considering these
effects situational core need fulfillment remains a strong and
consistent predictor of outgroup attitudes. In some cases, we even find
that core need fulfillment takes on some of the variance that would
otherwise be explained by the self-determination theory needs (see
Online Supplemental Material A).
