\section{Study 1}

The full surveys are available in our OSF repository
\citep{KreienkampMasked2022a} and the full data description is available
in Online Supplementary Material A. Correlations and descriptive
statistics of the included variables are available in Table
\ref{tab:descrFullWide} and Table \ref{tab:descrOutWide}.

\subsection{Methods}

\subsubsection{Materials}

\paragraph{General Psychological Needs}

In addition to the intergroup contact dummy and situational core need
reported in the main text, we included a common measure of three
self-determination theory needs \citep[see][]{Downie2008}. The items
were introduced either by ``\textit{During the interaction:}'' or
``\textit{This morning [/afternoon]:}'' and measured autonomy
(``\textit{I was myself.}''), competence
(``\textit{I felt competent.}''), and relatedness (without intergroup
contact ``\textit{I had a strong need to belong}''; with intergroup
contact: ``\textit{I shared information about myself.}'' and
``\textit{The other(s) shared information about themselves.}''). All
items were rated on a continuous slider scale from very little (-50) to
a great deal (+50).

\subsection{Results}

To build further confidence in our results, we assessed two additional
models that might offer alternative explanations. First, to ensure that
the effect of core need fulfillment is specific to an actual contact, we
compared the effect to core need fulfillment in situations without an
intergroup contact. For this, we analyzed the generalized situational
core need fulfillment (either during a contact or about the daytime in
general) and tested whether the effect differed during experience
sampling measurements with and without outgroup contacts. We found no
main effect of core need fulfillment (random slopes model; \textit{b} =
-0.10, \textit{t}(1,199) = -2.22, \textit{p} = 0.026,
\textit{95\%CI}{[}-0.18, -0.01{]}) but a significant interaction effect
of core need fulfillment and outgroup contact (\textit{b} = 0.13,
\textit{t}(1,199) = 4.31, \textit{p} \textless{} .001,
\textit{95\%CI}{[}0.07, 0.18{]}; also see Table
\ref{tab:robustnessTblLong} and Figure \ref{fig:Robustness}). Together
with a significant main effect of having an outgroup contact, this
indicates that it is not key need fulfillment in general --- but only
key need fulfillment during an outgroup contact that predicts more
positive outgroup attitudes.

In a final step, we controlled for other fundamental psychological needs
during the contact. We focus on the three commonly considered
self-determination needs (SDT): competence, autonomy, and relatedness.
We find that the core need fulfillment adds significantly above a model
with only the self-determination theory needs (random slopes models;
\(\chi^2\)(6, \textit{N} = 21) = 19.60, \textit{p} = 0.003). We also
find that next to relatedness (\textit{b} = 0.10, \textit{t}(361) =
4.04, \textit{p} \textless{} .001, \textit{95\%CI}{[}0.05, 0.14{]}), the
core need explains the most variance in outgroup attitudes after an
outgroup contact (\textit{b} = 0.09, \textit{t}(361) = 2.23, \textit{p}
= 0.026, \textit{95\%CI}{[}0.01, 0.18{]}). When compared to the model
with only the SDT needs, the core need fulfillment flexibly takes on
some of the explained variance of all three fundamental needs
(competence and autonomy needs turning non-significant; all \textit{b}
\textless{} 0.05, all \textit{p} \textgreater{} 0.544). For full results
see see Table \ref{tab:robustnessTblLong}, Figure \ref{fig:Robustness},
and Online Supplementary Material A. There is, thus, considerable
evidence lending confidence to the stability and relevance of
psychological need fulfillment as a predictor of positive outgroup
attitudes for natural intergroup contacts.

\section{Study 2}

\subsection{Methods}

\subsubsection{Materials}

\paragraph{General Psychological Needs}

The measurement of the self-determination needs was identical to Study
1.

\subsection{Results}

\subsubsection{Robustness}

Also for Study 2, we checked for alternative models. First, when
considering generalized situational core need fulfillment together with
whether an intergroup contact took place, we find that there is only a
minuscule main effect of core need fulfillment (random slopes model;
\textit{b} = -0.03, \textit{t}(4,849) = -1.34, \textit{p} = 0.181,
\textit{95\%CI}{[}-0.08, 0.02{]}) but a stronger interaction effect of
core need fulfillment and outgroup contact (\textit{b} = 0.06,
\textit{t}(4,849) = 3.03, \textit{p} = 0.002, \textit{95\%CI}{[}0.02,
0.10{]}). Together with a significant main effect of having an outgroup
contact (\textit{b} = 2.88, \textit{t}(4,849) = 3.71, \textit{p}
\textless{} .001, \textit{95\%CI}{[}1.36, 4.40{]}), this indicates that
it is not key need fulfillment in general --- but key need fulfillment
during an outgroup contact that predicts more positive outgroup
attitudes. This finding is consistent with the results of the previous
study, albeit with a slightly weaker effect (likely because of the large
number of measurements that did not include an outgroup interaction; For
full results see Table \ref{tab:intergroupNeedsTblLong} and for visual
comparison see Figure \ref{fig:AllportNeedFulfillment}).

In a final step, we again checked whether during the interaction the
core situational need remains a meaningful predictor even when taking
other fundamental psychological needs into account. We find that the
core need fulfillment adds significantly above a model with only the
self-determination theory needs (random slopes models; \(\chi^2\)(6,
\textit{N} = 108) = 22.90, \textit{p} \textless{} .001). We find that
the core need fulfillment explained the most variance in outgroup
attitudes after an outgroup contact (\textit{b} = 0.08, \textit{t}(823)
= 2.95, \textit{p} = 0.003, \textit{95\%CI}{[}0.03, 0.13{]}). When
compared to the model with only the SDT needs, the core need fulfillment
flexibly takes on some of the explained variance of all of the three
fundamental needs. However, different from the first study, relatedness
(\textit{b} = 0.07, \textit{t}(823) = 4.26, \textit{p} \textless{} .001,
\textit{95\%CI}{[}0.04, 0.10{]}) and autonomy (\textit{b} = 0.02,
\textit{t}(823) = 0.93, \textit{p} = 0.353, \textit{95\%CI}{[}-0.02,
0.08{]}) also predicted positive outgroup attitudes in this larger
sample. For full results see Table \ref{tab:intergroupNeedsTblLong},
Figure \ref{fig:Robustness}, and Online Supplementary Material A. This
means that also within this second sample, the fulfillment of
psychological needs during intergroup contact remained a key predictor
of positive outgroup attitudes, even when taking into account several
alternative models.

\section{Study 3}

\subsection{Methods}

\subsubsection{Materials}

\paragraph{General Psychological Needs}

The measurement of the self-determination needs was identical to Studies
1 and 2.

\subsection{Results}

\subsubsection{Robustness}

As with the previous two studies, we checked for alternative models of
the key need fulfillment. First, when considering generalized
situational core need fulfillment together with whether an intergroup
contact took place, we find that there is no significant main effect of
core need fulfillment (random slopes model; \textit{b} = 0.03,
\textit{t}(3,835) = 1.51, \textit{p} = 0.131, \textit{95\%CI}{[}-0.01,
0.06{]}) but a stronger interaction effect of core need fulfillment and
outgroup contact (\textit{b} = 0.17, \textit{t}(3,835) = 7.32,
\textit{p} \textless{} .001, \textit{95\%CI}{[}0.12, 0.21{]}). Together
with a significant main effect of having an outgroup contact (\textit{b}
= 5.41, \textit{t}(3,835) = 6.36, \textit{p} \textless{} .001,
\textit{95\%CI}{[}3.74, 7.08{]}), this indicates that it is not key need
fulfillment in general --- but key need fulfillment during an outgroup
contact that predicts more positive outgroup attitudes. This finding is
consistent with the results of the previous studies.

In a final step, we again checked whether during the interaction the
core situational need remains a meaningful predictor even when taking
other fundamental psychological needs into account. We find that the
core need fulfillment adds additional variance above a model with only
the self-determination theory needs (random slopes models; \(\chi^2\)(6,
\textit{N} = 70) = 100.20, \textit{p} \textless{} .001). We find that
the core need explains the most variance in outgroup attitudes after an
outgroup contact (\textit{b} = 0.15, \textit{t}(1,598) = 3.88,
\textit{p} \textless{} .001, \textit{95\%CI}{[}0.07, 0.22{]}). When
compared to the model with only the SDT needs, the core need fulfillment
flexibly takes on some of the explained variance of all of the three
fundamental needs. However, similar to the previous study, in this large
sample relatedness (\textit{b} = 0.05, \textit{t}(1,598) = 3.31,
\textit{p} \textless{} .001, \textit{95\%CI}{[}0.02, 0.09{]}),
competence (\textit{b} = 0.06, \textit{t}(1,598) = 2.73, \textit{p} =
0.006, \textit{95\%CI}{[}0.02, 0.10{]}) and autonomy (\textit{b} = 0.04,
\textit{t}(1,598) = 2.12, \textit{p} = 0.034, \textit{95\%CI}{[}-0.01,
0.08{]}) each also predicted positive outgroup attitudes independently.
This being said, the regression coefficient for the core need was three
times larger (with all scaling being equal). For full results see Table
\ref{tab:robustnessTblLong} and Figure \ref{fig:Robustness} as well as
Online Supplementary Material A. Across all three studies, psychological
need fulfillment, thus, remained a robust and flexible predictor of
positive outgroup attitudes.
